
This documentation aims at guiding the developer with OpenTURNS cross compilation for Windows target.\\

This documentation is separated into two main parts :
\begin{itemize}
\item[$\bullet$]  compile OpenTURNS under Linux for Windows target,
\item[$\bullet$]  validation and use of OpenTURNS on Windows.
\end{itemize}

\subsection{Linux side}

\subsubsection{Quick compilation guide}

OpenTURNS cross compilation is now quite straightforward:

\begin{itemize}
\item[$\bullet$] install Wine and NSIS (e.g. \verb|aptitude install wine nsis|)
\item[$\bullet$] fetch the last openturns-developers-windeps-x.y.tgz corresponding to your OpenTURNS version on http://sourceforge.net/projects/openturns/files/openturns/openturns-x.y, untar it in distro/windows/ folder:\\
\verb|cd openturns-src/distro/windows ; tar zxf openturns-developers-windeps-x.y.tgz|

Note: the given MinGW compiler is build for a x86\_64 Linux host.
\item[$\bullet$] launch compilation:\\
\verb|cd openturns-src/distro/windows ; make|

\item[$\bullet$] or launch compilation for a 64 bits target (the default is i686):\\
\verb|cd openturns-src/distro/windows ; make ARCH=x86_64|

\item[$\bullet$] or launch compilation for another python version (the default is 2.7):\\
\verb|cd openturns-src/distro/windows ; make PYBASEVER=2.6|

\item[$\bullet$] That's it. If you are lucky, it should produce 2 installers (.exe).\\
\end{itemize}

\subsubsection{Manual compilation guide}

The following sections are only useful to understand how cross compilation has been made.

\paragraph{Set cross compilation environment}

\subparagraph{MinGW \label{mingw-installation}}
To install MinGW, it is recommended to follow the web page \url {http://www.mingw.org/wiki/LinuxCrossMinGW} .

It is the official supported procedure for MinGW compilation, and will install GCC version 4.9.0.
It is recommended to use the installation scripts from the CVS repository.

\subparagraph{BLAS / LAPACK}

The BLAS / LAPACK library is cross-compiled using a MinGW-w64 cmake wrapper:

\begin{verbatim}
${_arch}-cmake \
    -DCMAKE_BUILD_TYPE=Release \
    -DBUILD_TESTING=OFF \
    ..
\end{verbatim}

\subparagraph{Pthreads\label{pthread-installation}}

The pthread library used is winpthreads library from MinGW-w64, see \url {http://mingw-w64.sourceforge.net/} .

\begin{verbatim}
../configure \
     --prefix=/usr/${_arch} \
     --host=${_arch} \
     --enable-static \
     --enable-shared \
\end{verbatim}


\subparagraph{dlfcn}

\begin{itemize}
\item[$\bullet$]  download the binary of dlfcn for Windows here :

\url{http://dlfcn-win32.googlecode.com/files/dlfcn-win32-shared-r11.tar.bz2}

\item[$\bullet$]  copy the library libdl.a and libdl.dll.a in the lib directory of MinGW.
\item[$\bullet$]  copy dlfcn.h in the include directory of MinGW.
\end{itemize}

\subparagraph{libxml2}

\begin{itemize}
\item[$\bullet$]  download the precompiled zip files of iconv zlib and libxml2 from \url{http://sourceforge.net/projects/gnuwin32/files/} (at this time mingw32-iconv-1.12-7.zip, mingw32-zlib-1.2.3-11.zip and mingw32-libxml2-2.7.2-4.zip).
\item[$\bullet$]  install the content of include/ and lib/ directories of iconv and zlip into the respective directory of MinGW
\item[$\bullet$]  decompress libxml2 in a separate directory (e.g. : /opt/mingw32-sharedlib/libxml2).
\item[$\bullet$]  modify libxml2.la file so that the iconv dependency becomes correct, e.g. :
\begin{verbatim}
replace the line :
dependency_libs=' -lz /usr/i686-pc-mingw32/sys-root/mingw/lib/libiconv.la -lws2_32'
by :
dependency_libs=' -lz /opt/mingw-3.4.5/lib/libiconv.la -lws2_32'

and the line :
libdir='/usr/i686-pc-mingw32/sys-root/mingw/lib'
by :
libdir='/opt/mingw32-sharedlib/libxml2/lib'
\end{verbatim}
\end{itemize}


\subparagraph{regex}

\begin{itemize}
\item[$\bullet$]  download mingw-libgnurx-2.5.1-bin.tar.gz and mingw-libgnurx-2.5.1-dev.tar.gz from the MinGW official website.

\item[$\bullet$]  decompress this two files in a same directory (e.g. /opt/mingw32-sharedlib/regex).
\end{itemize}


% -------------------------------------------------------------------------------------------------

\paragraph{Linux test Environment}

\subparagraph{WINE}

Install any WINE version (\url{http://www.winehq.org/}), for example, the one given by your Linux distribution.


\subparagraph{Windows shared libraries}

\begin{itemize}
\item[$\bullet$]  put the shared libraries of pthreads (pthreadGC2.dll), BLAS/LAPACK (lapack.dll and blas.dll), dlfcn (libdl.dll), libxml2 (libcharset-1.dll libiconv-2.dll libxml2-2.dll zlib1.dll), and regex (libgnurx-0.dll) in a directory where the PATH environment variable of WINE points to.
\item[$\bullet$]  put the shared library given by MinGW (mingwm10.dll) in this directory too.
\end{itemize}

Note : WINE's PATH can be modified in the file \textasciitilde{}/.wine/system.reg.


\subparagraph{Advice}

It is better to install OpenTURNS dependencies in directory without spaces (e.g. not in \emph{C:\textbackslash Program Files}).
The space between \emph{Program} and \emph{Files} can cause cumbersome problems.

\subparagraph{Ghostscript}
\begin{itemize}
\item[$\bullet$]  download and install ghostscript into WINE environment. The installer (e.g. gs864w32.exe) can be found here \url{http://sourceforge.net/projects/ghostscript/}. Launch the command like this : wine gs864w32.exe.
\item[$\bullet$]  add the path to gswin32c.exe to the PATH environment variable of WINE.
\end{itemize}

\subparagraph{R}
\begin{itemize}
\item[$\bullet$]  install R into WINE environment by using the standard Windows installer from the official site \url{http://cran.r-project.org}.
\item[$\bullet$]  add the path to R.exe to the PATH environment variable of WINE.
\end{itemize}

\subparagraph{R packages}

\begin{itemize}
\item[$\bullet$]  install the and rotR zip files with Rgui.exe  (menu packages => install R packages fom zip files).
\end{itemize}

The package rot\_1.4.5.tar.gz have been transformed to Windows packages with the website \url{http://win-builder.r-project.org}.

\subparagraph{Python}

\begin{itemize}
\item[$\bullet$]  download the 2.7 version of Windows python installer from the official site \url{http://www.python.org/download/}. Install python using this command : wine msiexec /i python-2.7.3.msi .
\item[$\bullet$]  add the path to Python.exe to the PATH environment variable of WINE.
\end{itemize}

\subparagraph{matplotlib module}

Install a matplotlib version compatible with Python2.7 (e.g.: matplotlib-1.2.1.win32-py2.7.exe) from matplotlib download section: \url{http://matplotlib.org/downloads.html}.

\paragraph{Compilation}

In order to cross-compile OpenTURNS :

First get the type of your computer in order to set the \verb|--build| configure settings :
\begin{verbatim}
export BUILD_MACHINE=`gcc -dumpmachine`
\end{verbatim}

The configuration step:
\begin{verbatim}
# adapt these following lines to your configuration:
ARCH=i686
TARGET = $(ARCH)-w64-mingw32

# OpenTurns paths
OT_SRC    ?= $(PWD)/../..
OT_BUILD  ?= $(OT_SRC)/build-$(TARGET)
OT_PREFIX ?= $(OT_BUILD)/install

PYBASEVER=2.7
PYTHON_PREFIX=$HOME/.wine/drive_c/Python$PYTHON_VERSION
R_PATH=$HOME/.wine/drive_c/R/R-2.9.0
MINGW_PREFIX=/opt/mingw32

# Python
PYBASEVER = 2.7
PYBASEVER_NODOT = $(shell echo $(PYBASEVER) | sed "s|\.||g")
PYTHON_EXECUTABLE=$(MINGW_PREFIX)/$(TARGET)/bin/python$(PYBASEVER_NODOT).exe
export PYTHONHOME := $(MINGW_PREFIX)/$(TARGET)
export PYTHONPATH := $(MINGW_PREFIX)/$(TARGET)/lib/python$(PYBASEVER_NODOT)

# launch as is :
$(TARGET)-cmake \
          -DCMAKE_TOOLCHAIN_FILE=toolchain-$(TARGET).cmake \
          -DCMAKE_VERBOSE_MAKEFILE=$(VERBOSE) \
          -DPYTHON_INCLUDE_DIR=$(MINGW_PREFIX)/$(TARGET)/include/python$(PYBASEVER_NODOT) \
          -DPYTHON_LIBRARY=$(MINGW_PREFIX)/$(TARGET)/lib/libpython$(PYBASEVER_NODOT).dll.a \
          -DPYTHON_EXECUTABLE=$(PYTHON_EXECUTABLE) \
          -DR_EXECUTABLE=$(R_PATH)/bin/R.exe \
          -DCMAKE_INSTALL_PREFIX=$(OT_PREFIX) \
          -DINSTALL_TESTS=$(INSTALL_TESTS_OPT) \
          $(OT_SRC)

\end{verbatim}

Debug symbols are stripped so that binaries are 3 times smaller:
\begin{verbatim}
$(TARGET)-strip --strip-unneeded $(OT_PREFIX)/bin/*.dll
$(TARGET)-strip -g $(OT_PREFIX)/lib/*.a
$(TARGET)-strip --strip-unneeded $(OT_PREFIX)/Lib/site-packages/*/*.pyd
\end{verbatim}

In the same shell, start the compilation :
\begin{verbatim}
# openturns compilation and installation
make; make install
\end{verbatim}

The validation : launch the following command :
\begin{verbatim}
# set the PATH to python.exe
PATH=$PATH:$PYTHON_PREFIX

make check && make installcheck
\end{verbatim}

\paragraph{How to create the installer}

Two installers are created using NSIS.
\begin{itemize}
\item[$\bullet$]   openturns-x.y-pyu.v-arch.exe installs OpenTURNS DLL and headers, and its dependencies. It is mainly for users that interact with OpenTURNS through Python.
\item[$\bullet$]   openturns-developers-x.y-arch.exe helps compiling OpenTURNS program and wrapper on Windows. It also permits to launch OpenTURNS checktests.
\end{itemize}

\subsection{Windows side}

\subsubsection{Install OpenTURNS manually}

To install OpenTURNS without installer (the following points are done automatically by the installer openturns-x.y-pyu.v-arch.exe) :

\begin{itemize}
\item[$\bullet$]   Copy the \emph{install} directory (created by the command make install) from Linux to Windows into the directory \emph{C:\textbackslash openturns}.

\item[$\bullet$]   Like with WINE, every DLL must be reachable (mingwm, pthread, BLAS/LAPACK, dlfcn, libxml2, regex and OpenTurns), and the programs must be installed : R with its packages, ghostscript, Python with the required modules.

On Windows, DLLs are searched in directories listed in the PATH environment variable. To set the PATH variable temporarily, hit on a DOS console :
\begin{verbatim}
set PATH=%PATH%;C:\openturns\bin;C:\openturns\lib\bin
echo %PATH%
\end{verbatim}

To set permanently the PATH variable :
configuration panel -> system -> tab "advanced" -> button "environment variable" -> list "system variable" -> modify PATH variable.
\end{itemize}


\subsubsection{Install OpenTURNS with a non-admin account}

Use OpenTURNS installer as usual.

OpenTURNS developer installer can be used too if you installed OpenTURNS in default directory. But MinGW and MSYS installation will need an administrator account.


\subsubsection{OpenTURNS validation}

To test OpenTURNS on Windows,\\

- if you have the OpenTURNS developer installer (openturns-developers-x.y.z.exe):
\begin{itemize}
\item[$\bullet$]   OpenTURNS should have been installed in default directory C:\textbackslash OpenTURNS
\item[$\bullet$]   install OpenTURNS developer with every checkboxes enabled.
\item[$\bullet$]   click on shortcuts : Start Menu -> OpenTurns -> Start-checktests.
\end{itemize}


- if you do not have the OpenTURNS installer :

\begin{itemize}
\item[$\bullet$]   install MinGW and MSYS
\item[$\bullet$]   install like with WINE : R with its packages, ghostscript, Python with the required modules.

\item[$\bullet$]   copy the \emph{install} directory (created by the command make install) from Linux to Windows into the directory \emph{C:\textbackslash openturns}.
\item[$\bullet$]   suppress the empty file openturns\textbackslash share\textbackslash openturns\textbackslash examples\textbackslash libOT-0.dll
(dead unix link).

\item[$\bullet$]   finally, from an msys shell, go to the examples directory

\begin{verbatim}
cd /c/OpenTURNS/share/openturns/examples/
\end{verbatim}

and launch the checktests :
\begin{verbatim}
export PRINTF_EXPONENT_DIGITS=2

./check_testsuite AUTOTEST_PATH="$PWD" OPENTURNS_CONFIG_PATH="$PWD/../../../etc/openturns"

export abs_srcdir="$PWD"

./installcheck_testsuite AUTOTEST_PATH="$PWD" \
OPENTURNS_NUMERICALSAMPLE_PATH="$PWD" \
OPENTURNS_WRAPPER_PATH="$PWD/../wrappers" \
OPENTURNS_CONFIG_PATH="$PWD/../../../etc/openturns"

PYTHON_VERSION=27
export examplesdir="$PWD"

./python_installcheck_testsuite AUTOTEST_PATH="$PWD"  \
OPENTURNS_NUMERICALSAMPLE_PATH="$PWD" \
OPENTURNS_WRAPPER_PATH="$PWD/../wrappers" \
OPENTURNS_CONFIG_PATH="$PWD/../../../etc/openturns" \
PYTHONPATH="$PWD/../../../lib/python$PYTHON_VERSION/site-packages"
\end{verbatim}
\end{itemize}


\subsubsection{OpenTURNS compilation examples}


\paragraph{Simple program\label{simple-program}}

Install MinGW from the official installer (provided by OpenTURNS developers installer). During the installation, choose the compiler g++.

In order to compile, g++ needs OpenTURNS headers and libraries.
If OpenTURNS is installed like this :
\begin{verbatim}
c:
`--openturns
|-- include
|   `-- openturns
|       `-- ...
|-- lib
|   |-- bin
|   |   |-- libOT.dll.a
|   |   `-- ...
`-- src
`-- mon_prog.cxx
\end{verbatim}

From a DOS console, compile with this command :
\begin{verbatim}
cd src
g++.exe mon_prog.cxx  -I..\include\openturns -L..\lib\bin -lOT -o mon_prog.exe
\end{verbatim}

An example is given in the directory openturns/share/openturns/examples/simple\_example.


\paragraph{Wrapper}

To compile an OpenTURNS wrapper on Windows, OpenTURNS developers installer must be installed.
An example of a wrapper for Windows can be found in the directory openturns/share/openturns/WrapperTemplates/mingw\_wrapper\_linked\_with\_C\_function (this example is installed by the developers installer).

In this directory, launch the compilation from an MSYS shell (or start the script ./build.sh) :
\begin{verbatim}
PATH=/c/MinGW/bin:$PATH
PATH=/c/msys/1.0/bin:$PATH
mkdir build; cd build
cmake .. -DOpenTURNS_DIR=/c/openturns --prefix="$PWD/install"
make && make install
\end{verbatim}

The test using this wrapper can be started :
\begin{verbatim}
./start-test.sh
\end{verbatim}
The test can be started too by using the test.py file.\\

The following points are automatically done by the OpenTURNS developers installer :

\begin{itemize}
\item[$\bullet$]   To compile a wrapper, MinGW, MSYS and msysDTK must be installed. These installers can be found on MinGW site.

\item[$\bullet$]   Then, the Pthread library must be installed in MinGW directory (like in paragraph \ref{pthread-installation})

\item[$\bullet$]   Wrappers scripts use the openturn-config command. These one is configured to be installed in c:\textbackslash openturns\textbackslash bin directory. If OpenTURNS is not installed in this directory, the prefix variable of openturns-config must be modified.
\item[$\bullet$]   If during the compilation of the wrapper, libtool cannot produce a dynamic library because it can not found shared library, check that the library is existing and that the corresponding .la file is correct.
\end{itemize}


\paragraph{Dev-C++}
Dev-C++ is an integrated development environment like Visual Studio.

Download the last Dev-C++ version.
The compilations options are the same with those of paragraph \ref{simple-program}.

Configure it so that is uses MinGW g++ 3.4.5. At this time, the linker fails with the Dev-C++ compiler (g++ 3.4.2).


\paragraph{Visual C++}

The ABI of C++ binaries produced by Visual C++ and g++ are not compatible (C ABI are compatible). ABI means Application Binary Interface.
Further informations can be found here : \url{http://chadaustin.me/cppinterface.html}.

\begin{itemize}
\item[$\bullet$] So if you need to link your program compiled with Visual C++ with OpenTURNS DLL, it is not possible.
But if you need to use only a small subset of the OpenTURNS C++ interface, one can use a workaround and make an had-hoc MinGW wrapper that wrap OpenTURNS C++ symbols to C symbols (C binaries are compatible between gcc and Visual C). The application compiled with Visual Studio will be able to interact with OpenTURNS through the C symbols of the wrapper. The following diagram explains this:
\begin{verbatim}
prog vc++
|
ABI C
|
hadhoc wrapper g++
|
ABI g++
|
OpenTURNS g++
\end{verbatim}

\item[$\bullet$] OpenTURNS wrapper (I mean wrapper that can be found there: openturns-src/wrappers/WrapperTemplates) are pure C.
So, OpenTURNS wrapper can be compiled with Visual Studio and linked to OpenTURNS.
\begin{verbatim}
OpenTURNS g++
|
ABI C
|
wrapper vc++
\end{verbatim}
\end{itemize}

\paragraph{Benchmark}

No official benchmark of OpenTURNS on Windows has been done, but windows version is slower than Linux one.


\subsection{Unresolved problems}


\subsubsection{Compilation of python modules}

In order to compile the python wrappers (Python -> OpenTURNS), the static library of python (libpython.a) must be included. Libtool does not allow to create DLL with static dependancies. Python wrappers are also created directly with the compiler without using libtool.

The symbols of the static library libpython.a are also in each python DLL. At OpenTURNS execution, until now, no problems have been reported.

Furthermore, if libtool succeeds to create a DLL, it does it by appending a "-0" to the filename. This is a problem, because python searches for a filename without -0 (e.g. \_stat.pyd and not \_stat-0.pyd).


\subsubsection{Compilation of OpenTURNS wrappers}

OpenTURNS wrappers must be compiled on Windows with the -lOT flag.

On Linux, this could cause problems (at execution, the wrappers symbols could be loaded twice).

On Windows, the compilation of dynamics libraries is different : when a DLL is compiled, every symbols must be present (through a DLL skeleton).

I do not know how to avoid this flag.

This remains a point to take care of. The problem seen on Linux should be tested on Windows.

\subsection{Resolved problems}

\begin{itemize}
\item[$\bullet$]   if \emph{NumericalMathFunction (exec external)} check test fails :

The temporary-dir element of openturns.conf could be misconfigured.

If OpenTURNS examples has been installed in a directory containing spaces, copy the files poutre\_external\_infile* to a directory without spaces and set the environment variable abs\_srcdir to this directory.

\item[$\bullet$]   if DLLs or programs are not found :

check your MSYS or Windows PATH environment variable.
\item[$\bullet$]   if OpenTURNS does not start from python interpreter and if the PYTHONPATH is correctly set :

check that the version of the python interpreter is the same as the version OpenTURNS has been compiled for.
\item[$\bullet$]   if a program is installed in C:\textbackslash Program Files and if it is not well detected,

reinstall it in directory without spaces in the name. The space between \emph{Program} and \emph{Files} can cause cumbersome problems.

\item[$\bullet$]   to modify the PATH variable of .wine/system.reg, no WINE process must be started. When a WINE process stops, it overwrites this files.

\item[$\bullet$]   libtool wrappers do not work on Ubuntu Intrepid. They work correctly on Mandriva 2009 32bits and Ubuntu Jaunty. If libtool wrappers do not work, modify ot-src/lib/config/test.am as described in this file.

\item[$\bullet$]   if the static libraries libgc2.a and libfrtbegin.a are included during the compilation, libtool produces only a static OpenTURNS library. These libraries are also disabled during cross-compilation.

\end{itemize}
