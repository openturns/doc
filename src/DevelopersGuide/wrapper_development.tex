% Copyright 2005-2016 Airbus-EDF-IMACS-Phimeca
% Permission is granted to copy, distribute and/or modify this document
% under the terms of the GNU Free Documentation License, Version 1.2
% or any later version published by the Free Software Foundation;
% with no Invariant Sections, no Front-Cover Texts, and no Back-Cover
% Texts.  A copy of the license is included in the section entitled "GNU
% Free Documentation License".


\subsection{Pure python wrappers}

Python wrappers aim to be an easy way for wrapping external code. The external code can be an analytical mathematical formula or a coupling involving several computational codes dedicated to the resolution of a very complex physical problem.

Python wrappers are not the best solution if your external code last less than a microseconds and if you need to resolve billions of points. In that case, for better performance, consider using the library wrapper. In any other cases, Python wrapper is the recommended choice. For further details on speed optimization see paragraph \ref{speedo}.

On OpenTURNS, two Python wrappers are available to wrap an external code:
\begin{itemize}
\item the PythonFunction is a simple monothreaded Python wrapper.
\item the DistributedPythonFunction is a Python wrapper than can launch code in parallel on the local machine or deploy it among several computers.
\end{itemize}
These two methods will be described in the following sections.


\subsubsection{PythonFunction}

A PythonFunction is a NumericalMathFunction where the \verb|_exec| or \verb|_exec_sample| function are launched in a Python interpreter.
Here is an example of how to implement it:

\inputscript{pythonfunction_point}

Some explanations of the code :
\begin{itemize}
\item[line 3-7] The \verb|compute_point| function constructs the out point from the in point. The in and out array size must correspond to the sizes set in the PythonFunction constructor. In this example, X will be an array of size 2 and Y must be an array of size 1. The output point can be a Python list, an \OT\ NumericalPoint or a Numpy array.
\item[line 9] Construct the PythonFunction by passing function reference.
\end{itemize}


The \verb|_exec_sample| function can be implemented to speedup the compute using vectorization on large sample. It receives an input sample and must return an output sample. For further details on speed optimization see paragraph \ref{speedo}. Here is an example using \verb|_exec_sample| function:

\inputscript{pythonfunction_sample}

The output sample can be a NumericalSample, a Python list of list or a Numpy array of array. This function is optional.
If \verb|_exec_sample| is not implemented and \OT\ must compute a sample, the \verb|_exec| function is called several time: once for each point of the sample. On contrary, if only \verb|_exec_sample| is given and a point must be computed, the point is inserted in a sample of size 1, computed through \verb|_exec_sample| and extracted from the result sample.


The PythonFunction is quite simple to use. When used with coupling\_tools module, it can wrap external program easily too.


\subsubsection{External code coupling tools}


\paragraph{Simple example}

Here is a simple example of wrapper where compute is made in an external program with the help of openturns.coupling\_tools module:

\inputscript{pythonfunction_couplingtools}

Some explanations of the code :
\begin{itemize}
\item[line 8] \verb|coupling_tools.replace| replace \verb|@E| and \verb|@F| occurence found in \verb|input_template.py| file and write the result to \verb|infile| file. \verb|X[0]| value will replace \verb|'@E'| token and \verb|X[1]| will replace \verb|'@F'| token.
\item[line 11] The external program is launched. The input filename is passed by parameters to the program.
\item[line 14] \verb|coupling_tools.get| get the value following  \verb|'Z='| token in \verb|output.py| file.
\end{itemize}
Template file example:

\inputscript{input_template}

External program example:

\inputscript{external_program}


\paragraph{More examples}

\subparagraph{The replace function} can edit file in place. It can format values in anyway. Actually, values can be of type "string", if not, they are converted using str() Python function:

\begin{description}
\item[Usage:] \rule{0pt}{1em}
  \begin{description}
  \item \textit{replace(infile, outfile, tokens, values, encoding=default\_encoding)}
  \end{description}
\item[Arguments:] \rule{0pt}{1em}
  \begin{description}
  \item \textit{infile}: template file that will be parsed
  \item \textit{outfile}: file that will received the template parsed. If equal to None or to \textit{infile}, the result file will be moved to infile
  \item \textit{tokens}: a list of regex that will be replaced
  \item \textit{values}: list of values (can be string, float, ...) that will replace the tokens. The list must have the same size as tokens
  \item \textit{encoding}: the file encoding, as string, i.e. ascii, latin\_1, utf\_8, ...
  \end{description}
\end{description}
\bigskip


\begin{lstlisting}
  replace(outfile='input_template.py', infile=None,
  tokens=['@E', '@F'], values=['%.2f' % 5.2569, 'toto'])
\end{lstlisting}


\begin{lstlisting}
  replace(outfile='input_template.py', infile=None,
  tokens=['@E', '@F'], values=['%.2f' % 5.2569, 'toto'])
\end{lstlisting}

The input\_template.py file will then be modify like this :
\begin{lstlisting}
  E=5.25
  F=toto
\end{lstlisting}

Be careful with overlapping tokens:
\begin{lstlisting}
  # if input_template.py = 'E=@E EE=@EE')
  replace(infile="input_template.py",
  outfile="None",
  tokens=["@E", "@EE"],
  values=[1, 2])
  # => raise exception!! -> @EE token not found!
  # (this is due to the first pass with token "@E" that modify
  # "input_template.py" like this : 'E=1 EE=1E')
\end{lstlisting}

Solution to overlapping tokens: put longest tokens first:
\begin{lstlisting}
  # template.in = 'E=@E EE=@EE')
  replace(infile="template.in",
  outfile="prgm_data.in",
  tokens=["@EE", "@E"],
  values=[2, 1])
  # => prgm_data.in = 'E=1 EE=2')
\end{lstlisting}


\subparagraph{The execute function} can launch an external code.

\begin{description}
\item[Usage:] \rule{0pt}{1em}
  \begin{description}
  \item \textit{execute(cmd, workdir=None, is\_shell=False, shell\_exe=None, hide\_win=True,
    check\_exit\_code=True, get\_stdout=False, get\_stderr=False)}
  \end{description}

\item[Arguments:] \rule{0pt}{1em}
  \begin{description}
  \item \textit{cmd}: a string representing the command. e.g.: 'ls -l /home'
  \item \textit{workdir}: set the current directory of the executed command
  \item \textit{is\_shell}: if set to True, the command is started in a shell (bash). default: False.
  \item \textit{shell\_exe}: path to the shell. e.g. /bin/zsh. default: None: /bin/bash.
  \item \textit{hide\_win}: hide cmd.exe popup on windows
  \item \textit{check\_exit\_code}: if set to True: raise a RuntimeError exception if return code of process != 0
  \item \textit{get\_stdout}: whether standard output of the command is returned
  \item \textit{get\_stderr}: whether standard error of the command is returned
  \end{description}

\item[Value:] \rule{0pt}{1em}
  \begin{description}
  \item the exit code of the command
  \item the stdout data if get\_stdout parameter is set
  \item the stderr data if get\_stderr parameter is set
  \end{description}

\end{description}
\bigskip

\subparagraph{The get\_value function}\label{getvalue} can deal with several type of output file.


\begin{description}
\item[Usage:] \rule{0pt}{1em}
  \begin{description}
  \item \textit{get\_value(filename, token=None, skip\_token=0, skip\_line=0, skip\_col=0, encoding=default\_encoding)}
  \end{description}
\item[Arguments:] \rule{0pt}{1em}
  \begin{description}
  \item \textit{filename}: a file that will be parsed
  \item \textit{token}: a regex that will be searched. The value right after the token is returned. Default: None (no token searched)
  \item \textit{skip\_token}: the number of tokens that will be skipped before getting the
    value. If set to != 0, the corresponding token parameter must not be
    equal to None.
    If skip\_tokens < 0: count tokens backward from the end of the file.
    Default: 0: no token skipped
  \item \textit{skip\_line}: number of lines skipped from the token found.
    If corresponding token equal None, skip from the beginning of the file.
    If corresponding token != None, skip from the token.
    If skip\_line < 0: count lines backward from the token or from the end
    of the file. Be careful: a last empty line is taken into account too.
    Default: 0: no line skipped
  \item \textit{skip\_col}: number of columns skipped from the token found.
    If corresponding token = None, skip words from the beginning of the line.
    If corresponding token != None, skip words from the token.
    If skip\_col < 0: count col backward from the end of the line or from
    the token.
    Default: 0: no column skipped
  \item \textit{encoding}: the file encoding, as string, i.e. ascii, latin\_1, utf\_8, ...
  \end{description}

\item[Value:] \rule{0pt}{1em}
  \begin{description}
  \item a real value
  \end{description}
\end{description}
\bigskip




\begin{itemize}
\item content of the results.out file used for the following examples
  \begin{lstlisting}
    1  2  3  04  5  6
    7  8  9  10
    11 12 13 14

    @Y1= 11.11celcius
    @Y2= -0.89
    @Y1= 22.22
    @Y1= 33.33

    line1: 100 101 102
    line2: 200 201 202
    line3: 300 301 302
  \end{lstlisting}
\item search token, the value right after the token is returned: \\
  \begin{lstlisting}
    Y = get_value('results.out', token='@Y1=') # 11.11
  \end{lstlisting}
\item skip lines and columns (useful for array search):\\
  \begin{lstlisting}
    get_value('results.out', skip_line=1, skip_col=2) # 9
  \end{lstlisting}
\item skip lines and columns backward (be careful: if there is an empty line at the end of the file, it is taken into account. i.e. this last empty line will be reached using skip\_line=-1):\\
  \begin{lstlisting}
    get_value('results.out', skip_line=-2, skip_col=-2) # 201
  \end{lstlisting}
\item search the 3rd appearance of the token:\\
  \begin{lstlisting}
    get_value('results.out', token='@Y1=', skip_token=2) # 33.33
  \end{lstlisting}
\item search the 2nd appearance of the token from the end of the file:\\
  \begin{lstlisting}
    get_value('results.out', token='@Y1=', skip_token=-2) # 22.22
  \end{lstlisting}
\item search a token and then skip lines and columns from this token:\\
  \begin{lstlisting}
    get_value('results.out', token='@Y1=', skip_line=5, skip_col=-2) # 101
  \end{lstlisting}
\item search the 2nd token and then skip lines and columns from this token:\\
  \begin{lstlisting}
    get_value('results.out', token='@Y1=', skip_token=1, skip_line=5, skip_col=1) # 300
  \end{lstlisting}
\end{itemize}

\subparagraph{The get function} works actually the same way the get\_value function do, but on several parameters:\\

\begin{description}
\item[Usage:] \rule{0pt}{1em}
  \begin{description}
  \item \textit{get(filename, tokens=None, skip\_tokens=None, skip\_lines=None, skip\_cols=None, encoding=default\_encoding)}
  \end{description}

\item[Arguments:] \rule{0pt}{1em}
  \begin{description}
  \item \textit{filename}: a file that will be parsed
  \item \textit{tokens}: see \ref{getvalue} function
  \item \textit{skip\_tokens}: see \ref{getvalue} function
  \item \textit{skip\_lines}: see \ref{getvalue} function
  \item \textit{skip\_cols}: see \ref{getvalue} function
  \item \textit{encoding}: the file encoding, as string, i.e. ascii, latin\_1, utf\_8, ...
  \end{description}

\item[Value:] \rule{0pt}{1em}
  \begin{description}
  \item a list of real values.
  \end{description}
\end{description}
\bigskip

\begin{lstlisting}
  get('results.out', tokens=['@Y1=', '@Y2'], skip_lines=[5, 0], skip_cols=[-2, 0]) # [101, -0.89]
\end{lstlisting}


\subparagraph{The get\_regex function} parses the outfile. It is provided for backward compatibility:\\

\begin{description}
\item[Usage:] \rule{0pt}{1em}
  \begin{description}
  \item \textit{get\_regex(filename, patterns)}
  \end{description}

\item[Arguments:] \rule{0pt}{1em}
  \begin{description}
  \item \textit{filename}: the file to parse
  \item \textit{patterns}: a list of patterns that will permit to get the values.
    \textbackslash \textbackslash R and \textbackslash \textbackslash I can be used to match float and integer.
    \textbackslash \textbackslash s can be used to match any whitespace character (= [ \textbackslash \textbackslash t\textbackslash \textbackslash n\textbackslash \textbackslash r\textbackslash \textbackslash f\textbackslash \textbackslash v])
    \textbackslash \textbackslash S can be used to match any non-whitespace character.
    The value to be searched must be surrounded by '(' and ')' (see example).
  \end{description}

\item[Value:] \rule{0pt}{1em}
  \begin{description}
  \item a list of values corresponding to each pattern.
    If nothing has been found, the corresponding value is set to None.
  \end{description}
\end{description}
\bigskip



\begin{lstlisting}
  Y = get_regex('results.out', patterns=['@Y2=(\R)']) # -0.89
\end{lstlisting}


\paragraph{Reference}

Most up to date coupling tools module documentation is available through docstring in Python console:

\begin{lstlisting}
  import openturns as ot
  help(ot.coupling_tools.get_value)
\end{lstlisting}

Or in IPython console:
\begin{lstlisting}
  ot.coupling_tools.replace?
\end{lstlisting}


\subsection{Performance considerations\label{speedo}}

Two differents cases can be encounter when wrapping code: the wrapping code is an analytical mathematical formula or it is an external code (an external process).

\subsubsection{Analytical formula}

A benchmark involving the differents wrapping methods available from \OT\ has been done using a dummy Analytical formula.


\paragraph{Benchmark sources} Optimizations of any parts of this benchmark are welcome.

\begin{itemize}
\item Benchmark of PythonFunction using \_exec function:
  \begin{lstlisting}
    big_sample = ot.Normal(2).getSample(1000*1000)
    import openturns as ot

    def _exec( X ):
    return [math.cos(pow(X[0]+1, 2)) - math.sin(X[1])]

    model = ot.PythonFunction(2, 1, _exec)
    # start timer
    out_sample = model( big_sample )
    # stop timer
  \end{lstlisting}

\item Benchmark of PythonFunction using \_exec\_sample function:
  \begin{lstlisting}
    def _exec_sample( Xs ):
    import numpy as np
    XsT = np.array(Xs).T
    return np.atleast_2d(np.cos(np.power(xT[0]+1, 2)) - np.sin(xT[1])).T

    model = ot.PythonFunction(2, 1, func_sample=_exec_sample)
  \end{lstlisting}

\item Benchmark of Analytical (muParser) function:
  \begin{lstlisting}
    model = ot.NumericalMathFunction( ('x0','x1'), ('y',),
    ('cos((x0+1) ^ 2) - sin(x1)',) )
  \end{lstlisting}

\end{itemize}


The benchmark is done on a bi XEON E5520 (Nehalem 16*2.27GHz, HT activated) with 12Go RAM.


\paragraph{Benchmark results}:

The sample containing 1 million of points is allocated in 0.282s.

\begin{tabular}{lll}
  wrapper type & time & comparison with fastest wrapper \\
  PythonFunction \_exec & 7.1s & x157 \\
  PythonFunction \_exec\_sample & 1.3s & x30 \\
  Analytical (muParser) & 0.43s & x10 \\
\end{tabular}

The previous results are linear to the size of the sample.

\begin{itemize}
\item muParser is the 2nd fastest (10 times slower than the first).

  The muParser library embedded in \OT\ is not multithreaded. Embedding a parallel version of muParser could give better results.
\item Using an optimized \_exec\_sample python function through numpy gives better results (6x faster) than a simple \_exec python function, but it is still much slower than the compiled library (30 times slower).

  Note that neither Python nor NumPy are multithreaded.
\end{itemize}


\paragraph{Conclusion} PythonFunction is the easiest and more adaptable wrapper but it's the slowest too. So, if you need to compute samples containing less than a million of points, PythonFunction is the good choice as the speed difference between wrappers will not be noticeable: every wrappers will compute the sample in less than a second. Otherwise choose muParser.




\subsubsection{External process}

\paragraph{Normal program}

For usual program (compute time of 1s and above), inner wrapper complexity/overhead are not an issue cause the external program compute time will be the main part of the whole compute time. Sample can be computed faster by launching this external program in parallel.

\begin{itemize}
\item PythonFunction can not launch the \_exec function in parallel.
\item the DistributedPythonFunction from otdistfunc module can launch external program on each core of the local Machine or on each core of several remote machine.
\end{itemize}

The DistributedPythonFunction is the best choice as it combine the ease of use of Python with the ability to deploy compute on a cluster of computers.



\paragraph{Tiny program}

If the external process compute time is really fast (< 0.1s), \OT\ wrapper point's launch time (overhead) becomes important.

If performance are an issue, one should first consider that the external process is perhaps fast because it does something simple: can it be easily reimplemented in Python? If the code is not too complex, execute Python code inside a PythonFunction is usually much faster than the time to start the external process (~1000x). Here is a naive example of external process (scilab) vs PythonFunction.

\begin{itemize}
\item The following scilab script takes 0.07s per point:

  \verb|$ scilab -nb -nwni -f code.sce|

  \begin{lstlisting}
    // code.sce
    exec("input.data", -1)
    y = x1 + x2;
    f = mopen("result.data", "wt");
    mfprintf(f, "y = %.20e", y);
    file("close", f);
    quit
  \end{lstlisting}


\item Conversion to Python of the scilab script. It takes now 0.00001s per point:

  \begin{lstlisting}
    def _exec( X ):
    return X[0] + X[1]
    model = ot.PythonFunction(2, 1, _exec)
  \end{lstlisting}
\end{itemize}

If you still need to launch tiny external process, slow overhead and parallel ability are the important factors of the wrapper.
Comparison of the differents wrapper compute time with a sample of size 1000 and an external code that last 0.07s per point on a 8 cores computer:

\begin{itemize}
\item PythonFunction overhead is really slow (0.000004s) but can not launch the \_exec function in parallel.

  $(0.000004+0.07)*1000 => 70s$
\item DistributedPythonFunction overhead is near 0.05s and can launch external program in parallel.

  $(0.05+0.07)*1000 (/8core) => 15s$

\item PythonFunction that reimplement the external program.

  $(0.00001)*1000 => 0.01s$

\end{itemize}

