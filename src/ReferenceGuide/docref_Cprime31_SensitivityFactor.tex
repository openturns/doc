% Copyright 2005-2016 Airbus-EDF-IMACS-Phimeca
% Permission is granted to copy, distribute and/or modify this document
% under the terms of the GNU Free Documentation License, Version 1.2
% or any later version published by the Free Software Foundation;
% with no Invariant Sections, no Front-Cover Texts, and no Back-Cover
% Texts.  A copy of the license is included in the section entitled "GNU
% Free Documentation License".
\renewcommand{\nomfichier}{docref_Cprime31_SensitivityFactor}
\renewcommand{\titrefiche}{Sensitivity Factors from FORM method}

\Header

\MathematicalDescription{
  \underline{\textbf{Goal}} \vspace{2mm}

  Sensitivity Factors are evaluated under the following context : $\vect{X}$ denotes a random input vector, representing the sources of uncertainties, $\pdf$ its joint density probability, $\vect{d}$ a determinist vector, representing the fixed variables $g(\vect{X}\,,\,\vect{d})$ the limit state function of the model, $\cD_f = \{\vect{X} \in \Rset^n \, / \, g(\vect{X}\,,\,\vect{d}) \le 0\}$ the  event considered here and ${g(\vect{X}\,,\,\vect{d}) = 0}$ its boundary (also called limit state surface).\\
  The probability content of the event $\cD_f$ is $P_f$:
  \begin{eqnarray}
    P_f =       \int_{g(\vect{X}\,,\,\vect{d}) \le 0}  \pdf\, d\vect{x}.
  \end{eqnarray}\label{PfX11}
  In this context, the probability $P_f$ can often be efficiently estimated by FORM or SORM approximations (refer to \otref{docref_C311_Form}{FORM} and \otref{docref_C311_Sorm}{SORM}).\\

  The FORM importance factors offer a way to analyse the sensitivity of the probability the realization of the event with respect to the parameters of the probability distribution of \vect{X}.

  \underline{\textbf{Principle}} \vspace{2mm}

  A sensitivity factor is defined as the derivative of the Hasofer-Lind reliability index with respect to the paramater $\theta$. The paramater $\theta$ is a parameter in a distribution of the random vector $\vect{X}$.\\

  If $\vect{\theta}$ represents the vector of all the parameters of the distribution of $\vect{X}$ which appear in the definition of the isoprobabilistic transformation $T$ (refer to \otref{docref_C311_TransIso}{IsoProbabiliticFunction}), and $U_{\vect{\theta}}^{*}$ the design point associated to the event considered in the $U$-space, and if the mapping of the limit state function by the $T$ is noted $G(\vect{U}\,,\,\vect{\theta}) =  g[T^{-1}(\vect{U}\,,\,\vect{\theta}), \vect{d}]$, then the sensitivity factors vector is defined as :
  \begin{align*}
    \nabla_{\vect{\theta}} \beta_{HL} =  \displaystyle +\frac{1}{||\nabla_{\vect{\theta}} G(U_{\vect{\theta}}^{*}, \vect{d})||} \nabla_{\vect{u}} G(U_{\vect{\theta}}^{*}, \vect{d}).
  \end{align*}

  The sensitivity factors indicate the importance on the Hasofer-Lind reliability index (refer to \otref{docref_C311_ReliabilityIndex}{Reliability Index}) of the value of the parameters used to define the distribution of the random vector $\vect{X}$.
}
{
  Here, the event considered is explicited directly from the limit state function $g(\vect{X}\,,\,\vect{d})$ : this is the classical structural reliability formulation.\\
  However, if the event is a threshold exceedance, it is useful to explicite the variable of interest $Z=\tilde{g}(\vect{X}\,,\,\vect{d})$, evaluated from the model $\tilde{g}(.)$. In that case, the event considered, associated to the threshold $z_s$ has the formulation: $\cD_f = \{ \vect{X} \in \Rset^n \, / \, Z=\tilde{g}(\vect{X}\,,\,\vect{d}) > z_s \}$
  and the limit state function is : $g(\vect{X}\,,\,\vect{d}) = z_s - Z = z_s - \tilde{g}(\vect{X}\,,\,\vect{d})$. $P_f$ is the threshold exceedance probability, defined as : $P_f     =       P(Z \geq z_s) = \int_{g(\vect{X}\,,\,\vect{d}) \le 0}  \pdf\, d\vect{x}$.
  Thus, the FORM sensitivity factors offer a way to rank the importance of the parameters of the input components with respect to the threshold exceedance by the quantity of interest $Z$. They can be seen as a specific sensitity analysis technique dedicated to the quantity Z around a particular threshold rather than to its variance.
}

\Methodology{
  Within the global methodology, sensitivity factors are evaluated in the step $C^{'}$: "Ranking sources of uncertainty" in the case of the evaluation of the probability of an event
  by an approximation method.\\
  It requires to have fulfilled before the following steps:
  \begin{itemize}
  \item step A: input vector $\vect{X}$,  final variable of interest (result of a model), probabilistic criteria (the event considered) ${g(\vect{X}\,,\,\vect{d}) \le 0}$,
  \item step B: one of the proposed techniques to describe the probabilistic modelisation of the input vector $\vect{X}$,
  \item step C: one method to evaluate the probability content of the event : the FORM or SORM approximation
  \end{itemize}
}
            {
              The standard version of OpenTURNS takes into account only the sensitivity with respect to the parameters of the distributino of $\vect{X}$ which appear in the definition of the isoprobabilistic transformation $T$. It does not calculate the sensitivity with respect to the other parameters, in particular those of the limit state function $\vect{d}$. \\

              The FORM importance factors (refer to \otref{docref_Cprime31_ImportanceFactor}{Importance Factors}) offer a way to rank the importance of the input components with respect the realization of the event. They are often interpreted also as indicators of the impact of modeling the input components as random variables rather than fixed values.\\

              Let's note some usefull references:
              \begin{itemize}
              \item O. Ditlevsen, H.O. Madsen, 2004, "Structural reliability methods," Department of mechanical engineering technical university of Denmark - Maritime engineering, internet publication.
              \end{itemize}

            }

            \Example{

              Let's apply this method to the following analytical example which considers a cantilever beam, of Young's modulus E, length L, section modulus I. We apply a concentrated bending force at the other end of the beam. The vertical displacement $y$ of the extrême end is equal to :
              \begin{align*}
                y(E, F, L, I) = \displaystyle \frac{FL^3}{3EI}
              \end{align*}
              The objective is to propagate until $y$ the uncertainties of the variables $(E, F, L, I)$.\\
              The input random vector is $\vect{X} = (E, F, L, I)$, which probabilistic modelisation is (unity is not provided):
              \newpage

              \begin{align*}
                \left\{
                \begin{array}{lcl}
                  E & = & Normal(50, 1) \\
                  F & = & Normal(1, 1) \\
                  L & = & Normal(10, 1) \\
                  I & = & Normal(5, 1)
                \end{array}
                \right.
              \end{align*}
              The event considered is the threshold exceedance : $\cD_f = \{(E, F, L, I) \in \Rset^4 \, / \, y(E, F, L, I) \geq 3\}$.\\

              If we note $\mu$ the mean and $\sigma$ the standard deviation a the random variable, we obtain the following results, gathered in the following tables.


              \begin{center}

                \begin{tabular}{cc}

                  \begin{tabular}{|c|c|c|}
                    \hline
                    $\beta_{HL}$ & $\mu$ & $\sigma$ \\
                    \hline
                    E  & 0.0307508 & -0.000954364 \\
                    \hline
                    F  & -0.834221 &  -0.000954364 \\
                    \hline
                    L  & -0.441319 &  -0.000954364 \\
                    \hline
                    I  & 0.329191 & -0.000954364 \\
                    \hline
                  \end{tabular}
                  &
                  \begin{tabular}{|c|c|c|}
                    \hline
                    $P_{f,FORM}$ & $\mu$ & $\sigma$ \\
                    \hline
                    E  & -0.00737194  & 0.000228791 \\
                    \hline
                    F  &  0.199989 & 0.000228791  \\
                    \hline
                    L  &  0.105798 & 0.000228791   \\
                    \hline
                    I  &  -0.0789175 & 0.000228791 \\
                    \hline
                  \end{tabular}

                \end{tabular}
              \end{center}
            }
