% Copyright (C) 2005-2015 Airbus - EDF - IMACS - Phimeca
% Permission is granted to copy, distribute and/or modify this document
% under the terms of the GNU Free Documentation License, Version 1.2
% or any later version published by the Free Software Foundation;
% with no Invariant Sections, no Front-Cover Texts, and no Back-Cover
% Texts.  A copy of the license is included in the section entitled "GNU
% Free Documentation License".
\renewcommand{\etapemethodo}{Resp. Surf.}
\renewcommand{\nomfichier}{docref_SurfRep_Kriging}
\renewcommand{\titrefiche}{Kriging}

\Header

\MathematicalDescription{

  \underline{\textbf{Goal}} \vspace{2mm}

  Kriging (also known as Gaussian process regression) is a Bayesian technique that aim at approximating functions (most often in order to surrogate it because it is expensive to evaluate). In the following it is assumed we aim at surrogating a scalar-valued model $\cM: \vect{x} \mapsto y$. Note the OpenTURNS implementation of Kriging can deal with vector-valued functions ($\cM: \vect{x} \mapsto \vect{y}$), but it simply loops over each output. It is also assumed the model was runned over a design of experiments in order to produce a set of observations gathered in the following dataset: $\left(\left(\vect{x}^{(i)}, y^{(i)}\right), i = 1, \ldots, m\right)$. Ultimately Kriging aims at producing a predictor (also known as a response surface or metamodel) denoted as $\tilde{\cM}$.

  \vspace{2mm}

  \underline{\textbf{Mathematical framework}} \vspace{2mm}

  We put the following Gaussian process prior on the model $\cM$:
  \begin{equation}
    Y(\vect{x}) = \Tr{\vect{f}(\vect{x})} \vect{\beta} + Z(\vect{x})
  \end{equation}
  where:
  \begin{itemize}
    \item $\Tr{\vect{f}(\vect{x})} \vect{\beta}$ is a generalized linear model based upon a functional basis $\vect{f} = \left(f_j, j = 1, \ldots, p\right)$ and a vector of coefficients $\vect{\beta} = \left(\beta_j, j = 1, \ldots, p\right)$,
    \item $Z$ is a zero-mean stationary Gaussian process whose covariance function reads:
    \begin{equation}
        \mathbb{E}[Z(\vect{x})\,Z(\vect{x'})] = \sigma^2 R(\vect{x} - \vect{x'}, \vect{\theta})
    \end{equation}
    where $\sigma^2 > 0$ is the variance and $R$ is the correlation function that solely depends on the Manhattan distance between input points $\vect{x} - \vect{x'}$ and a vector of parameters $\vect{\theta} \in \Rset^{n_\theta}$.
  \end{itemize}

  Under the Gaussian process prior assumption, the observations $\vect{Y} = \left(Y_i, i = 1, \ldots, m\right)$ and a prediction $Y(\vect{x})$ at some unobserved input $\vect{x}$ are jointly normally distributed:
  \begin{equation}
    \begin{pmatrix}
      \vect{Y} \\
      Y(\vect{x})
    \end{pmatrix}
    \thicksim
    \cN_{m + 1}\left(
      \begin{pmatrix}
        \mat{F} \vect{\beta} \\
        \Tr{\vect{f}(\vect{x})} \vect{\beta}
      \end{pmatrix}
      ,\,\sigma^2
      \begin{pmatrix}
        \mat{R} & \vect{r}(\vect{x}) \\
        \vect{r}(\vect{x})^t & 1
      \end{pmatrix}
    \right)
  \end{equation}
  where:
  \begin{equation}
    \mat{F} = \left[f_j\left(\vect{x}^{(i)}\right), i = 1, \ldots, m, j = 1, \ldots, p\right]
  \end{equation}
  is the regression matrix,
  \begin{equation}
    \mat{R} = \left[R\left(\vect{x}^{(i)} - \vect{x}^{(j)}, \vect{\theta}\right), i,\,j = 1, \ldots, m\right]
  \end{equation}
  is the observations' correlation matrix, and:
  \begin{equation}
    \vect{r}(\vect{x}) = \Tr{\left(R\left(\vect{x} - \vect{x}^{(i)}, \vect{\theta}\right), i = 1, \ldots, m\right)}
  \end{equation}
  is the vector of cross-correlations between the prediction and the observations.

  As such, the Kriging predictor is defined as the following conditional distribution:
  \begin{equation}
    \tilde{Y}(\vect{x}) = \left[Y(\vect{x}) \mid \vect{Y} = \vect{y}, \vect{\theta}=\vect{\theta}^*, \sigma^2=(\sigma^2)^*\right]
  \end{equation}
  where $\vect{\theta}^*$ and $(\sigma^2)^*$ are the maximum likelihood estimates of the correlation parameters $\vect{\theta}$ and variance $\sigma^2$ (see references).

  It can be shown (see references) that the predictor $\tilde{Y}(\vect{x})$ is also Gaussian:
  \begin{equation}
    \tilde{Y}(\vect{x}) = \cN_1\left(\mu_{\tilde{Y}}(\vect{x}), \sigma^2_{\tilde{Y}}(\vect{x})\right)
  \end{equation}

  \begin{itemize}
    \item with mean:
    \begin{equation}
      \mu_{\tilde{Y}}(\vect{x}) = \Tr{\vect{f}(\vect{x})} \tilde{\vect{\beta}} + \Tr{\vect{r}(\vect{x})} \vect{\gamma}
    \end{equation}
    where $\underline{\tilde{\beta}}$ is the generalized least squares solution of the underlying linear regression problem:
    \begin{equation}
      \tilde{\vect{\beta}} = \left(\Tr{\mat{F}} \mat{R}^{-1} \mat{F}\right)^{-1} \Tr{\mat{F}} \mat{R}^{-1} \vect{y}
    \end{equation}
    and
    \begin{equation}
      \vect{\gamma} = \mat{R}^{-1} \left(\vect{y} - \mat{F} \tilde{\vect{\beta}}\right)
    \end{equation}
    \item and variance:
    \begin{equation}
      \sigma^2_{\tilde{Y}}(\vect{x}) =
        \sigma^2 \left[1 -
            \Tr{\vect{r}(\vect{x})} \mat{R}^{-1} \vect{r}(\vect{x})
            + \Tr{\vect{u}(\vect{x})} \left(\Tr{\mat{F}} \mat{R}^{-1} \mat{F}\right)^{-1} \vect{u}(\vect{x})
            \right]
    \end{equation}
    where:
    \begin{equation}
      \vect{u}(\vect{x}) = \Tr{\mat{F}} \mat{R}^{-1} \vect{r}(\vect{x}) - \vect{f}(\vect{x})
    \end{equation}
  \end{itemize}

  For now only the Kriging mean is returned as the metamodel $\tilde{\cM} \equiv \mu_{\tilde{Y}}$.

}
{
  Kriging may also be referred to as \emph{Gaussian process regression}.
}

\Methodology{}
            {
              More resources in Kriging may be found in:
              \begin{itemize}
                \item V. Dubourg, 2011, ``Adaptative surrogate models for reliability and reliability-based design optimization'', University Blaise Pascal - Clermont II. http://bruno.sudret.free.fr/dubourg.html
                \item S. Lophaven, H. Nielsen and J. Sondergaard, 2002, ``DACE, A Matlab kriging toolbox'', Technichal University of Denmark. http://www2.imm.dtu.dk/~hbni/dace/              \item T. Santner, B. Williams and W. Notz, 2003. ``The design and analysis of computer experiments'', Springer, New York.
                \item C. Rasmussen and C. Williams, 2006, T. Dietterich (Ed.), ``Gaussian processes for machine learning'', MIT Press.
              \end{itemize}
            }
