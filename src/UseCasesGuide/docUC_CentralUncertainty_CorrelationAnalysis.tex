% Copyright 2005-2016 Airbus-EDF-IMACS-Phimeca
% Permission is granted to copy, distribute and/or modify this document
% under the terms of the GNU Free Documentation License, Version 1.2
% or any later version published by the Free Software Foundation;
% with no Invariant Sections, no Front-Cover Texts, and no Back-Cover
% Texts.  A copy of the license is included in the section entitled "GNU
% Free Documentation License".
\renewcommand{\filename}{docUC_CentralUncertainty_CorrelationAnalysis.tex}
\renewcommand{\filetitle}{UC : Correlation analysis on samples : Pearson and Spearman coefficients, PCC, PRCC, SRC, SRRC coefficients}

% \HeaderNNIILevel
% \HeaderIILevel
\HeaderIIILevel

\label{correlationAnalysis}


\index{Correlation!Pearson correlation coefficient}
\index{Correlation!Partial Pearson correlation coefficient (PCC)}
\index{Correlation!Spearman correlation coefficient}
\index{Correlation!Partial rank correlation coefficient (PRCC)}
\index{Correlation!Standard regression coefficient (SRC)}
\index{Correlation!Standard rank regression coefficient (SRRC)}


This Use Case  describes the correlation analysis we can perform between the input random  vector, described by a numerical sample, and the output variable of interest described by a numerical sample too.\\

Details on design of experiments correlation coefficients may be found in the Reference Guide (\extref{ReferenceGuide}{see files Reference Guide - Step C' -- Uncertainty Ranking using Pearson's correlation}{stepCprime) and files around it}.\\



\requirements{
  \begin{description}
  \item[$\bullet$] a numerical sample : {\itshape inputSample}, may be of dimension $\geq 1$
  \item[type:] NumericalSample
  \item[$\bullet$] two scalar  numerical samples : {\itshape inputSample2, outputSample}
  \item[type:] NumericalSample
  \end{description}
}
             {
               \begin{description}
               \item[$\bullet$] the different correlation coefficients : {\itshape PCCcoefficient, PRCCcoefficient, SRCcoefficient, SRRCcoefficient, pearsonCorrelation, spearmanCorrelation}
               \item[type:] NumericalPoint
               \end{description}
             }

             \textspace\\
             Python script for this UseCase :

             \inputscript{script_docUC_CentralUncertainty_CorrelationAnalysis}
