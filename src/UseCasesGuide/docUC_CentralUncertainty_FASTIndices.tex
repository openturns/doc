% Copyright 2005-2016 Airbus-EDF-IMACS-Phimeca
% Permission is granted to copy, distribute and/or modify this document
% under the terms of the GNU Free Documentation License, Version 1.2
% or any later version published by the Free Software Foundation;
% with no Invariant Sections, no Front-Cover Texts, and no Back-Cover
% Texts.  A copy of the license is included in the section entitled "GNU
% Free Documentation License".
\renewcommand{\filename}{docUC_CentralUncertainty_FASTIndices.tex}
\renewcommand{\filetitle}{UC : Sensitivity analysis : FAST indices}

% \HeaderNNIILevel
% \HeaderIILevel
\HeaderIIILevel


\index{Sensitivity!FAST indices}

The objective of the Use Case is to quantify the correlation between the input variables and the output variable of a model described by a numerical function : it is called sensitivity analysis. The FAST method, based upon the Fourier decomposition of the model response, is a relevant alternative to the classical simulation approach (See Use Case \ref{SobolIndices}) for computing Sobol sensitivity indices. The FAST indices, like the Sobol indices, allow to evaluate the importance of a single variable or a specific set of variables.\\
In theory, FAST indices range is $\left[0; 1\right]$ ; the closer to 1 the index is, the greater the model response sensitivity to the variable is. The FAST method compute the first and total order indices.\\
The first order indices evaluate the importance of one variable at a time ($d$ indices stored in a NumericalPoint, with $d$ the input dimension of the model).\\
The $d$ total indices give the relative importance of every variables except the variable $X_i$, for every variable.\\

Details on the FAST method may be found in the Reference Guide (\extref{ReferenceGuide}{see files Reference Guide - Step C' -- Sensitivity analysis by Fourier decomposition}{stepCprime}).\\

\requirements{
  \begin{description}

  \item[$\bullet$] an independent joint distribution : {\itshape distribution}
  \item[type:] Distribution
  \item[$\bullet$] a function : {\itshape model}, which input dimension must fit the dimension of the distribution
  \item[type:] NumericalMathFunction
  \item[$\bullet$] sample size : {\itshape N}, from which the Fourier series are calculated. It represents the length of the discretization of the s-space.
  \item[type:] int
  \item[$\bullet$] number of resamplings : {\itshape Nr}, which enables to realize the procedure Nr times and then to calculate the arithmetic means of the results over the Nr estimates.
  \item[type:] int
  \item[$\bullet$] the interference factor : {\itshape M}, usually equal to 4 or higher. It corresponds to the truncation level of the Fourier series, i.e. the number of harmonics that are retained in the decomposition.
  \end{description}
}
{
  \begin{description}
  \item[$\bullet$] the different FAST indices
  \item[type:] NumericalPoint, for first and total indices
  \end{description}
}

\textspace\\
Python script for this UseCase :

\inputscript{script_docUC_CentralUncertainty_FASTIndices}
