% Copyright 2005-2016 Airbus-EDF-IMACS-Phimeca
% Permission is granted to copy, distribute and/or modify this document
% under the terms of the GNU Free Documentation License, Version 1.2
% or any later version published by the Free Software Foundation;
% with no Invariant Sections, no Front-Cover Texts, and no Back-Cover
% Texts.  A copy of the license is included in the section entitled "GNU
% Free Documentation License".
\renewcommand{\filename}{docUC_CentralUncertainty_QuantileEstimation.tex}
\renewcommand{\filetitle}{UC : Quantile estimations : Wilks and empirical estimators}

% \HeaderNNIILevel
% \HeaderIILevel
\HeaderIIILevel



\index{Quantile!Empirical estimation}
\index{Quantile!Wilks estimation}
\index{Wilks}

The objective of this Use Case is to evaluate a particular quantile, with the empirical estimator or the Wilks one, from a numerical sample of the random variable. Each estimation is associated to a confidence interval, which level is specified. \\


Details on probability estimators  may be found in the Reference Guide (\extref{ReferenceGuide}{see files Reference Guide - Step C -- Estimating a quantile by Sampling / Wilks Method}{stepC}).\\



Let's suppose we want to estimate the quantile $q_{\alpha}$ of order $\alpha$ of the variable $Y$ : $\Prob{Y \leq q_{\alpha}} = \alpha$, from the numerical sample $(Y_1, ..., Y_n)$ of size $n$, with a confidence level equal to $\beta$. We note $(Y^{(1)}, ..., Y^{(n)})$ the numerical sample where the values are sorted in ascending order.\\
The empirical estimator,  noted $q_{\alpha}^{emp}$, and its confidence intervall, is defined by the expressions :
\begin{align*}
  \left\{
  \begin{array}{lcl}
    q_{\alpha}^{emp} & = & Y^{(E[n\alpha])} \\
    P(q_{\alpha} \in [Y^{(i_n)}, Y^{(j_n)}]) & = & \beta \\
    i_n & = & E[n\alpha - a_{\alpha}\sqrt{n\alpha(1-\alpha)}] \\
    i_n & = & E[n\alpha + a_{\alpha}\sqrt{n\alpha(1-\alpha)}]
  \end{array}
  \right.
\end{align*}
The Wilks estimator,  noted $q_{\alpha, \beta}^{Wilks}$, and its confidence intervall, is defined by the expressions :
\begin{align*}
  \left\{
  \begin{array}{lcl}
    q_{\alpha, \beta}^{Wilks} & = & Y^{(n-i)} \\
    P(q_{\alpha}  \leq q_{\alpha, \beta}^{Wilks}) & \geq & \beta \\
    i\geq 0 \, \, /  \, \, n \geq N_{Wilks}(\alpha, \beta,i)
  \end{array}
  \right.
\end{align*}

Once the order $i$ has been chosen, the Wilks number $N_{Wilks}(\alpha, \beta,i)$ is evaluated by OpenTURNS, thanks to the static method $ComputeSampleSize(\alpha, \beta, i)$ of the \textit{Wilks} object.\\

In the example, we want to evaluate a quantile $\alpha = 95\%$, with a confidence level of $\beta = 90\%$ thanks to the $4$th maximum of the ordered sample (associated to the order $i = 3$).\\
Care : $i=0$ signifies that the Wilks estimator is the maximum of the numerical sample : it corresponds to the first maximum of the numerical sample.\\

Before any simulation, we initialise the state of the random generator.\\

The method {\itshape computeQuantile} evaluates the empirical quantile from a numerical sample in the case of dimension $n \geq 1$. However, the evaluation of the confidence interval is given only in the case of dimension 1.\\
Furter more, the Wilks estimator and its confidence interval is evaluated in the case of dimension 1 only.\\

\requirements{
  \begin{description}
  \item[$\bullet$] the output variable of interest  of dimension 1 : {\itshape output}
  \item[type:] RandomVector
  \end{description}
}
             {
               \begin{description}
               \item[$\bullet$] the quantile estimators
               \item[type:] NumericalScalar
               \item[$\bullet$] Confidence Interval length
               \item[type:] NumericalScalar
               \end{description}
             }

             \textspace\\
             Python script for this UseCase :

             \inputscript{script_docUC_CentralUncertainty_QuantileEstimation}
