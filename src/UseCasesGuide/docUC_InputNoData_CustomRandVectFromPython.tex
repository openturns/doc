% Copyright 2005-2016 Airbus-EDF-IMACS-Phimeca
% Permission is granted to copy, distribute and/or modify this document
% under the terms of the GNU Free Documentation License, Version 1.2
% or any later version published by the Free Software Foundation;
% with no Invariant Sections, no Front-Cover Texts, and no Back-Cover
% Texts.  A copy of the license is included in the section entitled "GNU
% Free Documentation License".
\renewcommand{\filename}{docUC_InputNoData_CustomRandVectFromPython.tex}
\renewcommand{\filetitle}{UC : Creation of a custom random vector from the python script}

% \HeaderNNIILevel
% \HeaderIILevel
\HeaderIIILevel


\label{manipulation_random vector}

The objective of this Use Case is to describe the main functionalities enabling to define a random vector from the python script.\\

The principle is to inherit from the \textit{PythonRandomVector} class and overload the methods of the RandomVector object.\\
Then an instance of the new class can be passed on into a RandomVector object.\\
At least getRealization should be overriden.

\requirements{
  \begin{description}
  \item[$\bullet$] at least a realization function expressed in python.
  \end{description}
}
             {
               \begin{description}
               \item[$\bullet$] an object proposing the same services RandomVector does.
               \item[type:] RandomVector
               \end{description}
             }

             \textspace\\
             Python script for this UseCase :

             \inputscript{script_docUC_InputNoData_CustomRandVectFromPython}
