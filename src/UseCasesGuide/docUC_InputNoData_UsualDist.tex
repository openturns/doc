% Copyright (C) 2005-2015 Airbus - EDF - IMACS - Phimeca
% Permission is granted to copy, distribute and/or modify this document
% under the terms of the GNU Free Documentation License, Version 1.2
% or any later version published by the Free Software Foundation;
% with no Invariant Sections, no Front-Cover Texts, and no Back-Cover
% Texts.  A copy of the license is included in the section entitled "GNU
% Free Documentation License".
\renewcommand{\filename}{docUC_InputNoData_UsualDist}
\renewcommand{\filetitle}{UC : List of usual distributions}

% \HeaderNNIILevel
% \HeaderIILevel
\HeaderIIILevel

\index{Usual Distribution!Arcsine}
\index{Usual Distribution!Beta}
\index{Usual Distribution!Bernoulli}
\index{Usual Distribution!Binomial}
\index{Usual Distribution!Burr}
\index{Usual Distribution!Chi}
\index{Usual Distribution!ChiSquare}
\index{Usual Distribution!Dirac}
\index{Usual Distribution!Dirichlet}
\index{Usual Distribution!Epanechnikov}
\index{Usual Distribution!Exponential}
\index{Usual Distribution!Fisher-Snedecor}
\index{Usual Distribution!GeneralizedPareto}
\index{Usual Distribution!Gamma}
\index{Usual Distribution!Gumbel}
\index{Usual Distribution!Histogram}
\index{Usual Distribution!InverseChiSquare}
\index{Usual Distribution!InverseGamma}
\index{Usual Distribution!InverseNormal}
\index{Usual Distribution!InverseWishart}
\index{Usual Distribution!Laplace}
\index{Usual Distribution!Logistic}
\index{Usual Distribution!LogNormal}
\index{Usual Distribution!LogUniform}
\index{Usual Distribution!Negative Binomial}
\index{Usual Distribution!Non Central Chi Square}
\index{Usual Distribution!Non Central Student}
\index{Usual Distribution!Normal}
\index{Usual Distribution!NormalGamma}
\index{Usual Distribution!Rayleigh}
\index{Usual Distribution!Rice}
\index{Usual Distribution!Student}
\index{Usual Distribution!Triangular}
\index{Usual Distribution!Trapezoidal}
\index{Usual Distribution!TruncatedNormal}
\index{Usual Distribution!Uniform}
\index{Usual Distribution!Weibull}
\index{Usual Distribution!Wishart}
\index{Usual Distribution!Geometric}
\index{Usual Distribution!Multinomial}
\index{Usual Distribution!Poisson}
\index{Usual Distribution!User defined}
\index{Usual distribution!Zipf-Mandelbrot}

The objective of this Use Case is to list all the usual distributions proposed by OpenTURNS and to precise how each distribution is created, with its different arguments.\\

The different distributions proposed by OpenTURNS are listed here after.\\[1em]

Details on each object may be found in the User Manual  (\extref{UserManual_TUI}{see User Manual - Probabilistic modeling / Usual Distributions}{probModeling}).\\[1em]

\begin{itemize}
\item Continuous distributions :
\end{itemize}

{\footnotesize

  \noindent \begin{tabular}{|p{1.8cm}|p{6.0cm}|p{2.7cm}|p{1.7cm}|p{4.6cm}|}
    % 
    \hline
    % 
    Name & probability density function & conditions & param. 1 & param. $2^{\strut}_{\strut}$\\
    % 
    \hline
    % 
    Arcsine & $\frac{1}{\pi \frac{b-a}{2} \sqrt{1-\left(\frac{x-\frac{a+b}{2}}{\frac{b-a}{2}}\right)^{2}}}$ & $a <b$& $(a, b)$ & $(\mu, \sigma)$ with
    $
    \left\{
      \begin{array}{l}
        \mu = \frac{a+b}{2} \\
        \sigma = \frac{b-a}{2\sqrt{2}}
      \end{array}
    \right.
    $ \\
    % 
    \hline
    % 
    Beta & $\displaystyle  \frac{(x-a)^{(r-1)^{\strut}}(b-x)^{(t-r-1)}}{(b-a)^{(t-1)}B(r,t-r)}\boldsymbol{1}_{[a,b]}(x)$  & $r>0$, $t>r$, $a < b$ & $(r, t, a, b)$ & $(\mu, \sigma, a,b)$ with

    $
    \left\{
      \begin{array}{l}
        \mu = a+(b-a)\frac{r}{t} \\
        \sigma = (b-a)\frac{r}{t}\frac{\sqrt{t-r}}{\sqrt{r(t+1)}}
      \end{array}
    \right._{\strut}
    $\\
    % 
    \hline
    % 
    Burr & $\displaystyle ck\frac{x^{(c-1)}}{(1+x^c)^{(k+1)}} \boldsymbol{1}_{]0,+\infty[}(x)$  & $c>0$, $k>0$,  & $(c,k)$ & -- \\
    % 
    \hline
    % 
    Chi & $\displaystyle x^{\nu-1}e^{-x^2/2}\frac{2^{1-\nu^{\strut}/2}}{\Gamma(\nu/2)_{\strut}} \boldsymbol{1}_{[0,+\infty[}(x)$ & $\nu > 0$& $\nu$ & -- \\
    % 
    \hline
    % 
    ChiSquare & $\displaystyle \frac{2^{-\nu^{\strut}/2}}{\Gamma(\nu/2)_{\strut}} x^{(\nu/2-1)}e^{-x/2}\boldsymbol{1}_{[0,+\infty[}(x)$ & $\nu > 0$& $\nu$ & -- \\
    % 
    \hline
    % 
    Dirichlet & $\displaystyle A \left[ 1-\sum_{j=1}^{d} x_j\right]^{(\theta_{d+1}-1)}\prod_{j=1}^d x_j^{(\theta_j-1)}\mathbf{1}_{\Delta}(\vect{x}) $ with $A = \frac{\Gamma(\sum_{j=1}^{d+1}\theta_j)}{\prod_{j=1}^{d+1}\Gamma(\theta_j)}$, $\Delta = \{ \vect{x} \in \Rset^d / \forall i, x_i \geq 0, \sum_{i=1}^{d} x_i \leq 1 \}$ & $d \geq 1$, $\theta_i>0$ & $(\theta_1, \hdots, \theta_{d+1})$  & -- \\
    % 
    \hline
    % 
    Epanechnikov & $\displaystyle \frac{3^{\strut}}{4_{\strut}}(1 - x^2)\boldsymbol{1}_{[-1,1]}(x)$ & -- & -- & -- \\
    % 
    \hline
    % 
    Exponential & $ \displaystyle \lambda e^{-\lambda(x-\gamma)^{\strut}}\boldsymbol{1}_{[\gamma,+\infty[_{\strut}}(x)$ & $\lambda>0$ & $(\lambda, \gamma)$ & -- \\
    % 
    \hline
    % 
    Fisher-Snedecor & $\displaystyle  \left[\left(\frac{d_1x}{d_1x+d_2}\right)^{d_1/2} \left(\frac{d_2}{d_1x+d_2}\right)^{{d_2/2}^{\strut}} \right]\frac{\mathbf{1}_{x \geq 0}}{Ax} $ with $A =  B(d_1/2, d_2/2)$ & $d_i>0$ & $\left(d_1, d_2 \right)_{\strut}$ & -- \\
    % 
    \hline
    % 
    Gamma & $ \displaystyle \frac{\lambda^{k^{\strut}}}{\Gamma(k)_{\strut}}(x-\gamma)^{(k-1)} e^{-\lambda(x-\gamma)}\boldsymbol{1}_{[\gamma,+\infty[}(x)$ & $k>0$, $\lambda > 0$ & $(k, \lambda, \gamma)$ & $(\mu, \sigma, \gamma)$ with
    $
    \left\{
      \begin{array}{l}
        \mu = \frac{k}{\lambda} +  \gamma \\
        \sigma = \frac{\sqrt{k}}{\lambda}
      \end{array}
    \right.
    $\\
    % 
    \hline
    % 
    \ifpdf % Manual splitting of the tabular in PDF mode
  \end{tabular}\\


  \noindent \begin{tabular}{|p{2cm}|p{5.3cm}|p{2.7cm}|p{1.7cm}|p{4.6cm}|}
    % 
    \hline
    % 
    Name & probability density function & conditions & param. 1 & param. $2^{\strut}_{\strut}$\\
    % 
    \hline
    % 
    Generalized Pareto & cumulative density function: $  \left\{
      \begin{array}{ll}
        \displaystyle F(x) =  1-\left[
          1+\xi\frac{x}{\sigma}\right]_{\strut}^{-1/\xi} & \mbox{if } \xi \neq 0 \\
        \displaystyle F(x) =  1-exp(-\frac{x}{\sigma}) & \mbox{if } \xi = 0
      \end{array}
    \right. $
    & $\sigma >0$  & $( \sigma, \xi)$ & -- \\
    % 
    \hline
    % 
    \fi % Manual splitting of the tabular in PDF mode
    Gumbel & $ \displaystyle \alpha e^{-\alpha(x-\beta)-e^{-\alpha(x-\beta)}}$ & $\alpha >0 $ & $(\alpha, \beta)$ & $(\mu, \sigma)$ with $
    \left\{
      \begin{array}{l}
        \mu = \frac{\gamma^{*}}{\alpha} + \beta \\
        \sigma = \frac{\pi}{\sqrt{6}} \frac{1}{\alpha}
      \end{array}
    \right.^{\strut}
    $
    where $ \displaystyle \gamma^{*} = -\int_0^{\infty} \log(t)e^{-t}dt$ is Euler's constant.
    or (a,b) with $
    \left\{
      \begin{array}{l}
        a = \beta \\
        b = \frac{1}{\alpha}
      \end{array}
    \right.^{\strut}
    $
    (param. 3)
    \\
    % 
    \hline
    % 
    Histogram & $ \displaystyle \sum_{i=1}^{n} h_i1_{[x_i, x_{i+1}]}(x)/S $ & $\begin{array}{lcl}
      l_i^{\strut} & = & x_{i+1} - x_i\\
      S & = & \sum_{i=1}^n h_i l_i \\
      l_i & \geq & 0
    \end{array}
    $ & $(x_1, (l_i, h_i))$ ${1\leq i \leq n} $ & -- \\
    % 
    \hline
    % 
    Inverse ChiSquare & $\displaystyle \dfrac{\exp \left( -\dfrac{1}{2 x}\right)}{\Gamma \left(\dfrac{\nu}{2}\right)\lambda^{\dfrac{\nu^{\strut}}{2}}x^{\dfrac{\nu}{2}+1}} \mathbf{1}_{x>0} $ & $\nu>0$& $(\nu)$ & -- \\
    % 
    \hline
    % 
    Inverse Gamma & $\displaystyle  \frac{\exp \left( -\dfrac{1}{\lambda x}\right)^{\strut}}{\Gamma^{\strut}(k)\lambda^kx^{k+1}} \mathbf{1}_{x>0}$ & $k>0$, $\lambda>0$ & $(k,\lambda)$ & -- \\
    % 
    \hline
    % 
    Inverse Normal & $\displaystyle \left(\frac{\lambda}{2\pi x^3} \right)^{1^{\strut}/2}e^{-\lambda(x-\mu)^2/(2\mu^2x)} \mathbf{1}_{x>0}$ & $\lambda>0$, $\mu>0$ & $(\lambda, \mu)$ & -- \\
    % 
    \hline
    % 
    InverseWishart &  $\displaystyle \frac{|\mat{V}|^{\frac{\nu}{2}}e^{-\frac{\mathrm{tr}(\mat{V}\mat{X}^{-1})^{\strut}}{2}}}{2^{\frac{\nu p}{2}}|\mat{X}|^{\frac{\nu+p+1}{2}}\Gamma_p\left(\frac{\nu}{2}\right)_{\strut}}\fcar{\cM_p^+(\Rset)}{\mat{X}}$ where $\cM_p^+(\Rset)$ is the set of symmetric positive matrices of dimension $p$ & $\mat{V}\in\cM_p^+(\Rset)$, $\nu>p-1$ &  $(\mat{V},\nu)$ & -- \\
    % 
    \hline
    % 
    Laplace & $ \displaystyle \frac{\lambda^{\strut}}{2_{\strut}}e^{-\lambda |x-\mu|}$ & $\lambda>0$ & $(\lambda, \mu)$ & -- \\
    % 
    \hline
    % 
    Logistic & $ \displaystyle \frac{e^{\left(-\frac{x-\alpha}{\beta}\right)^{\strut}}} {\beta\left(1+ e^{\left(-\frac{x-\alpha}{\beta}\right)}\right)^2_{\strut}}$ & $\beta > 0$ & $(\alpha, \beta)$ & -- \\
    % 
    \hline
    % 
    LogNormal & $ \displaystyle \frac{e^{-\frac{1}{2}\left(\frac{log(x-\gamma)-\mu_\ell}{\sigma_\ell}\right)^{2^{\strut}}}}{\sqrt{2\pi}\sigma_\ell (x-\gamma)}\boldsymbol{1}_{[\gamma,+\infty[}(x) $ & $\sigma_\ell >0$ & $(\mu_\ell, \sigma_\ell, \gamma)$ & $(\mu, \sigma, \gamma)$ or $(\mu, \frac{\sigma}{\mu}, \gamma)$ (param. 3) with $\mu > \gamma$, $\sigma > 0$. We have :
    $
    \left\{
      \begin{array}{@{}l@{}}
        \mu =  e^{\frac{1}{2}\sigma_\ell^2 + \mu_\ell} + \gamma\\
        \sigma =  (e^{\frac{1}{2}\sigma_\ell^2 + \mu_\ell})\sqrt{e^{\sigma_\ell^2}-1}
      \end{array}
    \right.
    $\\
    % 
    \hline
    % 
    LogUniform & $ \displaystyle \frac{1}{x(b_\ell-a_\ell)}\boldsymbol{1}_{[a_\ell, b_\ell]}(\log(x)) $ & $b_\ell > a_\ell$ & $(a_\ell, b_\ell)$ & -- \\
    % 
    \hline
    % 
    Meixner & $ \displaystyle  \frac{\left[ 2 \cos(\beta/2)\right]^{2\delta^{\strut}}}{2\alpha \pi \Gamma(2\delta)}e^{\frac{\beta(x-\mu)}{\alpha}}\left|\Gamma(\delta +i\frac{x-\mu}{\alpha})\right|^2$ with $i^2=-1$ & $\alpha>0$, $\beta \in ]-\pi, \pi[$, $\delta >0$& $ \left(\alpha, \beta, \delta, \mu \right)$& -- \\
    % 
    \hline
    % 
    Non Central Chi Square & $ \displaystyle \sum_{j=0}^{\infty} e^{-\lambda}\frac{\lambda^j}{j!}p_{\chi^2(\nu+2j)}(x)$ where $p_{\chi^2(q)}$ is the PDF of a $\chi^2(q)$ random variate.& $\nu>0$, $\lambda \geq 0$ & $(\nu,\lambda )$ & -- \\
    % 
    \hline
    \ifpdf % Manual splitting of the tabular in PDF mode
  \end{tabular}



  \noindent \begin{tabular}{|p{2cm}|p{5.3cm}|p{2.7cm}|p{1.7cm}|p{4.6cm}|}
    % 
    \hline
    % 
    Name & probability density function & conditions & param. 1 & param. $2^{\strut}_{\strut}$\\
    % 
    \hline
    % 
    Non Central Student & See text for $p_{NCS}(x)^{\strut}$ & -- & $(\nu,\delta, \gamma )$ & -- \\
    % 
    \hline
    % 
    Normal (nD) & $\displaystyle
    \frac{1}
    {
      \displaystyle (2\pi)^{\frac{n}{2}}|\mat{\Sigma}|^{\frac{1}{2}}
    }
    \displaystyle e^{-\frac{1}{2}\Tr{(\vect{x}-\vect{\mu})}\mat{\Sigma}^{-1^{\strut}}(\vect{x}-\vect{\mu})}$
    &
    $\mat{\Sigma} = \mat{\Lambda}_{\vect{\sigma}} \mat{R} \mat{\Lambda}_{\vect{\sigma}}$, $\mat{\Lambda}_{\vect{\sigma}} = \mathrm{diag}(\vect{\sigma})$, $\mat{R}$ SPD, $\sigma_i >0$ & $(\vect{\mu}, \vect{\sigma},\mat{R})$ or $(\vect{\mu}, \mat{\Sigma})$ & -- \\
    % 
    \hline
    % 
    Normal Gamma (2D) & See text for $p_{NG}(x,y)^{\strut}$
    & $\kappa>0, \alpha>0, \beta>0$ &
    $(\mu, \kappa, \alpha, \beta)$  & -- \\
    % 
    \hline
    % 
    Rayleigh & $\displaystyle \frac{(x - \gamma)^{\strut}}{\sigma^2_{\strut}}e^{-\frac{(x-\gamma)^2}{2\sigma^2}}\boldsymbol{1}_{[\gamma,+\infty[}(x)$ & $\sigma > 0$ & $(\sigma, \gamma)$ & -- \\
    % 
    \hline
    % 
    Rice & $\displaystyle 2\frac{x}{\sigma_{\strut}^{2^{\strut}}}p_{\chi^2(2,\frac{\nu^2}{\sigma^2})}(\frac{x^2}{\sigma^2})$ where $p_{\chi^2(\nu, \lambda)}$ is the PDF of a Non Central Chi Square random variate $(\nu, \lambda)$. & $\nu \geq 0, \sigma>0$ & $( \sigma, \nu)$ & -- \\
    % 
    \hline
    % 
    Student (nD) & See text for $p_T(x)^{\strut}$ & $\nu > 2$ & $(\nu,\vect{\mu}, \vect{\sigma}, \mat{R}_{\strut} )$ & $(\nu, \mu, \sigma)$ with $d=1$ \\
    % 
    \hline
    % 
    Trapezoidal & $  \left\{
      \begin{array}{ll}
        \displaystyle h \frac {x-a}{b-a} & \textrm{if}\ a\leq x < b \\
        \displaystyle h & \textrm{if}\ b\leq x < c \\
        \displaystyle h \frac{d-x}{d-c}& \textrm{if}\ c\leq x < d \\
        0 & \textrm{otherwise}
      \end{array}
    \right. $ with  $h=\frac{2}{d+c-a-b}$
    &  $a\leq b < c\leq d$ & $(a, b, c, d)$ & -- \\
    % 
    \hline
    % 
    \fi % Manual splitting of the tabular in PDF mode
    Triangular & $  \displaystyle \left\{
      \begin{array}{ll}
        \displaystyle 2\frac{x-a}{(m-a)(b-a)} & a \leq x \leq m \\
        \displaystyle 2\frac{b-x}{(b-m)(b-a)} & m \leq x \leq b \\
        0 & \mbox{otherwise.}
      \end{array}
    \right.^{\strut} $ & $a < m < b$ & $(a, m, b)$ & -- \\
    % 
    \hline
    % 
    Truncated Normal & $  \displaystyle \frac{\frac{1^{\strut}}{\sigma_n}\phi(\frac{x-\mu_n}{\sigma_n})}
    {\Phi(\frac{b-\mu_n}{\sigma_n}) - \Phi(\frac{a-\mu_n}{\sigma_n})}\boldsymbol{1}_{[a, b]}(x)$
    & $\sigma_n >0$ & $(\mu_n, \sigma_n, a, b)$ & -- \\
    % 
    \hline
    % 
    Uniform & $  \displaystyle \frac{1^{\strut}}{b-a}\boldsymbol{1}_{[a, b]}(x)$ & $a < b$ & $(a, b)$ & -- \\
    % 
    \hline
    % 
    Weibull &  $\displaystyle \frac{\beta^{\strut}}{\alpha}\left(\frac{x-\gamma}{\alpha}\right)^{\beta-1}e^{-\left(\frac{x-\gamma}{\alpha}\right)^\beta} \boldsymbol{1}_{[\gamma,+\infty[}(x)$ & $\alpha>0$, $\beta>0$ &  $(\alpha, \beta, \gamma)$ & $(\mu, \sigma, \gamma)$ with
    $
    \left\{
      \begin{array}{@{}l@{}}
        \mu =  \alpha\Gamma(1+\frac{1}{\beta}) + \gamma\\
        \sigma =  \alpha\sqrt{\Gamma(1+\frac{2}{\beta}) - \Gamma^2(1+\frac{1}{\beta})}
      \end{array}
    \right._{\strut}
    $ \\
    % 
    \hline
    %
    Wishart &  $\displaystyle \frac{|\mat{X}|^{\frac{\nu-p-1}{2}}e^{-\frac{\mathrm{tr}(\mat{V}^{-1}\mat{X})^{\strut}}{2}}}{2^{\frac{\nu p}{2}}|\mat{V}|^{\frac{\nu}{2}}\Gamma_p\left(\frac{\nu}{2}\right)_{\strut}}\fcar{\cM_p^+(\Rset)}{\mat{X}}$ where $\cM_p^+(\Rset)$ is the set of symmetric positive matrices of dimension $p$ & $\mat{V}\in\cM_p^+(\Rset)$, $\nu>p-1$ &  $(\mat{V},\nu)$ & -- \\
    % 
    \hline
    % 
  \end{tabular}

} % end of footnotesize



\vspace*{0.2cm}
Let's note that a random variable $X$ is said to have a  {\bf standard non-central student distribution} $\cT(\nu, \delta)$ if it can be written as:
\begin{equation}
  X = \frac{N}{\sqrt{C/\nu}}
\end{equation}
where $N$ has the normal distribution $\cN(\delta, 1)$ and $C$ has the $\chi^2(\nu)$ distribution, $N$ and $C$ being independent.\\
The non-central Student distribution in OpenTURNS has an additional parameter $\gamma$ such that the random variable $X$ is said to have a non-central Student distribution $\cT(\nu, \delta, \gamma)$ if $X-\gamma$ has a standard $\cT(\nu,\delta)$ distribution.\\

The probability density function of the Non Central Student writes:
\begin{align*}
  p_{NCS}(x) = \frac{\exp(-\delta^2 / 2)}{\sqrt{\nu\pi} \Gamma(\nu / 2)}\left(\frac{\nu}{\nu + (x-\gamma)^2}\right) ^ {(\nu + 1) / 2} \sum_{j=0}^{\infty} \frac{\Gamma\left(\frac{\nu + j + 1}{2}\right)}{\Gamma(j + 1)}\left(\delta(x-\gamma)\sqrt{\frac{2}{\nu + (x-\gamma)^2}}\right) ^ j
\end{align*}

The  {\bf Student} distribution has the following  probability density function, written en dimension $d$ :
\begin{align*}
  p_T(\vect{x}) = \frac{\Gamma\left(\frac{\nu+d}{2}\right)}
  {(\pi d)^{\frac{d}{2}}\Gamma\left(\frac{\nu}{2}\right)}\frac{|\mat{R}|^{-1/2}}{\prod_{k=1}^d\sigma_k}\left(1+\frac{\vect{z}^t\mat{R}^{-1}\vect{z}}{\nu}\right)^{-\frac{\nu+d}{2}}
\end{align*}
where $\vect{z}=\mat{\Delta}^{-1}\left(\vect{x}-\vect{\mu}\right)$ with $\mat{\Delta}=\mat{\mathrm{diag}}(\vect{\sigma})$.\\

In dimension $d=1$ we have the following expression :
\begin{align*}
  \displaystyle p_T(x) = \frac{\Gamma\left(\frac{\nu+1}{2}\right)}
  {\sqrt{\pi}\Gamma\left(\frac{\nu}{2}\right)}\frac{1}{\sigma}\left(1+\frac{(x-\mu)^2}{\nu}\right)^{-\frac{\nu+1}{2}}
\end{align*}

The {\bf Normal Gamma} distribution is the distribution of the random vector $(X,Y)$ where $Y$ follows the distribution $\Gamma(\alpha, \beta)$ with $\alpha>0$ and $\beta>0$, $X|Y$ follows the distribution $\mathcal{N}(\mu, \dfrac{1}{\sqrt{\kappa Y}})$. Its probability density function writes:

\begin{align*}
   p_{NG}(x,y) =  \dfrac{\Gamma(\alpha)}{\beta^\alpha}\sqrt{\dfrac{2\pi}{\kappa}}y^{\alpha-1/2}\exp\left(-\dfrac{y}{2}\left[\kappa(x-\mu)^2+2\beta\right]\right)
\end{align*}


The {\bf Inverse ChiSquare} distribution parametered by $\nu$, with $\nu>0$ is the distribution of the random variable $X$ such that $\displaystyle \frac{1}{X}$ follows the $ChiSquare(\nu)$ distribution.\\
Note also that the Inverse ChiSquare distribution parametered by $\nu$ is exactly the Inverse Gamma$(\dfrac{\nu}{2}, 2)$ distribution.


The {\bf Inverse Gamma} distribution parametered by $(k,\lambda)$, with $k>0$ and~$\lambda>0$, is the distribution of the random variable $X$ such that $\displaystyle \frac{1}{X}$ follows the $\Gamma(k, \frac{1}{\lambda})$ distribution.





\begin{itemize}
\item Discrete distributions :
\end{itemize}

{\footnotesize
  \noindent \begin{tabular}{|p{2cm}|p{8cm}|p{4cm}|p{2cm}|}
    % 
    \hline
    % 
    Name & Distribution & \multicolumn{1}{l|}{conditions} & param. $1^{\strut}_{\strut}$\\
    % 
    \hline
    % 
    Bernoulli & $\displaystyle P(X = 1)^{\strut} = p, P(X = 0) = 1-p $ & $p \in [0,1]$ & $p$\\
    % 
    \hline
    % 
    Binomial & $\displaystyle P(X = k) = C_n^k p^k (1-p)^{{n-k}^{\strut}}$ &
    $\begin{array}{@{}l@{}}
      k^{\strut} \in \{0, \hdots, n\} \\
      n \in \Nset \\
      p \in [0,1]
    \end{array}
    $
    & $(n,p)$\\
    % 
    \hline
    % 
    Dirac & $ \Prob{\vect{X} = \vect{point}_{\strut}^{\strut}} = 1$ & - & \textit{point}\\
    % 
    \hline
    % 
    Geometric & $\displaystyle P(X = k) = p(1-p)^{k^{\strut}-1}_{\strut}$ & $k \in \Nset^{*}$ & $p$\\
    % 
    \hline
    % 
  \end{tabular}
}



{\footnotesize
  \noindent \begin{tabular}{|p{2cm}|p{8cm}|p{4cm}|p{2cm}|}
    % 
    \hline
    % 
    Name & Distribution & \multicolumn{1}{l|}{conditions} & param. $1^{\strut}_{\strut}$\\
    % 
    \hline
    % 
    KPermutations Distribution & $\displaystyle \Prob{\vect{X} =
      point}=1/d$ whith $d=A_n^k=\displaystyle \frac{n!}{(n-k)!}$
    and \textit{point} is an injective function $(i_0, \hdots, i_{k_1})$ from $\{0, \dots, k-1\}$ into $\{0, \dots, n-1\}$ &
    $k \geq 1, n\geq 1$ & $(k,n)$\\
    % 
    \hline
    % 
    Multinomial (nD) & $\displaystyle P(\vect{X} = \vect{x}) = \frac{N!}{x_1!\dots x_n! (N-s)!}p_1^{x_1}\dots p_n^{x_n}(1-q)^{N-s}$ &
    $\begin{array}{@{}l@{}}
      0^{\strut}\leq p_i \leq 1 \\
      x_i\in \Nset \\
      \displaystyle q = \sum_{k=1}^n p_k \leq 1 \\
      s=  \sum_{k=1}^n x_k \leq N_{\strut}
    \end{array}
    $
    & $((p_k)_{1 \leq k \leq n}, N)$\\
    % 
    \hline
    % 
    Negative Binomial & $\displaystyle P(X = k) = \frac{\Gamma(k + r)}{\Gamma(r)\Gamma(k+1)}p^k (1-p)^{r^{\strut}}$ &
    $\begin{array}{@{}l@{}}
      k^{\strut} \in \Nset\\
      r \in (0,+\infty) \\
      p \in (0,1)
    \end{array}
    $
    & $(r,p)$\\
    % 
    \hline
    % 
    Poisson & $ \displaystyle P(X = k) =  \frac{\lambda^{k^{\strut}}}{k!_{\strut}}e^{-\lambda}$ & $k \in \Nset$, $\lambda >0$ & $\lambda$ \\
    % 
    \hline
    % 
    Skellam & $ \displaystyle \Prob{X = k}^{\strut} = 2\Prob{Y=2\lambda_1}$, $Y \sim  \chi^2_{\nu=2(k+1), \delta=2_{\strut}\lambda_2}$& $k \in \Zset$, $\lambda_i >0$ & $(\lambda_1, \lambda_2)$ \\
    % 
    \hline
    % 
    User defined (nD) &  $P(\vect{X} = \vect{x}_k) = p_k)_{1 \leq k \leq N}$ & $0\leq p_k \leq 1$, $\displaystyle \sum_{k=1}^{N^{\strut}} p_k = 1$ & $(\vect{x_k}, p_k)_{1 \leq k \leq N}$\\
    % 
    \hline
    % 
    Zipf-Mandelbrot &  $P(X=k) = \frac{1}{(k+q)^s} \frac{1}{H(N,q,s)}$ $ \forall k\in [1,N]$, where $H(N,q,s) = \sum_{i=1}^{N} \displaystyle \frac{1}{(i+q)^s}$ (Generalized Harmonic Number) & $0\leq p_k \leq 1$, $\displaystyle \sum_{k=1}^{N^{\strut}} p_k = 1$ & $N \geq 1$, $q \geq 0$, $s>0$ \\
    % 
    \hline
    % 
  \end{tabular}
}

\textspace\\
Let's note that in dimension 1, the Multinomial distribution is the Binomial distribution $B(n,p)$ described as :
\begin{align*}
  \forall k \in  \Nset, P(X=k) = C^k_n p^k (1-p)^{n-k}.
\end{align*}

Furthermore, for all these 1D usual distributions, it is possible to truncate them within $[a,b]$, $[a, +\infty[$ or $]-\infty, b]$ (see UC.\ref{truncatedistribution}).

\textspace\\
\noindent%
\requirements{
  \begin{description}
  \item[$\bullet$] none
  \end{description}
}
{
  \begin{description}
  \item[$\bullet$] the random input distribution
  \item[type:] Distribution
  \end{description}
}

\textspace\\
Python  script for this UseCase :

\inputscript{script_docUC_InputNoData_UsualDist}

\textspace\\

Refer to the Reference Guide to get graphs of the distributions pdf.
