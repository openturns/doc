% Copyright 2005-2016 Airbus-EDF-IMACS-Phimeca
% Permission is granted to copy, distribute and/or modify this document
% under the terms of the GNU Free Documentation License, Version 1.2
% or any later version published by the Free Software Foundation;
% with no Invariant Sections, no Front-Cover Texts, and no Back-Cover
% Texts.  A copy of the license is included in the section entitled "GNU
% Free Documentation License".
\renewcommand{\filename}{docUC_InputWithData_CSV.tex}
\renewcommand{\filetitle}{UC : Import / Export data from a file at format CSV (Comma Separated Value)}

% \HeaderNNIILevel
% \HeaderIILevel
\HeaderIIILevel




\index{CSV file}

The objective of this Use Case is to import a file at format CSV containing a list of data and to export a NumericalSample into a file at format CSV. \\


To be a proper sample file, the following rules must be respected :
\begin{itemize}
\item Data are presented in line : each line corresponds to the realization of the random vector. The number of lines is the size of the sample. The number of data on each line is the dimension of the sample.
\item Data must be separated by the same specific character, ";" by default. To change the separator, you must use either the ResourceMap class or specify it in the \emph{export} or \emph{Import} methods.
\item If a line does not have the same number of data as the first valid line in the file, it is disregarded.
\item The format of a data is either an integer value (2 or -5 for example), a floating-point value in decimal notation (-1.23 or 4.56 for example) or in scientific notation (-1.2e3 or 3.4e-5 for example).
\end{itemize}

When a line presents an error, the line is ignored but all the right ones are taken into account. The number of lines which don't follow the previous rules are signaled and the reason of the discard is given in the logs. To see then, you must use the Log class. There can be any number of white spaces or tabulations between the data and the separator, and the lines can be ended in a UNIX-like fashion or a Windows-like fashion.

\textspace\\
\requirements{
  \begin{description}
  \item[$\bullet$] a file containing data : {\itshape sampleFile.csv}
  \item[type:] a CSV format file respecting rules explicited before
  \item[$\bullet$] or a numerical sample to be stored : {\itshape mySampleToBeStored}
  \item[type:] a NumericalSample
  \end{description}
}
             {
               \begin{description}
               \item[$\bullet$] the sample issued from the data file {\itshape sampleFile.csv}: {\itshape aSample}
               \item[type:]  a NumericalSample
               \item[$\bullet$]  a file containing{\itshape mySampleToBeStored}: {\itshape mySampleStoredFile.csv}
               \item[type:]  a CSV format file respecting rules explicited before
               \end{description}
             }

             \textspace\\
             Python script for this UseCase :

             \inputscript{script_docUC_InputWithData_CSV}
