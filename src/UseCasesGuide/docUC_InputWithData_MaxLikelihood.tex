% Copyright 2005-2016 Airbus-EDF-IMACS-Phimeca
% Permission is granted to copy, distribute and/or modify this document
% under the terms of the GNU Free Documentation License, Version 1.2
% or any later version published by the Free Software Foundation;
% with no Invariant Sections, no Front-Cover Texts, and no Back-Cover
% Texts.  A copy of the license is included in the section entitled "GNU
% Free Documentation License".
\renewcommand{\filename}{docUC_InputWithData_MaxLikelihood.tex}
\renewcommand{\filetitle}{UC : Maximum likelihood of a given probability density function }

% \HeaderNNIILevel
% \HeaderIILevel
\HeaderIIILevel



\index{Fitting Distribution! Maximum likelihood}


The objective of this Use Case is to explicitate how to implement the evaluation of the parameters of a fitted distribution thanks to the Maximum Likelihood Principle.\\


Details on the Maximum Likelihood  Principle may be found in the Reference Guide (\extref{ReferenceGuide}{see files Reference Guide - Step B -- Maximum Likelihood  Principle}{stepB}).\\

Let us denote $(\vect{x}_1, \dots, \vect{x}_n)$ the sample, $p_{\vect{\theta}}$ the particular distribution of probability density function we want to fit to the sample, and  $\vect{\theta} \in \Theta \in \Rset^p$ its the parameter vector.\\

The likelihood of the sample according to  $p_{\vect{\theta}}$ is :
\begin{align*}
  likelihood(\vect{x}_1, \dots, \vect{x}_n,\vect{\theta}) = \prod_{i=1}^n p_{\vect{\theta}}(\vect{x}_i)
\end{align*}
It may be implemented :
\begin{itemize}
\item either as a python function thanks to the class OpenTURNSPythonFunction, as explained in the UC.\ref{OpenTURNSPythonFunction},
\item or thanks to the NumericalMathFunction class, as explained in the UC.\ref{NumericalMathFunction}.
\end{itemize}

The maximum likelihood estimation $\vect{\theta}_{MLE}$ is solution of the equation :
\begin{align*}
  \max_{\vect{\theta} \in \Theta} likelihood\, (\vect{x}_1, \dots, \vect{x}_n,\vect{\theta})
\end{align*}
or
\begin{align*}
  \max_{\vect{\theta} \in \Theta}  \log likelihood\, (\vect{x}_1, \dots, \vect{x}_n,\vect{\theta})
\end{align*}


The following UC illustrates the example of a Normal distribution, where $(\mu, \sigma) \in \Rset \times \Rset^+$ are defined through the maximum likelihood principle, where the log of the likelihood funciton  is implemented thanks to the OpenTURNSPythonFunction class.\\

Note that to avoid underflow problems with the log of the likelihood function, it is necessary to bound the probability density function to a min value \textit{eps}, here equal to $1.0e-16$. The example illustrates how to proceed.\\


\requirements{
  \begin{description}
  \item[$\bullet$] a sample of data : {\itshape sample}
  \item[type:]  a NumericalSample
  \end{description}
}
             {
               \begin{description}
               \item[$\bullet$] the log likelihood function in the openturns library : {\itshape myLogLikelihoodOT}
               \item[type:] a NumericalMathFunction
               \item[$\bullet$] the  MLE of $(\mu, \sigma)$ : {\itshape optimizer}
               \item[type:] a NumericalPoint
               \end{description}
             }

             \textspace\\
             Python script for this UseCase :

             \inputscript{script_docUC_InputWithData_MaxLikelihood}
