% Copyright 2005-2016 Airbus-EDF-IMACS-Phimeca
% Permission is granted to copy, distribute and/or modify this document
% under the terms of the GNU Free Documentation License, Version 1.2
% or any later version published by the Free Software Foundation;
% with no Invariant Sections, no Front-Cover Texts, and no Back-Cover
% Texts.  A copy of the license is included in the section entitled "GNU
% Free Documentation License".
\renewcommand{\filename}{docUC_InputWithData_StatisticalManipulation.tex}
\renewcommand{\filetitle}{UC :  Statistical manipulations on data : min, max, covariance, skewness, kurtosis, quantile, empirical CDF, Pearson, Kendall and Spearman correlation matrixes and rank/sort functionnalities}

% \HeaderNNIILevel
% \HeaderIILevel
\HeaderIIILevel

\label{statistical}


\index{Sample Statistics!Min - Max}
\index{Sample Statistics!Moments evaluation}
\index{Sample Statistics!Covariance}
\index{Sample Statistics!Skewness}
\index{Sample Statistics!Kurtosis}
\index{Sample Statistics!Empirical quantile}
\index{Sample Statistics!Empirical cumulative density function}
\index{Sample Statistics!Pearson correlation coefficient}
\index{Sample Statistics!Kendall's tau}
\index{Sample Statistics!Spearman correlation coefficient}
\index{Sample Statistics! Rank - Sort functionnalities}
\index{Sample Statistics!Cholesky factor}
\index{Quantile!Empirical estimation}



The objective of this Use Case is to describe the main statistical functionalities that OpenTURNS enables to manipulate some data, represented by a NumericalSample.\\





OpenTURNS enables to calculate per components :
\begin{itemize}
\item min and max per component, with the methods {\itshape getMin, getMax}
\item range per component, with the method {\itshape computeRangePerComponent}
\item mean, variance, standard deviation , skewness and kurtosis  per component, with the methods {\itshape computeMean, computeVariancePerComponent, computeStandardDeviationPerComponent, computeSkewnessPerComponent, computeKurtosisPerComponent}
\item empirical median and other quantiles per component, with the methods {\itshape computeMedianPerComponent, computeQuantilePerComponent}
\end{itemize}

OpenTURNS enables some global calculs :
\begin{itemize}
\item covariance of the sample, with the methods {\itshape computeCovariance},
\item standard deviation of the sample : the Cholesky factor of the covariance matrix, with the methods {\itshape computeStandardDeviation },
\item Pearson, Kendall and Spearman correlation matrix, with the methods {\itshape computePearsonCorrelation, computeKendallTau, computeSpearmanCorrelation},
\item empirical CDF evaluated on a point $\vect{x}$ : $\Prob{X_1 \leq x_1, \hdots, X_n \leq x_n}$, with the methods {\itshape computeEmpiricalCDF},
\item empirical tail CDF evaluated on a point $\vect{x}$ : $\Prob{X_1 > x_1, \hdots, X_n > x_n}$, with the methods {\itshape computeEmpiricalCDF},
\item empirical quantiles, with the method {\itshape computeQuantile}. For the quantile $x_q$ of order $q$, OpenTURNS proceeds as follows :
  \begin{itemize}
  \item $\forall q \in [\frac{1}{2n}, 1-\frac{1}{2n}]$, then OpenTURNS approximates the empirical  cumulative density function by interpolating all the middles of the steps and then evaluates  $x_q$ from this continuous approximation.
  \item $\forall q \leq \frac{1}{2n}$, then OpenTURNS returns $min(X_i)$.
  \item $\forall q > \frac{1}{2n}$, then OpenTURNS returns $max(X_i)$.
  \end{itemize}
\end{itemize}

At last, it is possible :
\begin{itemize}
\item to copy into a NumericalSample whose components are the respective ranks of the components, with the method {\itshape rank},
\item to copy into a NumericalSample whose components are all sorted in ascending order, with the method {\itshape sort },
\item to extract the $(i+1)$ component whose components are all sorted in ascending order, with the method {\itshape sort(i) },
\item to copy into a NumericalSample whose NumericalPoints are reordered such that the $(i+1)$ component is sorted in ascending order, with the method  {\itshape sortAccordingAComponent(i)},
\item to keep from the Numericalsample only the $i$ first points, with the method  {\itshape split(i)},
\item to translate the points of the NumericalSample, with the method  {\itshape translate },
\item to multiply all the components of the points by a factor, with the method  {\itshape  scale},
\item to remove a particular point from the NumericalSample, with the method  {\itshape  erase}.
\end{itemize}



\requirements{
  \begin{description}
  \item[$\bullet$] a numerical sample : {\itshape sample}
  \item[type:]  NumericalSample
  \end{description}
}
             {
               \begin{description}
               \item[$\bullet$] statistical elements listed previously
               \item[type:]  NumericalPoint, SquareMatrix or CorrelationMatrix
               \end{description}
             }

             \textspace\\
             Python script for this UseCase :

             \inputscript{script_docUC_InputWithData_StatisticalManipulation}

             \textspace\\



             To illustrate each method, we give here an example in dimension 2 : consider the following NumericalSample \textit{numSample = [(1.3, 1.2); (4.1, 1.0); (2.3, 2.7)]}. Then,

             At last, it is possible :
             \begin{itemize}
             \item \textit{new = numSample.rank()}: \textit{new = [(0,1); (2,0); (1,2)]},
             \item \textit{new = numSample.sort()}: \textit{new = [(1.3,1.0); (2.3,1.2); (4.1,2.7)]},
             \item \textit{new = numSample.sort(0))}: \textit{new = [(1.3); (2.3); (4.1)]},
             \item \textit{new = numSample.sortAccordingAComponent(1)}: \textit{new  = [(4.1, 1.0);(1.3, 1.2);(2.3, 2.7)]},
             \item \textit{new = numSample.split(2)}: \textit{new  = [(2.3, 2.7)]} and \textit{numSample = [(1.3, 1.2); (4.1, 1.0)]},
             \item \textit{new = numSample.translate(NumericalPoint(2,1.0))}: \textit{new  = [(2.3, 2.2); (4.1, 2.0); (3.3, 3.7)]},
             \item \textit{new = numSample.scale(NumericalPoint(2,2.0))}: \textit{new  = [(2.6, 2.4); (8.2, 2.0); (4.6, 5.4)]},
             \item \textit{new = numSample.erase(1)}: \textit{new  = [(1.3, 1.2); (2.3, 2.7)]}.
             \end{itemize}
