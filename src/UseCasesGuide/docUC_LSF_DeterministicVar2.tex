% Copyright 2005-2016 Airbus-EDF-IMACS-Phimeca
% Permission is granted to copy, distribute and/or modify this document
% under the terms of the GNU Free Documentation License, Version 1.2
% or any later version published by the Free Software Foundation;
% with no Invariant Sections, no Front-Cover Texts, and no Back-Cover
% Texts.  A copy of the license is included in the section entitled "GNU
% Free Documentation License".
\renewcommand{\filename}{docUC_LSF_DeterministicVar2.tex}
\renewcommand{\filetitle}{UC : Introducing some deterministic variables, optimizing memory and CPU time}

% \HeaderNNIILevel
% \HeaderIILevel
\HeaderIIILevel



\index{Limit State Function!Reducing the initial limit state function}
\index{Random Vector!Output random vector}

The objective of this Use Case is to restrict a model which has been previously declared, to some of its variables, through an optimized way.\\

Let's have the same context than in the UC\ref{linearNumericalMathFunction}. The idea here is to avoid the introduction of the potentially huge matrix $\mat{A}$ and the gradient matrix and hessian tensor  of the functions \textit{increase} and \textit{poutre}. For that last problem, it is sufficient to define the gradient matrix and hessian tensor to the final function \textit{poutreReduced} from a finite difference technique.\\

The function {\itshape increase} is defined as follows :
\begin{align*}
  \begin{array}{l|lcl}
    increase : & \Rset^{n_{prob}}  & \rightarrow & \Rset^n \\
    &  \vect{X}_{prob} =
    \begin{array}{|l}
      "inputProb1" \\
      \cdots       \\
      "inputProbNprob"
    \end{array}
    & \mapsto     & increase(\vect{X}_{prob}) =
    \begin{array}{|l}
      "inputProb1" \\
      \cdots       \\
      "inputProbNprob" \\
      valDet1 \\
      \cdots       \\
      valDetNdet
    \end{array}
  \end{array}
\end{align*}

where all the $(valDet1, ..., valDetNdet)$ are the $n_{det}$ values of the determinist components of $\vect{X}$.\\

The same example is re-written in the folloing Use Case.

\requirements{
  \begin{description}
  \item[$\bullet$] the initial limit state function : {\itshape poutre}
  \item[type:] LinearNumericalMathFunction $(\Rset^4 \rightarrow \Rset)$
  \end{description}
}
{
  \begin{description}
  \item[$\bullet$] the {\itshape increase} function
  \item[type:] NumericalMathFunction $(\Rset^2 \rightarrow \Rset^4)$
  \item[$\bullet$]  the new limit state function : {\itshape poutreReduced = poutre $\circ$ increase}
  \item[type:] NumericalMathFunction $(\Rset^2 \rightarrow \Rset)$
  \end{description}
}

\textspace\\
Python script for this UseCase :

\inputscript{script_docUC_LSF_DeterministicVar2}
