% Copyright 2005-2016 Airbus-EDF-IMACS-Phimeca
% Permission is granted to copy, distribute and/or modify this document
% under the terms of the GNU Free Documentation License, Version 1.2
% or any later version published by the Free Software Foundation;
% with no Invariant Sections, no Front-Cover Texts, and no Back-Cover
% Texts.  A copy of the license is included in the section entitled "GNU
% Free Documentation License".
\renewcommand{\filename}{docUC_LSF_DynamicalFunction.tex}
\renewcommand{\filetitle}{UC : Creation of a Dynamical Function}

% \HeaderNNIILevel
% \HeaderIILevel
\HeaderIIILevel

\label{UCDynamicalFunction}


\index{Limit State Function!Dynamical Function}


The objective here is to create dynamical functions, that can act on fields.\\
OpenTURNS defines a new function called {\itshape dynamical function} and two particular dynamical functions: the {\itshape spatial function} and the {\itshape temporal function}.\\

{\bf Dynamical function}:\\
A dynamical function $f_{dyn}:\cD \times \Rset^d \rightarrow \cD' \times \Rset^q$ where $\cD \in \Rset^n$ and  $\cD' \in \Rset^p$ is defined by:
\begin{align}\label{dynFct}
  f_{dyn}(\vect{t}, \vect{x}) = (t'(\vect{t}), v'(\vect{t}, \vect{x}))
\end{align}
with $t': \cD \rightarrow \cD'$ and $v': \cD \times \Rset^d \rightarrow \Rset^q$.\\

A dynamical function $f_{dyn}$ transforms a multivariate stochastic process:
\begin{align}
  X: \Omega \times \cD \rightarrow \Rset^d
\end{align}
where $\cD \in \Rset^n$ is discretized according to the $\cM$ into the multivariate stochastic process:
\begin{align}
  Y=f_{dyn}(X)
\end{align}
such that:
\begin{align*}
  Y: \Omega \times \cD' \rightarrow \Rset^q
\end{align*}
where the mesh $\cD' \in \Rset^p$  is discretized according to the $\cM'$.\\

A dynamical function $f_{dyn}$ also acts on fields and produces fields of possibly different dimension ($q\neq d$) and mesh ($\cD \neq \cD'$ or $\cM \neq \cM'$).



OpenTURNS v1.3 only proposes dynamical functions where $\cD'=\cD$ and $\cM'=\cM$ which means that $t'=Id$ through the {\itshape spatial function} and the {\itshape temporal function}. It follows that the process $Y$ shares the same mesh with $X$, only its values have changed.\\

{\bf Spatial function}:\\
A spatial function $f_{spat}: \cD \times \Rset^d \rightarrow \cD \times \Rset^q$ is a particular dynamical function that lets invariant the mesh of a field and defined by a function $g : \Rset^d  \rightarrow \Rset^q$ such that:
\begin{align}\label{spatFunc}
  f_{spat}(\vect{t}, \vect{x})=(\vect{t}, g(\vect{x}))
\end{align}

Let's note that the input dimension of $f_{spat}$  still designs the dimension of $\vect{x}$ : $d$. Its output dimension is equal to $q$.\\

The creation of the \emph{SpatialFunction} object of OpenTURNS requires the numerical math function $g$ and the integer $n$: the dimension of the vertices of the mesh $\cM$. This data is required for tests on the compatibility of dimension when a composite process is created using the spatial function.\\


{\bf Temporal function}:\\
A temporal function $f_{temp}: \cD \times \Rset^d \rightarrow \cD \times \Rset^q$ is a particular dynamical function that lets invariant the mesh of a field and defined by a function $h :  \Rset^n \times \Rset^d  \rightarrow \Rset^q$ such that:
\begin{align}\label{tempFunc}
  f_{temp}(\vect{t}, \vect{x})=(\vect{t}, h(\vect{t},\vect{x}))
\end{align}

Let's note that the input dimension of $f_{temp}$  still design the dimension of $\vect{x}$ : $d$. Its output dimension is equal to $q$.\\


The creation of the \emph{TemporalFunction} object of OpenTURNS requires the numerical math function $h$ and the integer $n$: the dimension of the vertices of the mesh $\cM$.\\



The use case illustrates the creation of a spatial (dynamical) function from the function $g: \Rset^2  \rightarrow \Rset^2$ such as :
\begin{align}
  g(\vect{x})=(x_1^2, x_1+x_2)
\end{align}

and the creation  of a temporal (dynamical) function from the function $h:\Rset^2\times\Rset^2$ such as :
\begin{align}
  h(\vect{t}, \vect{x})=(t_1+t_2+x_1^2+x_2^2)
\end{align}

\requirements{
  \begin{description}
  \item[$\bullet$] some functions {\itshape g,h}
  \item[type:] NumericalMathFunction
  \end{description}
}
             {
               \begin{description}
               \item[$\bullet$] a spatial function {\itshape mySpatialFunction}
               \item[type:] SpatialFunction
               \item[$\bullet$] a temporal function {\itshape myTemporalFunction}
               \item[type:] TemporalFunction
               \item[$\bullet$] two dynamical functions {\itshape myDynamicalFunctionFromSpatial} and {\itshape myDynamicalFunctionFromTemporal}
               \item[type:] DynamicalFunction
               \end{description}

             }

             \textspace\\
             Python script for this use case :

             \inputscript{script_docUC_LSF_DynamicalFunction}
