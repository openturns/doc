% Copyright 2005-2016 Airbus-EDF-IMACS-Phimeca
% Permission is granted to copy, distribute and/or modify this document
% under the terms of the GNU Free Documentation License, Version 1.2
% or any later version published by the Free Software Foundation;
% with no Invariant Sections, no Front-Cover Texts, and no Back-Cover
% Texts.  A copy of the license is included in the section entitled "GNU
% Free Documentation License".
\renewcommand{\filename}{docUC_MinMax_ExpPlaneDrawing.tex}
\renewcommand{\filetitle}{UC : Drawing an design of experiments  in dimension 2 }

% \HeaderNNIILevel
% \HeaderIILevel
\HeaderIIILevel

\index{Design of Experiments !Drawing}
\index{Graph Manipulation!View}




The objective of this Use Case is to draw an design of experiments  in dimension 2.\\


\requirements{
  \begin{description}
  \item[$\bullet$] the points of an design of experiments  : {\itshape mySample}
  \item[type:] a NumericalSample
  \end{description}
}
             {
               \begin{description}
               \item[$\bullet$] the files containing the graph, in format .EPS, .FIG, .PNG : {\itshape DoE}
               \item[type:] -
               \end{description}
             }

             \textspace\\
             Python script for this UseCase :


             \begin{lstlisting}
               # Draw it
               from openturns.viewer import View
               mySampleDrawable = Cloud(mySample, "blue", "square", "My design of experiments")
               graph = Graph("My design of experiments", "x", "y", True)
               graph.add(mySampleDrawable)
               view = View(graph)
               view.save('DoE.png')
               view.show()

               # In order to see the drawable without creating the associated files
               # CARE : it requires to have created the graph structure before
               View(mySampleDrawable).show()
               # or to see the graph without creating the associated files
               View(graph).show()
             \end{lstlisting}
