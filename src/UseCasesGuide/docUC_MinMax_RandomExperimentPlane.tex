% Copyright (C) 2005-2015 Airbus - EDF - IMACS - Phimeca
% Permission is granted to copy, distribute and/or modify this document
% under the terms of the GNU Free Documentation License, Version 1.2
% or any later version published by the Free Software Foundation;
% with no Invariant Sections, no Front-Cover Texts, and no Back-Cover
% Texts.  A copy of the license is included in the section entitled "GNU
% Free Documentation License".
\renewcommand{\filename}{docUC_MinMax_RandomExperimentPlane.tex}
\renewcommand{\filetitle}{UC: Creation of a random design of experiments : Monte Carlo, LHS patterns}

% \HeaderNNIILevel
% \HeaderIILevel
\HeaderIIILevel

\label{randomExpPlane}


\index{Design of Experiments !Monte Carlo design of experiments }
\index{Design of Experiments !LHS design of experiments }
\index{Random Generator}

The objective of this Use Case is to define a random design of experiments  : the design of experiments  does not follow a specified scheme any more. The experiment grid is generated according to a specified distribution and a specified number of points.\\


Details on design of experiments  may be found in the Reference Guide (\extref{ReferenceGuide}{see files Reference Guide - Step C -- Min-Max approach using Designs Of Experiment}{stepC}).\\


OpenTURNS proposes many sampling methods to generate the experiment grid, two of them will be detailed here:
\begin{itemize}
\item the Monte Carlo method: the numerical sample is generated by sampling the specified distribution. When recalled, the {\itshape generate} method regenerates a new numerical sample.
\item the LHS method: the numerical sample is generated by sampling the specified distribution with the LHS technique:  some cells are determined, with the same probabilistic content according to the specified distribution, each line and each column contains exactly one cell, then points are selected among these selected cells. When recalled, the {\itshape generate} method regenerates a new numerical sample: the point selection within the cells changes but not the cells selection. To change the cell selection, it is necessary to create a new LHS Experiment.
\end{itemize}

Before any simulation, it is usefull to initialize the state of the random generator, as defined in the Use Case \ref{randomGenerator}.\\

\requirements{
  \begin{description}
  \item[$\bullet$] the specified distribution: {\itshape distribution}
  \item[type:] Distribution
  \item[$\bullet$] the number of points of the design of experiments : {\itshape number}
  \item[type:] UnsignedLong
  \end{description}
}
             {
               \begin{description}
               \item[$\bullet$] the sample generated: {\itshape experimentSample}
               \item[type:] NumericalSample
               \end{description}
             }

             \textspace\\
             Python script for this UseCase:

             \begin{lstlisting}
               # Create a Monte Carlo design of experiments
               myRandomExp = MonteCarloExperiment(distribution, number)

               # Create a LHS design of experiments
               myRandomExp = LHSExperiment(distribution, number)

               # Generate the design of experiments  numerical sample
               experimentSample = myRandomExp.generate()
             \end{lstlisting}
