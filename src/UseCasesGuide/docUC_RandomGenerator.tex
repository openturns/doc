% Copyright 2005-2016 Airbus-EDF-IMACS-Phimeca
% Permission is granted to copy, distribute and/or modify this document
% under the terms of the GNU Free Documentation License, Version 1.2
% or any later version published by the Free Software Foundation;
% with no Invariant Sections, no Front-Cover Texts, and no Back-Cover
% Texts.  A copy of the license is included in the section entitled "GNU
% Free Documentation License".
\renewcommand{\filename}{docUC_RandomGenerator.tex}
\renewcommand{\filetitle}{UC : Parametrisation of the Random Generator}

% \HeaderNNIILevel
\HeaderIILevel
% \HeaderIIILevel

\label{randomGenerator}


\index{Random Generator}

The seed of the random generator is automatically initialized to 0. It means that as soon as a the openturns session is launched, the sequence of random values generated within $[0,1]$ is the same one : if a script is launched several times, within different openturns sessions, the same results will be obtained. \\


Details on the random generator may be found in the Reference Guide (\extref{ReferenceGuide}{see files Reference Guide - Step B -- Uniform Random Generator}{stepB}).\\


Before any simulation, it is possible to initialise differently than the value by default or get the state of the random generator. \\
To initialize the random generator state, it is possible :
\begin{itemize}
\item to use an easy procedure thanks to the method {\itshape SetSeed()}  parameterized with an integer in $[0, 2^{32}-1]$ :
  \begin{itemize}
  \item to obtain a reproductible sequence of generated random values, we need to explicitely give a deterministic integer,
  \item to obtain a non reproductible sequence of generated random values (it means a new one each time the openturns session is launched), we can give a random integer, determined thanks to the time of the day or the number of the current python session.
  \end{itemize}
\item to specify a complete state of the random generator, usually previously obtained thanks to the {\itshape GetState()} method.
\end{itemize}


\begin{lstlisting}
  # INITIALIZE THE RANDOM GENERATOR STATE

  # Case 1 : reproductible sequence of generated random vector
  # the seed is reproductible

  # Initialise the state of the random generator
  # thanks to the fonctionality SetSeed(n) where n is an UnsignedLong in [0, 2^(32)-1]
  # which enables an easy initialisation for the user
  RandomGenerator.SetSeed(77)

  # or by specifying a complete state of the random generator : particularState
  # coming from a previous particularState = RandomGenerator.GetState() :
  RandomGenerator.SetState(particularState)

  # Case 2 : non reproductible sequence of generated random vector
  # the seed is not reproductible

  # Example 1 : the number of the openturns python session
  from os import getpid
  RandomGenerator.SetSeed(getpid())

  # Example 2 : times of the moment
  from os import times
  RandomGenerator.SetSeed(int(100*times()[4]))


  # GET THE RANDOM GENERATOR STATE

  # Get the complete state of the random generator before simulation
  randomGeneratorStateBeforeRandomExperiment = RandomGenerator.GetState()
\end{lstlisting}
