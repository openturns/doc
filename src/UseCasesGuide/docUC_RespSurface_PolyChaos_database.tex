% Copyright 2005-2016 Airbus-EDF-IMACS-Phimeca
% Permission is granted to copy, distribute and/or modify this document
% under the terms of the GNU Free Documentation License, Version 1.2
% or any later version published by the Free Software Foundation;
% with no Invariant Sections, no Front-Cover Texts, and no Back-Cover
% Texts.  A copy of the license is included in the section entitled "GNU
% Free Documentation License".
\renewcommand{\filename}{docUC_RespSurface_Polynomial_database.tex}
\renewcommand{\filetitle}{UC : Polynomial chaos approximation from a design experiment}

% \HeaderNNIILevel
% \HeaderIILevel
\HeaderIIILevel

\label{krigingApprox}

\index{Response Surface!Polynomial chaos}

This Use Case details the method to construct a response surface from a design experiment by polynomial chaos.\\
You will need the distribution of the input parameters. If not known, statistical inference can be used to select a possible candidate,
and fitting tests can validate such an hypothesis.

\requirements{
  \begin{description}
  \item[$\bullet$] a sample of the input vector: {\itshape X}
  \item[type:] NumericalSample
  \item[$\bullet$] a sample of the output vector: {\itshape Y}
  \item[type:] NumericalSample
  \item[$\bullet$] the distribution of input parameters: {\itshape distribution}
  \item[type:] Distribution
  \end{description}
}
{
  \begin{description}
  \item[$\bullet$] the polynomial chaos algorithm: {\itshape algo}
  \item[type:] a FunctionalChaosAlgorithm
  \item[$\bullet$] the meta-model function: {\itshape metamodel}
  \item[type:] a NumericalMathFunction
  \end{description}
}

\textspace\\
Python script for this Use Case :

\inputscript{script_docUC_RespSurface_PolyChaos_database}

