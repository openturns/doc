% Copyright (C) 2005-2015 Airbus - EDF - IMACS - Phimeca
% Permission is granted to copy, distribute and/or modify this document
% under the terms of the GNU Free Documentation License, Version 1.2
% or any later version published by the Free Software Foundation;
% with no Invariant Sections, no Front-Cover Texts, and no Back-Cover
% Texts.  A copy of the license is included in the section entitled "GNU
% Free Documentation License".
\renewcommand{\filename}{docUC_StocProc_SecondOrderModel_Param.tex}
\renewcommand{\filetitle}{UC : Creation of stationary parametric second order model}

% \HeaderNNIILevel
%\HeaderIILevel
\HeaderIIILevel

\label{SecondOrderModel}

\index{Stochastic Process!Second Order Model}

This use case details how to  create a stationary second order model that insures the coherence between the covariance function $C^{stat}:  \cD\times \cD \rightarrow  \mathcal{M}_{d \times d}(\Rset)$ and the spectral density function   $S : \Rset \rightarrow \mathcal{H}^+(d)$.
We only treat here the case where the domain is of dimension 1: $\cD \in \Rset$ ($n=1$). \\
If the process is continuous, then $\cD=\Rset$. In the discrete case, $\cD$  is a lattice. \\

The coherence is done through the relation (\ref{specdensFunc}) and in some cases, it is not possible:  for example, the spectral model is defined but the associated covariance model is not analytical.\\


OpenTURNS saves the complete information of a second order model in the object {\itshape SecondOrderModel}. \\

A second order model can be used to create zero-mean stationary normal processes, stored either in a {\itshape TemporalNormalProcess} object or in a {\itshape SpectralNormalProcess} one (see the use case of section \ref{StationaryNormalProcessCreation}).\\

OpenTURNS implements the parametric second order model {\itshape ExponentialCauchy} where the covariance function is the Exponential model (see the use case \ref{ParamStationaryCovarianceFunction}) and the associated spectral density function is the Cauchy model (see the use case \ref{DensitySpectralFunctionParam}) .\\


\requirements{
  \begin{description}
  \item[$\bullet$]  $\vect{a}$, $\vect{\lambda}$   : {\itshape amplitude, scale}
  \item[type:]  NumericalPoint
  \end{description}

  \begin{description}
  \item[$\bullet$]  $\mat{R}$  : {\itshape spatialCorrelation}
  \item[type:]  CorrelationMatrix
  \end{description}

  \begin{description}
  \item[$\bullet$]  $\mat{C}$  : {\itshape spatialCovariance}
  \item[type:]  CovarianceMatrix
  \end{description}

}
{

  \begin{description}
  \item[$\bullet$] a second order model : {\itshape mySecondOrderModel}
  \item[type:] SecondOrderModel
  \end{description}
}

\textspace\\
Python script for this UseCase :

\begin{lstlisting}
  # Create the second order model
  # for example : the Exponential Cauchy

  # from the amplitude, scale and spatialCovariance
  mySecondOrderModel = ExponentialCauchy(amplitude, scale, spatialCorrelation)

  # or from the scale and spatialCovariance
  mySecondOrderModel = ExponentialCauchy(scale,spatialCovariance)
\end{lstlisting}
