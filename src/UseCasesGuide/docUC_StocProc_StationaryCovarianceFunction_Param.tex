% Copyright (C) 2005-2015 Airbus - EDF - IMACS - Phimeca
% Permission is granted to copy, distribute and/or modify this document
% under the terms of the GNU Free Documentation License, Version 1.2
% or any later version published by the Free Software Foundation;
% with no Invariant Sections, no Front-Cover Texts, and no Back-Cover
% Texts.  A copy of the license is included in the section entitled "GNU
% Free Documentation License".
\renewcommand{\filename}{docUC_StocProc_StationaryCovarianceFunction_Param.tex}
\renewcommand{\filetitle}{UC : Creation of a  parametric stationary covariance function}

% \HeaderNNIILevel
% \HeaderIILevel
\HeaderIIILevel

\label{ParamStationaryCovarianceFunction}

\index{Stochastic Process!Covariance Function}

Let $X: \Omega \times \cD \rightarrow \Rset^d$  be a multivariate  stationary normal process where $\cD \in \Rset^n$. The process is supposed to be zero mean. It is entirely defined  by  its covariance function $C^{stat}:  \cD \rightarrow  \mathcal{M}_{d \times d}(\Rset)$, defined by $C^{stat}(\vect{\tau})=\Expect{X_{\vect{s}}X_{\vect{s}+\vect{\tau}}^t}$ for all $\vect{s}\in \Rset^n$.\\
If the process is continuous, then $\cD=\Rset^n$. In the discrete case, $\cD$  is a lattice. \\

This use case illustrates how the User can create a covariance function from parametric models. OpenTURNS implements the \emph{multivariate Exponential model} as a parametric model for the covariance function $C^{stat}$. \\

{\bf The multivariate exponential model}: This model defines the  covariance function $C^{stat}$ by:
\begin{equation}
  \label{fullMultivariateExponential2}
  \forall \vect{\tau} \in \cD,\quad C^{stat}( \vect{\tau} )=\left[\mat{A}\mat{\Delta}( \vect{\tau} ) \right] \,\mat{R}\, \left[ \mat{\Delta}( \vect{\tau} )\mat{A}\right]
\end{equation}
where $\mat{R} \in  \mathcal{M}_{d \times d}([-1, 1])$ is a correlation matrix, $\mat{\Delta}( \vect{\tau} ) \in \mathcal{M}_{d \times d}(\Rset)$ is defined by:
\begin{equation}
  \label{fullMultivariateExponential3}
  \mat{\Delta}( \vect{\tau} )= \mbox{Diag}(e^{-\lambda_1|\tau|/2}, \dots, e^{-\lambda_d|\tau|/2})
\end{equation}
and $\mat{A}\in \mathcal{M}_{d \times d}(\Rset)$ is defined by:
\begin{equation}
  \label{fullMultivariateExponential4}
  \mat{A}= \mbox{Diag}(a_1, \dots, a_d)
\end{equation}
whith $\lambda_i>0$ and $a_i>0$ for any $i$.\\
We call $\vect{a}$ the amplitude vector and $\vect{\lambda}$ the scale vector.\\

The expression of $C^{stat}$ is the combination of:
\begin{itemize}
\item the matrix $\mat{R}$ that models the spatial correlation between the components of the process $X$ at any vertex $\vect{t}$ (since the process is stationary):
  \begin{equation}
    \label{fullMultivariateExponential1}
    \forall \vect{t}\in \cD,\quad \mat{R} = \Cor{X_{\vect{t}}, X_{\vect{t}}}
  \end{equation}

\item the matrix $\mat{\Delta}( \vect{\tau} )$ that models the correlation between the marginal random variables  $X^i_{\vect{t}}$ and  $X^i_{\vect{t}+\vect{\tau}}$:
  \begin{align}
    \Cor{X_{\vect{t}}^i,X^i_{\vect{t}+\vect{\tau}}} = e^{-\lambda_i|\tau|}
  \end{align}

\item the matrix $\mat{A}$ that models the variance of each marginal random variable:
  \begin{align}
    \Var{X_{\vect{t}}} = (a_1, \dots, a_d)
  \end{align}
\end{itemize}

This model is such that:
\begin{align}
  C_{ij}^{stat}(\vect{\tau})&  = a_ie^{-\lambda_i|\tau|/2}R_{i,j}a_je^{-\lambda_j|\tau|/2},\quad i\neq j\\
  C_{ii}^{stat}(\vect{\tau}) & =a_ie^{-\lambda_i|\tau|/2}R_{i,i}a_ie^{-\lambda_j|\tau|/2}=a_i^2e^{-\lambda_i|\tau|} \label{diago}
\end{align}

It is possible to define the exponential model from the spatial covariance matrix $\mat{C}^{spat}$ rather than the correlation matrix $\mat{R}$ :
\begin{equation}\label{relRA}
  \forall \vect{t} \in \cD,\quad \mat{C}^{spat} = \Expect{X_{\vect{t}}X^t_{\vect{t}}} = \mat{A}\,\mat{R}\, \mat{A}
\end{equation}


OpenTURNS implements the multivariate exponential model thanks to the object {\itshape ExponentialModel} wich is created from :
\begin{itemize}
\item the scale and amplitude vectors $(\vect{\lambda}, \vect{a})$: in that case, by default $\mat{R} = \mat{I}$;
\item the scale and amplitude vectors and the spatial correlation matrix  $(\vect{\lambda}, \vect{a},\mat{R})$;
\item the scale and amplitude vectors and the spatial covariance matrix  $(\vect{\lambda}, \vect{a},\mat{C})$; Then  $\mat{C}$ is mapped into the associated correlation matrix $\mat{R}$ according to (\ref{relRA}) and  the previous constructor is used.
\end{itemize}






\requirements{
  \begin{description}
  \item[$\bullet$]  $\vect{a}$, $\vect{\lambda}$   : {\itshape amplitude, scale}
  \item[type:]  NumericalPoint
  \end{description}

  \begin{description}
  \item[$\bullet$]  $\mat{R}$  : {\itshape spatialCorrelation}
  \item[type:]  CorrelationMatrix
  \end{description}

  \begin{description}
  \item[$\bullet$]  $\mat{C}$  : {\itshape spatialCovariance}
  \item[type:]  CovarianceMatrix
  \end{description}

}
{
  \begin{description}
  \item[$\bullet$] a covariance model : {\itshape myCovarianceModel }
  \item[type:] StationaryCovarianceModel
  \end{description}

}

\textspace\\
Python script for this UseCase :

\inputscript{script_docUC_StocProc_StationaryCovarianceFunction_Param}
