% Copyright (C) 2005-2015 Airbus - EDF - IMACS - Phimeca
% Permission is granted to copy, distribute and/or modify this document
% under the terms of the GNU Free Documentation License, Version 1.2
% or any later version published by the Free Software Foundation;
% with no Invariant Sections, no Front-Cover Texts, and no Back-Cover
% Texts.  A copy of the license is included in the section entitled "GNU
% Free Documentation License".

% =================================================================

\newpage
\index{DistFunc}
\extanchor{DistFunc}
\subsection{Low level distribution functions}
OpenTURNS implements the following functions. As a general rule, the
prefix $d$ is used for the evaluation of the density probability
function, the prefix $q$ for the quantile function, $p$ for the
evaluation of the cumulative density function and $r$ for the
random number generation.

\begin{itemize}
\item \textit{dNonCentralChiSquare}: see Norman L. Johnson, Samuel Kotz, N. Balakrishnan, "Continuous univariate distributions volume 2", second edition, 1995, Wiley Inter-Science
\item \textit{dNonCentralStudent}: see Norman L. Johnson, Samuel Kotz, N. Balakrishnan, "Continuous univariate distributions volume 2", second edition, 1995, Wiley Inter-Science
\item \textit{eZ1}: Computes the expectation of the $\min$ of $n$ independent standard normal random variables. Usefull for the modified moment estimator of the LogNormal distribution.
\item \textit{pBeta}: uses the special function \textit{BetaRatioInc}
\item \textit{pDickeyFullerConstant}
\item \textit{pDickeyFullerNoConstant}
\item \textit{pDickeyFullerTrend}
\item \textit{pGamma}: normalized Gamma distribution i.e. with a PDF equals
to $\frac{1}{ \Gamma(k)} x ^ {k - 1} \exp(-x)\quad  \forall (x > 0)$.
\item \textit{pKolmogorov}: The algorithms and the selection strategy is described in:
Simard, R. and L'Ecuyer, P. "Computing the Two-Sided Kolmogorov-Smirnov Distribution", Journal of Statistical Software, 2010.
The implementation is from the first author, initially published under the GPL v3 license but used here with written permission of the author.
\item \textit{pNonCentralChiSquare}: See  Denise Benton, K. Krishnamoorthy,
"Computing discrete mixtures of continuous distributions:
noncentral  chisquare, noncentral t and the distribution of the
square of the sample multiple correlation coefficient", Computational Statistics \& Data Analysis, 43 (2003) pp
249-267.
\item \textit{pNonCentralStudent}:  see Denise Benton, K. Krishnamoorthy, "Computing discrete mixtures of continuous distributions: noncentral chisquare, noncentral t
and the distribution of the square of the sample multiple correlation coefficient",
Computational Statistics \& Data Analysis, 43 (2003) pp 249-267
\item \textit{pNonCentralStudentAlt}
\item \textit{pNormal}
\item \textit{pNormal2D}
\item \textit{pNormal3D}
\item \textit{pStudent}
\item \textit{qBeta}: see the algorithm of Cheng (1978), Johnk, Atkinson and Whittaker (1979) 1 \& 2 described in:   Luc Devroye, "Non-Uniform RandomVariate Generation", Springer-Verlag, 1986, available online at:   http://cg.scs.carleton.ca/~luc/nonuniformrandomvariates.zip   and with the important errata at:   http://cg.scs.carleton.ca/~luc/errors.pdf.
\item \textit{qDickeyFullerConstant}
\item \textit{qDickeyFullerNoConstant}
\item \textit{qDickeyFullerTrend}
\item \textit{qGamma}
\item \textit{qNormal}
\item \textit{qStudent}
\item \textit{rBeta}: the strategy is:
\begin{itemize}
\item If ($p_1 = 1$ and $p_2 = 1$), rBeta(1,1) = Uniform(0,1).getRealization()
\item If ($p_1 = 1$ or $p_2 = 1$), analytic cases
\item If( $p_1 + p_2 \leq 1$), Johnk
\item If ($p_1 + p_2 > 1$): If ($p_1 < 1$ and $p_2 < 1$), Atkinson and Whittaker
1; If ($p_1 < 1$ and $p_2 > 1$) or ($p_1 > 1$ and $p_2 < 1$), Atkinson and
Whittaker 2; If ($p_1 > 1$ and $p_2 > 1$), Cheng.
\end{itemize}
\item \textit{rBinomial}:  see the rejection algorithm described in:
Wolfgang Hormann, "The Generation of Binomial Random Variates",
Journal of Statistical Computation and Simulation 46, pp. 101-110, 1993
http://epub.wu.ac.at/1242/.
\item \textit{rGamma}: see the algorithm described in:
George Marsaglia and Wai Wan Tsang, "A Simple Method for Generating Gamma
Variables": ACM Transactions on Mathematical Software, Vol. 26, No. 3,
September 2000, Pages 363-372
with a small optimization on the beta that appears in the squeezing function $(1 + \beta x^4)exp(-x^2/2)$.
We also add the special treatment of the case $k < 1$.
\item \textit{rNonCentralChiSquare}: We use the following transformation method: if $J$ is distributed
according to $Poisson(\lambda/2)$ and $Y$ knowing $J$ is distributed according
to $\chi^2_{k+2J}$, then $Y$ is distributed according to $rNonCentralChiSquare(k,\lambda)$.
\item \textit{rNonCentralStudent}: We use a transformation method based on Gamma and Normal transformation:
If $N$ is $Normal(\delta, 1)$ distributed and $G$ is $\Gamma(\nu / 2)$ distributed,   $\frac{\sqrt{2 \nu} N }{ \sqrt{G}}$ is distributed according to $NonCentralStudent(\nu, \delta)$.
\item \textit{rNormal}:  We use the improved ziggurat method, see:
Doornik, J.A. (2005), "An Improved Ziggurat Method to Generate Normal
Random Samples", mimeo, Nuffield College, University of Oxford,
and www.doornik.com/research.
\item \textit{rPoisson}: For the small values of lambda, we use the method of inversion by sequential search described in:
Luc Devroye, "Non-Uniform RandomVariate Generation", Springer-Verlag, 1986, available online at:
http://cg.scs.carleton.ca/~luc/nonuniformrandomvariates.zip
and with the important errata at:
http://cg.scs.carleton.ca/~luc/errors.pdf
For the large values of lambda, we use the ratio of uniform method described in:
E. Stadlober, "The ratio of uniforms approach for generating discrete random variates". Journal of Computational and Applied Mathematics, vol. 31, no. 1, 1990, pp. 181-189.
\item \textit{rStudent}: We use a transformation method based on Gamma and Normal transformation:
If $N$ is $Normal(0, 1)$ distributed and $G$ is $\Gamma(\nu / 2)$ distributed,
$\frac{\sqrt{2 \nu} N }{ \sqrt{G}}$ is distributed according to $Student(\nu)$.
\end{itemize}
