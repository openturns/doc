% Copyright (C) 2005-2015 Airbus - EDF - IMACS - Phimeca
% Permission is granted to copy, distribute and/or modify this document
% under the terms of the GNU Free Documentation License, Version 1.2
% or any later version published by the Free Software Foundation;
% with no Invariant Sections, no Front-Cover Texts, and no Back-Cover
% Texts.  A copy of the license is included in the section entitled "GNU
% Free Documentation License".


\newpage
\index{FFT}
\section{FFT}

The objective of this class is to implement the Fourier transformations (direct and inverse) using external tools such as KissFFT. \\

As there are many ways to define the Discret Fourier Transform (DFT) and its inverse, varying for example the sign of the exponential term, the normalization factor...etc.
we fix here the convention for direct and inverse transformations. Thus, for the DFT (see \textit{transform} method of the class), the convention used is : \\
\begin{align*}
Z_{k} = \sum_{n=0}^{N-1} X_{n} \exp\left\{ -2 \pi \imath \frac{nk}{N} \right\},\ k=0,1,\ldots,N-1
\end{align*} \\

The inverse DFT is defined as :\\
\begin{align*}
Y_{k} = \frac{1}{N} \sum_{n=0}^{N-1} Z_{n} \exp\left\{2 \pi \imath \frac{nk}{N} \right\},\ k=0,1,\ldots,N-1
\end{align*} \\


We describe here the methods of this class. \\

\begin{description}

\item[Usage:] \rule{0pt}{1em}
\begin{description}
\item \textit{FFT()}
\item
\end{description}

\item[Value:] FFT
\begin{description}
\item This instantiates the Fast Fourier Transform (FFT) class
\end{description}

\item[Some methods :]  \rule{0pt}{1em}

\item \textit{transform}
\begin{description}
\item[Usage:] \textit{transform(collection)}
\item[Arguments:] NumericalComplexCollection
\item[Value:] a NumericalComplexCollection. This computes the Fourier transformation of the values contained in collection.
\end{description}
\bigskip

\item \textit{transform}
\begin{description}
\item[Usage:] \textit{transform(scalarCollection)}
\item[Arguments:] NumericalScalarCollection
\item[Value:] a NumericalComplexCollection. This computes the Fourier transformation of the values contained in a real collection.
\end{description}
\bigskip

\item \textit{transform}
\begin{description}
\item[Usage:] \textit{transform(collection, first, size)}
\item[Arguments:] NumericalScalarCollection, integer, integer
\item[Value:] a NumericalComplexCollection. This computes the Fourier transformation of some values contained in a real collection (a block of the collection).
This block has for size \textit{size} and the first element of the considerd block is the \textit{collection[first]}
\end{description}
\bigskip

\item \textit{transform}
\begin{description}
\item[Usage:] \textit{transform(collection, first, step, last)}
\item[Arguments:] NumericalScalarCollection, integer, integer, integer
\item[Value:] a NumericalComplexCollection. This computes the Fourier transformation of some values contained in a real collection.
This block is composed of element \textit{collection[first]}, \textit{collection[first + step]}, \textit{collection[first + 2 * step]},\textit{\ldots}, \textit{collection[last]}
\end{description}
\bigskip

\item \textit{inverseTransform}
\begin{description}
\item[Usage:] \textit{inverseTransform(collection)}
\item[Arguments:] NumericalComplexCollection
\item[Value:] a NumericalComplexCollection. This computes the Fourier inverse transformation of the values contained in collection.
\end{description}
\bigskip

\item \textit{inverseTransform}
\begin{description}
\item[Usage:] \textit{inverseTransform(collection, first, size)}
\item[Arguments:] NumericalScalarCollection, integer, integer
\item[Value:] a NumericalComplexCollection. This computes the Fourier inverse transformation of some values contained in a real collection (a block of the collection).
This block has for size \textit{size} and the first element of the considerd block is the \textit{collection[first]}
\end{description}
\bigskip

\item \textit{inverseTransform}
\begin{description}
\item[Usage:] \textit{inverseTransform(collection, first, step, last)}
\item[Arguments:] NumericalScalarCollection, integer, integer, integer
\item[Value:] a NumericalComplexCollection. This computes the inverse Fourier transformation of some values contained in a real collection.
This block is composed of element \textit{collection[first]}, \textit{collection[first + step]}, \textit{collection[first + 2 * step]}..., \textit{collection[last]}
\end{description}
\bigskip

\begin{description}
\item \textit{getName}
\begin{description}
\item[Usage:] \textit{getName()}
\item[Arguments:] none
\item[Value:] a string, the name of the FFT.
\end{description}
\bigskip

\item \textit{setName}
\begin{description}
\item[Usage:] \textit{setName(name)}
\item[Arguments:] name : a string
\item[Value:] the FFT is named \textit{name}
\end{description}
\bigskip

\end{description}
\end{description}



% ==========================================================================

\newpage
%\index{FFT!KissFFT}
\index{KissFFT}
\subsubsection{KissFFT}
The KissFFT class inherits from the FFT class. The methods are the same as the FFT class (there is no additional method).
This class interacts with the \emph{kissfft} implemented and return results as OpenTurns objects (NumericalComplexCollection).
