% Copyright (C) 2005-2015 Airbus - EDF - IMACS - Phimeca
% Permission is granted to copy, distribute and/or modify this document
% under the terms of the GNU Free Documentation License, Version 1.2
% or any later version published by the Free Software Foundation;
% with no Invariant Sections, no Front-Cover Texts, and no Back-Cover
% Texts.  A copy of the license is included in the section entitled "GNU
% Free Documentation License".


\newpage

\section{Graphs}


% \index{Graphs!Graph}
\index{Graph}
\extanchor{graph}
\subsection{Graph}

The class Graph is the structure which contains:
\begin{itemize}
\item the drawable elements (may be several drawables elements): class Drawable
\item the graphical context: the potential axes and labels, the bounding box, the global title, the global legend and its position
\end{itemize}


\begin{description}
\item[Usage:] \rule{0pt}{1em}
  \begin{description}
  \item \textit{Graph(title, xTitle, yTitle, showAxes)}
  \item \textit{Graph(title, xTitle, yTitle, showAxes, legendPosition)}
  \item \textit{Graph(title, xTitle, yTitle, showAxes, legendPosition, legendFontSize, logScale)}
  \end{description}

\item[Arguments:]\rule{0pt}{1em}
  \begin{description}
  \item \textit{title}: a String, the title of the graph
  \item \textit{xTitle}: a String, the legend of the X axe
  \item \textit{yTitle}: a String, the legend of the Y axe
  \item \textit{showAxes}: a boolean which indicates if the axes are drawn (yes = 1, no = 0)
  \item \textit{legendPosition}: a String which indicates the position of the legend. If \textit{legendPosition} is not specified, the Graph has no legend.
  \item \textit{legendFontSize}: an interger, the font size of the legend. If not specified, the default width wil be used.
  \item \textit{logScale}: a LogScale that indicates whether the logarithmic scale is used aither for one or both axes:
    \begin{itemize}
    \item \textit{GraphImplementation.NONE}: no log scale is used,
    \item \textit{GraphImplementation.LOGX}: log scale is used only for horizontal data,
    \item \textit{GraphImplementation.LOGY}: log scale is used only for vertical data,
    \item \textit{GraphImplementation.LOGXY}: log scale is used  for both data.
    \end{itemize}
  \end{description}

\item[Some methods:]  \rule{0pt}{1em}
  \begin{description}

  \item \textit{add}
    \begin{description}
    \item[Usage:] \rule{0pt}{1em}
      \begin{description}
      \item \textit{add(aDrawable)}
      \item \textit{add(aGraph)}
      \end{description}
    \item[Arguments:]  \rule{0pt}{1em}
      \begin{description}
      \item \textit{aDrawable}: a Drawable, a drawable element we want to add on the graph
      \item \textit{aGraph}: a Graph
      \end{description}
    \item[Value:] none, it adds the new drawable or graph inside the first one, with its legend. It keeps the graphical context of the first graph. Each drawable keeps its graphical context. care: different drawables might be colored the same....
    \end{description}
    \bigskip

  \item \textit{addDrawables} \rule{0pt}{1em}
    \begin{description}
    \item[Usage:] \textit{addDrawable(aDrawableCollection)}
    \item[Arguments:] \textit{aDrawableCollection}: a DrawableCollection, a collection of drawables we want to add on the graph.
    \item[Value:] none, it adds the drawable collection to the graph. It keeps the graphical context of the first graph.
    \end{description}
    \bigskip

  \item \textit{draw}
    \begin{description}
    \item[Usage:] \rule{0pt}{1em}
      \begin{description}
      \item \textit{draw(path, file, width, height)}
      \item \textit{draw(path, file, width, height, format)}
      \item \textit{draw(file, width, height)}
      \item \textit{draw(file, width, height, format)}
      \item \textit{draw(file)}
      \end{description}
    \item[Arguments:]\rule{0pt}{1em}
      \begin{description}
      \item  \textit{path}: a String which indicates the adress where the created file will be put. When not specified, the files is created in the current repertory.
      \item  \textit{file}: a String which indicates the name of the created file (without the suffixe). The files created will be file.png and file.ps.
      \item  \textit{width}, \textit{height}: two real positive values, number of pixels fixing the width and the height of the graph. When not specified, the couple (640,480) is taken into account.
      \item  \textit{format = GraphImplementation.EPS,  GraphImplementation.PNG, GraphImplementation.FIG or  GraphImplementation.PDF}. When not precised, by default, \textit{format = ALL}.
      \end{description}
    \item[Value:] It generates the files \textit{file.png}, \textit{file.eps}, \textit{file.fig} and \textit{file.pdf} or only the format specified in \textit{format}.
    \end{description}
    \bigskip

  \item \textit{getAxes}
    \begin{description}
    \item[Usage:] \textit{getAxes()}
    \item[Arguments:] none
    \item[Value:] a boolean which indicates if the axes are drawn (yes = 1, no = 0)
    \end{description}
    \bigskip

  \item \textit{getBoundingBox}
    \begin{description}
    \item[Usage:] \textit{getBoundingBox()}
    \item[Arguments:] none
    \item[Value:] a NumericalPoint of dimension 4, the bounding box of the drawable element, which is a rectangle determined by its range along X and its range along Y. The BoundingBox is  $(x_{min}, x_{max}, y_{min}, y_{max})$.
    \end{description}
    \bigskip

  \item \textit{getColors}
    \begin{description}
    \item[Usage:] \textit{getColors()}
    \item[Arguments:] none
    \item[Value:] a Description: the list of all the colors used for the drawables contained inside the graph.
    \end{description}
    \bigskip

  \item \textit{setColors}
    \begin{description}
    \item[Usage:] \textit{setColors(listColors)}
    \item[Arguments:] \textit{listColors}: a Description: the list of the colors used for each Drawable of the Graph. If \textit{listColors.getSize()} is lower than the number of Drawables, the first colors of \textit{listColors} are re-used. If \textit{listColors.getSize()} is greated than the number of Drawables, the last colors of the list are ignored.\\
      The \textit{listColors} argument can be the result of the static method \textit{BuildDefaultPalette(n)} of the object Drawable where $n$ is the number of colors needed.
    \item[Value:] none. The colors of the Drawables inside the Graph are updated.
    \end{description}
    \bigskip

  \item \textit{setDefaultColors}
    \begin{description}
    \item[Usage:] \textit{setDefaultColors()}
    \item[Arguments:] none
    \item[Value:] none. The method automatically assigns colors according to a default palette to al the drawables of the graph. This method ensures that drawables of the graph have different colors.
    \end{description}
    \bigskip

  \item \textit{getDrawables}
    \begin{description}
    \item[Usage:] \textit{getDrawables()}
    \item[Arguments:] none
    \item[Value:] a DrawableCollection, the collection of the Drawables included in the graph
    \end{description}
    \bigskip

  \item \textit{getLegends}
    \begin{description}
    \item[Usage:] \textit{getLegends()}
    \item[Arguments:] none
    \item[Value:] a Description: the list of all the legends used for the drawables contained inside the graph.
    \end{description}
    \bigskip

  \item \textit{setLegends}
    \begin{description}
    \item[Usage:] \textit{setLegends(listLegends)}
    \item[Arguments:]\textit{listLegends}: a Description, the list of all the legends to be assigned to the drawables contained inside the graph.
    \item[Value:] none. The legends of the Drawables are updated.
    \end{description}
    \bigskip

  \item \textit{getGridColor}
    \begin{description}
    \item[Usage:] \textit{getGridColor()}
    \item[Arguments:] none
    \item[Value:] a String, the color of the grid.
    \end{description}
    \bigskip


  \item \textit{getFileName}
    \begin{description}
    \item[Usage:] \textit{getFileName()}
    \item[Arguments:] none
    \item[Value:] a String, the name of the files containing the graph
    \end{description}
    \bigskip

  \item \textit{getLegendFontSize}
    \begin{description}
    \item[Usage:] \textit{getLegendFontSize()}
    \item[Arguments:] none
    \item[Value:] a positive real, the legend font size
    \end{description}
    \bigskip

  \item \textit{getLegendPosition}
    \begin{description}
    \item[Usage:] \textit{getLegendPosition()}
    \item[Arguments:] none
    \item[Value:] a String, the position of the legend.
    \end{description}
    \bigskip

  \item \textit{getPath}
    \begin{description}
    \item[Usage:] \textit{getPath()}
    \item[Arguments:] none
    \item[Value:] a String, the file path excluding the suffix
    \end{description}
    \bigskip

  \item \textit{getBitmap}
    \begin{description}
    \item[Usage:] \textit{getBitmap()}
    \item[Arguments:] none
    \item[Value:] a String, the adress of the file file.png created by the method .draw()
    \end{description}
    \bigskip

  \item \textit{getPostscript}
    \begin{description}
    \item[Usage:] \textit{getPostscript()}
    \item[Arguments:] none
    \item[Value:] a String, the adress of the file file.ps created by the method .draw()
    \end{description}
    \bigskip

  \item \textit{getPDF}
    \begin{description}
    \item[Usage:] \textit{getPDF()}
    \item[Arguments:] none
    \item[Value:] a String, the adress of the file file.pdf created by the method .draw()
    \end{description}
    \bigskip

  \item \textit{getVectorial}
    \begin{description}
    \item[Usage:] \textit{getVectorial()}
    \item[Arguments:] none
    \item[Value:] a String, the adress of the file file.fig created by the method .draw()
    \end{description}
    \bigskip


  \item \textit{getTitle}
    \begin{description}
    \item[Usage:] \textit{getTitle()}
    \item[Arguments:] none
    \item[Value:] a String, the title of the graph
    \end{description}
    \bigskip

  \item \textit{getXTitle}
    \begin{description}
    \item[Usage:] \textit{getXTitle()}
    \item[Arguments:] none
    \item[Value:] a String, the title of the X axe
    \end{description}
    \bigskip

  \item \textit{getYTitle}
    \begin{description}
    \item[Usage:] \textit{getYTitle()}
    \item[Arguments:] none
    \item[Value:] a String, the title of the Y axe
    \end{description}
    \bigskip

  \item \textit{setBoundingBox}
    \begin{description}
    \item[Usage:] \rule{0pt}{1em}
      \begin{description}
      \item \textit{setBoundingBox(myBoundingBox)}
      \item \textit{setBoundingBox(myInterval)}
      \end{description}
    \item[Arguments:]  \rule{0pt}{1em}
      \begin{description}
      \item \textit{myBoundingBox}: a BoundingBox, which is a Numericalpoint(4) composed by $[x_{min}, x_{max}, y_{min}, y_{max}]$ if we want to impose the x-range to $[x_{min}, x_{max}]$ and the y-range to $[y_{min}, y_{max}]$.
      \item \textit{myInterval}: a Interval, which is created as follows myInterval(BottomLeftPoint, UpRightPoint) where BottomLeftPoint is a NumericalPoint(2) representing the corner $(x_{min}, y_{min})$ and UpRightPoint the corner  $(x_{max}, y_{max})$.
      \end{description}
    \item[Value:] none
    \end{description}
  \end{description}

  The methods \textit{getAxes}, \textit{getDrawables},  \textit{getLegendPosition},  \textit{getTitle},  \textit{getXTitle},  \textit{getYTitle}have their corresponding \textit{setMethod}.

\end{description}

Here is the list of legend positions accepted by OpenTURNS: "bottomright", "bottom", "bottomleft", "left", "topleft", "topright", "right", "center".


% -------------------------------------------------
\newpage
% \index{Graphs!Drawable}
\index{Drawable}
\subsection{Drawable}

A Drawable is a  drawable element described by:
\begin{itemize}
\item its data,
\item their attributes: color, line stype, point style, fill style
\item the specific legend of the drawable element.
\end{itemize}


\begin{description}
\item[Usage:] \textit{Drawable(drawableImplementation)}

\item[Arguments:] \textit{drawableImplementation}: a DrawableImplementation, the implementation of  Drawable

\item[Some methods:]  \rule{0pt}{1em}
  \begin{description}


  \item \textit{BuildDefaultPalette}
    \begin{description}
    \item[Usage:] \textit{BuildDefaultPalette(n)}
    \item[Arguments:] $n$: an integer.
    \item[Value:] a Description of size $n$: the list of $n$ color codes defined according to the default OpenTURNS palette.
    \end{description}
    \bigskip

  \item \textit{getBoundingBox}
    \begin{description}
    \item[Usage:] \textit{getBoundingBox()}
    \item[Arguments:] none
    \item[Value:] a NumericalPoint of dimension 4, the bounding bow of the drawable element, which is a rectangle determined by its range along X and its range along Y. The BoundingBox is $(x_{min}, x_{max}, y_{min}, y_{max})$.
    \end{description}
    \bigskip

  \item \textit{getColor}
    \begin{description}
    \item[Usage:] \textit{getColor()}
    \item[Arguments:] none
    \item[Value:] a String which describes the color of the lines within the drawable element. It can be either the name of a color (e.g. "red") or an hexadecimal code corresponding to the RGB (Red, Green, Blue) components of the color (e.g. \#A1B2C3) or the RGBA (Red, Green, Blue, Alpha) components of the color (e.g. \#A1B2C3D4). The alpha channel is taken into account only by the PDF and PNG formats, for the other format the color is fully transparent as soon as its alpha channel is less than 255 (or 1.0). Use \textit{GetValidColors} for a list of available values.
    \end{description}
    \bigskip


  \item \textit{getData}
    \begin{description}
    \item[Usage:] \textit{getData()}
    \item[Arguments:] none
    \item[Value:] a NumericalSample, from which the Drawable is built
    \end{description}
    \bigskip
  \item \textit{getFillStyle}
    \begin{description}
    \item[Usage:] \textit{getFillStyle()}
    \item[Arguments:] none
    \item[Value:] a String which describes the fill style of the surfaces within the drawable element. Use \textit{GetValidFillStyles} for a list of available values.
    \end{description}
    \bigskip
  \item \textit{getLabels}
    \begin{description}
    \item[Usage:] \textit{getLabels()}
    \item[Arguments:] none
    \item[Value:] a Description, the labels of both axes
    \end{description}
    \bigskip
  \item \textit{getLegend}
    \begin{description}
    \item[Usage:] \textit{getLegend()}
    \item[Arguments:] none
    \item[Value:] a String which is the legend of the drawable element
    \end{description}
    \bigskip
  \item \textit{getLineStyle}
    \begin{description}
    \item[Usage:] \textit{getLineStyle()}
    \item[Arguments:] none
    \item[Value:] a String which describes the  style of the lines within the drawable element. Use \textit{GetValidLineStyles} for a list of available values.
    \end{description}
    \bigskip
  \item \textit{getLineWidth}
    \begin{description}
    \item[Usage:] \textit{getLineWidth()}
    \item[Arguments:] none
    \item[Value:] an interger, the width of the line included in the Drawable (if such the case)
    \end{description}
    \bigskip
  \item \textit{getPointCode}
    \begin{description}
    \item[Usage:] \textit{getPointCode()}
    \item[Arguments:] none
    \item[Value:] an integer which describes the style of the points within the drawable element
    \end{description}
    \bigskip

  \item \textit{getPointStyle}
    \begin{description}
    \item[Usage:] \textit{getPointStyle()}
    \item[Arguments:] none
    \item[Value:] a string which describes the style of the points within the drawable element. Use \textit{GetValidPointStyles} for a list of available values.
    \end{description}
    \bigskip

  \item \textit{ConvertFromRGB}
    \begin{description}
    \item[Usage:] \textit{ConvertFromRGB(red, green, blue)}
    \item[Arguments:] \textit{red}, \textit{green} and \textit{blue} are either three nonnegative integer (UnsignedLong) values or three nonnegative real (NumericalScalar) values (which will be scaled and converted to integer values). These values are the red, green and blue components of a color, a value of 0 (or 0.0) meaning that the component is absent in the color, a value of 255 (or 1.0) meaning that the component is fully saturated.
    \item[Value:] a String giving the associated hexadecimal color code.
    \end{description}
    \bigskip
  \item \textit{ConvertFromRGBA}
    \begin{description}
    \item[Usage:] \textit{ConvertFromRGBA(red, green, blue, alpha)}
    \item[Arguments:] \textit{red}, \textit{green}, \textit{blue} and \textit{alpha} are either four nonnegative integer (UnsignedLong) values or four nonnegative real (NumericalScalar) values (which will be scaled and converted to integer values). These values are the red, green and blue components of a color, a value of 0 (or 0.0) meaning that the component is absent in the color, a value of 255 (or 1.0) meaning that the component is fully saturated. The parameter alpha gives the level of transparency of the color, 0 (or 0.0) meaning that the color is fully transparent and 255 (or 1.0) meaning that the color is fully opaque. The alpha channel is only supported by a few devices, namely the PDF and PNG formats, for the other format the color is fully transparent as soon as its alpha channel is less than 255 (or 1.0).
    \item[Value:] a String giving the associated hexadecimal color code.
    \end{description}
    \bigskip
  \item \textit{GetValidColors}
    \begin{description}
    \item[Usage:] \textit{GetValidColors()}
    \item[Arguments:] none
    \item[Value:] a Description which indicates the list of the valid colors of a Drawable
    \end{description}
    \bigskip
  \item \textit{GetValidFillStyles}
    \begin{description}
    \item[Usage:] \textit{GetValidFillStyles()}
    \item[Arguments:] none
    \item[Value:] a Description which indicates the list of the valid fill styles of a Drawable
    \end{description}
    \bigskip
  \item \textit{GetValidLineStyles}
    \begin{description}
    \item[Usage:] \textit{GetValidLineStyles()}
    \item[Arguments:] none
    \item[Value:] a Description which indicates the list of the valid line styles of a Drawable
    \end{description}
    \bigskip
  \item \textit{GetValidPointStyles}
    \begin{description}
    \item[Usage:] \textit{GetValidPointStyles()}
    \item[Arguments:] none
    \item[Value:] a Description which indicates the list of the valid point  styles of a Drawable
    \end{description}
  \end{description}
\end{description}

All the methods \textit{getColor}, \textit{getFillStyle}, \textit{getLineStyle}, \textit{getPointCode} and \textit{getPointStyle} have their corresponding \textit{setMethod}.\\

Here is the list of codes, styles and width accepted by OpenTURNS:
\begin{itemize}
\item map  matching keys with R codes for point symbols:

  \begin{center}
    \begin{tabular}{c|c}
      Point Style & Point Code \\
      \hline
      square & 0 \\
      \hline
      circle & 1\\
      \hline
      triangleup & 2\\
      \hline
      plus & 3\\
      \hline
      times & 4 \\
      \hline
      diamond & 5\\
      \hline
      triangledown & 6\\
      \hline
      star & 8\\
      \hline
      fsquare & 15\\
      \hline
      fcircle & 16\\
      \hline
      ftriangleup & 17\\
      \hline
      fdiamond  & 18\\
      \hline
      bullet & 20 \\
      \hline
      dot & 127
    \end{tabular}
  \end{center}

\item authorized colors:
  \begin{itemize}
  \item All the codes of the form \#RRGGBB or \#RRGGBBAA, where R, G, B, A are hexadecimal digits (0,\dots,9,A or a,..., F or f). Examples: \#03A21F, \#3b2E43ff.
  \item All the names in the list: "green", "red", "blue", "yellow", "darkblue", "orange", "lightgreen", "darkcyan", "cyan", "magenta", "darkgreen", "violet", "brown", "darkred", "pink", "ivory", "gold", "darkgrey", "grey", "white", "aliceblue", "antiquewhite", "antiquewhite1", "antiquewhite2", "antiquewhite3", "antiquewhite4", "aquamarine", "aquamarine1", "aquamarine2", "aquamarine3", "aquamarine4", "azure", "azure1", "azure2", "azure3", "azure4", "beige", "bisque", "bisque1", "bisque2", "bisque3", "bisque4", "black", "blanchedalmond", "blue1", "blue2", "blue3", "blue4", "blueviolet", "brown1", "brown2", "brown3", "brown4", "burlywood", "burlywood1", "burlywood2", "burlywood3", "burlywood4", "cadetblue", "cadetblue1", "cadetblue2", "cadetblue3", "cadetblue4", "chartreuse", "chartreuse1", "chartreuse2", "chartreuse3", "chartreuse4", "chocolate", "chocolate1", "chocolate2", "chocolate3", "chocolate4", "coral", "coral1", "coral2", "coral3", "coral4", "cornflowerblue"
    , "cornsilk", "cornsilk1", "cornsilk2", "cornsilk3", "cornsilk4", "cyan1", "cyan2", "cyan3", "cyan4", "darkgoldenrod", "darkgoldenrod1", "darkgoldenrod2", "darkgoldenrod3", "darkgoldenrod4", "darkgray", "darkkhaki", "darkmagenta", "darkolivegreen", "darkolivegreen1", "darkolivegreen2", "darkolivegreen3", "darkolivegreen4", "darkorange", "darkorange1", "darkorange2", "darkorange3", "darkorange4", "darkorchid", "darkorchid1", "darkorchid2", "darkorchid3", "darkorchid4", "darksalmon", "darkseagreen", "darkseagreen1", "darkseagreen2", "darkseagreen3", "darkseagreen4", "darkslateblue", "darkslategray", "darkslategray1", "darkslategray2", "darkslategray3", "darkslategray4", "darkslategrey", "darkturquoise", "darkviolet", "deeppink", "deeppink1", "deeppink2", "deeppink3", "deeppink4", "deepskyblue", "deepskyblue1", "deepskyblue2", "deepskyblue3", "deepskyblue4", "dimgray", "dimgrey", "dodgerblue", "dodgerblue1", "dodgerblue2", "dodgerblue3", "dodgerblue4"
    , "firebrick", "firebrick1", "firebrick2", "firebrick3", "firebrick4", "floralwhite", "forestgreen", "gainsboro", "ghostwhite", "gold1", "gold2", "gold3", "gold4", "goldenrod", "goldenrod1", "goldenrod2", "goldenrod3", "goldenrod4", "gray", "gray0", "gray1", "gray2", "gray3", "gray4", "gray5", "gray6", "gray7", "gray8", "gray9", "gray10", "gray11", "gray12", "gray13", "gray14", "gray15", "gray16", "gray17", "gray18", "gray19", "gray20", "gray21", "gray22", "gray23", "gray24", "gray25", "gray26", "gray27", "gray28", "gray29", "gray30", "gray31", "gray32", "gray33", "gray34", "gray35", "gray36", "gray37", "gray38", "gray39", "gray40", "gray41", "gray42", "gray43", "gray44", "gray45", "gray46", "gray47", "gray48", "gray49", "gray50", "gray51", "gray52", "gray53", "gray54", "gray55", "gray56", "gray57", "gray58", "gray59", "gray60", "gray61", "gray62", "gray63", "gray64", "gray65", "gray66", "gray67", "gray68", "gray69", "gray70", "gray71", "gray72", "gray73", "gray74", "gray75", "gray76"
    , "gray77", "gray78", "gray79", "gray80", "gray81", "gray82", "gray83", "gray84", "gray85", "gray86", "gray87", "gray88", "gray89", "gray90", "gray91", "gray92", "gray93", "gray94", "gray95", "gray96", "gray97", "gray98", "gray99", "gray100", "green1", "green2", "green3", "green4", "greenyellow", "grey0", "grey1", "grey2", "grey3", "grey4", "grey5", "grey6", "grey7", "grey8", "grey9", "grey10", "grey11", "grey12", "grey13", "grey14", "grey15", "grey16", "grey17", "grey18", "grey19", "grey20", "grey21", "grey22", "grey23", "grey24", "grey25", "grey26", "grey27", "grey28", "grey29", "grey30", "grey31", "grey32", "grey33", "grey34", "grey35", "grey36", "grey37", "grey38", "grey39", "grey40", "grey41", "grey42", "grey43", "grey44", "grey45", "grey46", "grey47", "grey48", "grey49", "grey50", "grey51", "grey52", "grey53", "grey54", "grey55", "grey56", "grey57", "grey58", "grey59", "grey60", "grey61", "grey62", "grey63", "grey64", "grey65", "grey66", "grey67", "grey68", "grey69", "grey70", "grey71", "grey72"
    , "grey73", "grey74", "grey75", "grey76", "grey77", "grey78", "grey79", "grey80", "grey81", "grey82", "grey83", "grey84", "grey85", "grey86", "grey87", "grey88", "grey89", "grey90", "grey91", "grey92", "grey93", "grey94", "grey95", "grey96", "grey97", "grey98", "grey99", "grey100", "honeydew", "honeydew1", "honeydew2", "honeydew3", "honeydew4", "hotpink", "hotpink1", "hotpink2", "hotpink3", "hotpink4", "indianred", "indianred1", "indianred2", "indianred3", "indianred4", "ivory1", "ivory2", "ivory3", "ivory4", "khaki", "khaki1", "khaki2", "khaki3", "khaki4", "lavender", "lavenderblush", "lavenderblush1", "lavenderblush2", "lavenderblush3", "lavenderblush4", "lawngreen", "lemonchiffon", "lemonchiffon1", "lemonchiffon2", "lemonchiffon3", "lemonchiffon4", "lightblue", "lightblue1", "lightblue2", "lightblue3", "lightblue4", "lightcoral", "lightcyan", "lightcyan1", "lightcyan2", "lightcyan3", "lightcyan4"
    , "lightgoldenrod", "lightgoldenrod1", "lightgoldenrod2", "lightgoldenrod3", "lightgoldenrod4", "lightgoldenrodyellow", "lightgray", "lightgrey", "lightpink", "lightpink1", "lightpink2", "lightpink3", "lightpink4", "lightsalmon", "lightsalmon1", "lightsalmon2", "lightsalmon3", "lightsalmon4", "lightseagreen", "lightskyblue", "lightskyblue1", "lightskyblue2", "lightskyblue3", "lightskyblue4", "lightslateblue", "lightslategray", "lightslategrey", "lightsteelblue", "lightsteelblue1", "lightsteelblue2", "lightsteelblue3", "lightsteelblue4", "lightyellow", "lightyellow1", "lightyellow2", "lightyellow3", "lightyellow4", "limegreen", "linen", "magenta1", "magenta2", "magenta3", "magenta4", "maroon", "maroon1", "maroon2", "maroon3", "maroon4", "mediumaquamarine", "mediumblue", "mediumorchid", "mediumorchid1", "mediumorchid2", "mediumorchid3", "mediumorchid4", "mediumpurple", "mediumpurple1", "mediumpurple2", "mediumpurple3", "mediumpurple4", "mediumseagreen", "mediumslateblue", "mediumspringgreen"
    , "mediumturquoise", "mediumvioletred", "midnightblue", "mintcream", "mistyrose", "mistyrose1", "mistyrose2", "mistyrose3", "mistyrose4", "moccasin", "navajowhite", "navajowhite1", "navajowhite2", "navajowhite3", "navajowhite4", "navy", "navyblue", "oldlace", "olivedrab", "olivedrab1", "olivedrab2", "olivedrab3", "olivedrab4", "orange1", "orange2", "orange3", "orange4", "orangered", "orangered1", "orangered2", "orangered3", "orangered4", "orchid", "orchid1", "orchid2", "orchid3", "orchid4", "palegoldenrod", "palegreen", "palegreen1", "palegreen2", "palegreen3", "palegreen4", "paleturquoise", "paleturquoise1", "paleturquoise2", "paleturquoise3", "paleturquoise4", "palevioletred", "palevioletred1", "palevioletred2", "palevioletred3", "palevioletred4", "papayawhip", "peachpuff", "peachpuff1", "peachpuff2", "peachpuff3", "peachpuff4", "peru", "pink1", "pink2", "pink3", "pink4", "plum", "plum1", "plum2", "plum3", "plum4", "powderblue", "purple", "purple1", "purple2", "purple3", "purple4"
    , "red1", "red2", "red3", "red4", "rosybrown", "rosybrown1", "rosybrown2", "rosybrown3", "rosybrown4", "royalblue", "royalblue1", "royalblue2", "royalblue3", "royalblue4", "saddlebrown", "salmon", "salmon1", "salmon2", "salmon3", "salmon4", "sandybrown", "seagreen", "seagreen1", "seagreen2", "seagreen3", "seagreen4", "seashell", "seashell1", "seashell2", "seashell3", "seashell4", "sienna", "sienna1", "sienna2", "sienna3", "sienna4", "skyblue", "skyblue1", "skyblue2", "skyblue3", "skyblue4", "slateblue", "slateblue1", "slateblue2", "slateblue3", "slateblue4", "slategray", "slategray1", "slategray2", "slategray3", "slategray4", "slategrey", "snow", "snow1", "snow2", "snow3", "snow4", "springgreen", "springgreen1", "springgreen2", "springgreen3", "springgreen4", "steelblue", "steelblue1", "steelblue2", "steelblue3", "steelblue4", "tan", "tan1", "tan2", "tan3", "tan4", "thistle", "thistle1", "thistle2", "thistle3", "thistle4", "tomato", "tomato1", "tomato2", "tomato3", "tomato4"
    , "turquoise", "turquoise1", "turquoise2", "turquoise3", "turquoise4", "violetred", "violetred1", "violetred2", "violetred3", "violetred4", "wheat", "wheat1", "wheat2", "wheat3", "wheat4", "whitesmoke", "yellow1", "yellow2", "yellow3", "yellow4", "yellowgreen"
  \end{itemize}
\item authorized line styles: "blank", "solid", "dashed", "dotted", "dotdash", "longdash",  "twodash"

\item authorized fill styles: "solid", "shaded"
\end{itemize}

The default values are the following ones:
\begin{itemize}
\item \textit{Color = "blue"}
\item \textit{SurfaceColor = "white"}
\item \textit{FillStyle = "solid"}
\item \textit{PointStyle = "plus"}
\item \textit{LineWidth = 1}
\item \textit{LineStyle = "solid"}
\item \textit{Pattern="s"}
\end{itemize}



% -------------------------------------------------
\newpage
% \index{Graphs!BarPlot}
\index{BarPlot}
\subsection{BarPlot}


It inherits from the methods of the Drawable class.

\begin{description}
\item[Usage:] \rule{0pt}{1em}
  \begin{description}
  \item \textit{BarPlot(data, origin, legend)}
  \item \textit{BarPlot(data, origin, color, fillStyle, lineStyle, legend)}
  \item \textit{BarPlot(data, origin, color, fillStyle, lineWidth, lineStyle, legend)}
  \end{description}

\item[Arguments:] \rule{0pt}{1em}
  \begin{description}
  \item \textit{data}: a NumericalSample, the data from which the BarPlot is built, must be of dimension 2: the discontinuous points and their corresponding height
  \item \textit{origin}: a real value which is where the BarPlot begins
  \item \textit{legend}: a String, the legend
  \item \textit{color}: a String, the color of the curve . If not specified, by default equal to "blue"
  \item \textit{lineStyle}: a String, the style of the curve. If not specified, by default equal to "solid"
  \item \textit{lineWidth}: an integer, the width of the curve. If not specified, by default equal to 1
  \item \textit{fillStyle}: a String, the fill style of the surfaces. If not specified, by default equal to "solid"
  \end{description}

\item[Some methods:]  \rule{0pt}{1em}

  \begin{description}

  \item \textit{getData}
    \begin{description}
    \item[Usage:] \textit{getData()}
    \item[Arguments:] none
    \item[Value:] a NumericalSample, of dimension 2, giving the discontinuous points and their corresponding height
    \end{description}
    \bigskip
  \item \textit{getOrigin}
    \begin{description}
    \item[Usage:] \textit{getOrigin()}
    \item[Arguments:] none
    \item[Value:] a real value which is where the BarPlot begins
    \end{description}
    \bigskip
  \item \textit{isConformData}
    \begin{description}
    \item[Usage:] \textit{isConformData(data)}
    \item[Arguments:] \textit{data}: a NumericalSample
    \item[Value:] a boolean which indicates if the type of data is conform to the type of the drawable (here a BarPlot): a NumericalSample, of dimension 2
    \end{description}
  \end{description}

  All the methods \textit{getColor},  \textit{getLegend}, \textit{getOrigin}, \textit{getFillStyle}  and \textit{getLineStyle} have their corresponding \textit{setMethod}.

\end{description}



% -------------------------------------------------
\newpage
% \index{Graphs!Cloud}
\index{Cloud}
\subsection{Cloud}


It inherits from the methods of the Drawable class.

\begin{description}
\item[Usage:] \rule{0pt}{1em}
  \begin{description}
  \item \textit{Cloud(data, legend)}
  \item \textit{Cloud(dataX, dataY, legend)}
  \item \textit{Cloud(data, color, pointStyle, legend)}
  \item \textit{Cloud(dataComplex, legend)}
  \end{description}

\item[Arguments:] \rule{0pt}{1em}
  \begin{description}
  \item \textit{data}: a NumericalSample, the points from which the cloud is built, must be of dimension 2
  \item \textit{dataX}, \textit{dataY}: two NumericalSamples of dimension 1, or two NumericaPoints
  \item \textit{legend}: a String, the legend
  \item \textit{color}: a String, the color of the points . If not specified, by default equal to "blue"
  \item \textit{pointStyle}: a String, the style of the points. If not specified, by default equal to "plus"
  \item \textit{dataComplex}: a NumericalComplexCollection, collection of complex points
  \end{description}

\item[Some methods:]  \rule{0pt}{1em}
  \begin{description}

  \item \textit{isValidData}
    \begin{description}
    \item[Usage:] \textit{isValidData(data)}
    \item[Arguments:] \textit{data}: a NumericalSample
    \item[Value:] a boolean which indicates if the type of data is conform to the type of the drawable : a NumericalSample of dimension 2
    \end{description}
    \bigskip
  \item \textit{getData}
    \begin{description}
    \item[Usage:] \textit{getData()}
    \item[Arguments:] none
    \item[Value:] a NumericalSample of dimension 2, the data from which the cloud is built
    \end{description}
  \end{description}

  All the methods \textit{getColor},  \textit{getLegend}, \textit{getPointCode} and \textit{getPointStyle} have their corresponding \textit{setMethod}.

\end{description}



% -------------------------------------------------
\newpage
% \index{Graphs!Contour}
\index{Contour}
\subsection{Contour}

It inherits from the methods of the Drawable class.

\begin{description}
\item[Usage:] \rule{0pt}{1em}
  \begin{description}
  \item \textit{Contour(dimX, dimY, data)}
  \item  \textit{Contour(dimX, dimY, data, legend)}
  \item  \textit{Contour(sampleX, sampleY, sampleValues, levels, labels)}
  \end{description}

\item[Arguments:] \rule{0pt}{1em}
  \begin{description}
  \item \textit{dimX}: an integer,
  \item \textit{dimY}: an integer,
  \item \textit{data}: a NumericalSample, of dimension 1 and of size $dimX*dimY$. These values are those of a function $f: \Rset^2  \longrightarrow \Rset$ on each point of the grid with \textit{dimX} points along the $X-$direction and \textit{dimX} points along the $Y-$direction. The $(X,Y)$-values are stocked row-by-row.
  \item \textit{legend}: a string which gives the legend.
  \item \textit{levels}: a NumericalPoint, the levels where the contour will be drawn. If 2 points of the grid have values bracheting the \textit{level}, a linear interpolation is made in order to find the point associated to the \textit{level} considered.
  \item \textit{labels}: a \textit{Description}, the labels of each curve associated to one \textit{level}. By default, the \textit{labels} are the values of the \textit{levels}.
  \end{description}

\item[Some methods:]  \rule{0pt}{1em}
  \begin{description}

  \item \textit{buildDefaultLevels}
    \begin{description}
    \item[Usage:] \textit{buildDefaultLevels(n)}
    \item[Arguments:] $n$: an integer, the number of levels. If not specified, the default value is taken in the RessourceMap and equal to $n=10$.
    \item[Value:] it builds $n$ level values and the associated labels which are the level values. The level values are the empirical quantiles of the data to be sliced at orders $q_k$ regularly distributed over $]0,1[$: $\displaystyle q_k = \frac{k+\frac{1}{2}}{n}$ for $0 \leq  k \leq n-1$.
    \end{description}
    \bigskip

  \item \textit{setLevels}
    \begin{description}
    \item[Usage:] \textit{setLevels(levels)}
    \item[Arguments:] \textit{levels}: a NumericalPoint, the different levels where the iso-curves wil be drawn
    \item[Value:] none
    \end{description}
    \bigskip

  \item \textit{setDrawLabels}
    \begin{description}
    \item[Usage:] \textit{setDrawLabels(bool)}
    \item[Arguments:] \textit{bool}: a Boolean, \textit{True} if the labels of the iso-curves must be explicited, \textit{False} otherwise.
    \item[Value:] none
    \end{description}
    \bigskip

  \item \textit{setLabels}
    \begin{description}
    \item[Usage:] \textit{setDrawLabels(labels)}
    \item[Arguments:] \textit{labels}: a \textit{Description}, the list of the labels
    \item[Value:] none
    \end{description}
    \bigskip
  \end{description}

  All the methods \textit{setMethod} have their corresponding \textit{getMethod}.

\end{description}


% -------------------------------------------------
\newpage
% \index{Graphs!Curve}
\index{Curve}
\subsection{Curve}

It inherits from the methods of the Drawable class.

\begin{description}
\item[Usage:] \rule{0pt}{1em}
  \begin{description}
  \item \textit{Curve(data, legend)}
  \item \textit{Curve(dataX, dataY, legend)}
  \item \textit{Curve(data, color, lineStyle, lineWidth, legend)}
  \end{description}

\item[Arguments:] \rule{0pt}{1em}
  \begin{description}
  \item \textit{data}: a NumericalSample, the points from which the curve is built, must be of dimension 2
  \item \textit{dataX}, \textit{dataY}:  two NumericalSample of dimension 1, or two NumericalPoints
  \item \textit{legend}: a String, the legend
  \item \textit{color}: a String, the color of the curve
  \item \textit{lineStyle}: a String, the style of the curve
  \item \textit{lineWidth}: an integer, the line width  of the curve
  \item \textit{showPoints}: a boolean which indicates whether the points that define the curve are drawn or not.
  \end{description}

\item[Some methods:]  \rule{0pt}{1em}
  \begin{description}

  \item \textit{isValidData}
    \begin{description}
    \item[Usage:] \textit{isValidData(data)}
    \item[Arguments:] \textit{data}: a NumericalSample
    \item[Value:] a boolean which indicates if the type of data is conform to the type of the drawable (here a Curve): a NumericalSample of dimension 2
    \end{description}
    \bigskip
  \item \textit{getData}
    \begin{description}
    \item[Usage:] \textit{getData()}
    \item[Arguments:] none
    \item[Value:] a NumericalSample of dimension 2, the data from which the curve is built
    \end{description}
    \bigskip
  \item \textit{getLineWidth}
    \begin{description}
    \item[Usage:] \textit{getLineWidth()}
    \item[Arguments:] none
    \item[Value:] an integer, the line width  of the curve
    \end{description}
  \end{description}

  All the methods \textit{getColor},  \textit{getLegend}, \textit{getLineStyle} and \textit{getLineWidth} have their corresponding \textit{setMethod}.

\end{description}

% -------------------------------------------------
\newpage
% \index{Graphs!Staircase}
\index{Staircase}
\subsection{Staircase}


It inherits from the methods of the Drawable class.

\begin{description}
\item[Usage:] \rule{0pt}{1em}
  \begin{description}
  \item \textit{StairCase(data, legend)}
  \item \textit{StairCase(data, color, lineStyle, pattern, legend)}
  \item \textit{StairCase(data, color, lineStyle, lineWidth, pattern, legend)}
  \end{description}

\item[Arguments:] \rule{0pt}{1em}
  \begin{description}
  \item \textit{data}: a NumericalSample, the points from which the Staircase is built, must be of dimension 2: the discontinuous points and their corresponding height
  \item \textit{legend}: a String, the legend
  \item \textit{color}: a String, the color of the curve If not specified, by default equal to "blue"
  \item \textit{lineStyle}: a String, the style of the curve. If not specified, by default equal to "solid"
  \item \textit{lineWidth}: an integer, the width of the curve. If not specified, by default equal to 1
  \item \textit{pattern}: a String, the pattern of the staircase which is "S" or "s". If not specified, by default equal to "s". Going from $(x_1,y_1)$ to $(x_2,y_2)$ with $x_1<x_2$,  \textit{pattern="s"} moves first horizontal then vertical, whereas  \textit{pattern="S"} moves the other way around.
  \end{description}

\item[Some methods:]  \rule{0pt}{1em}

  \begin{description}

  \item \textit{getData}
    \begin{description}
    \item[Usage:] \textit{getData()}
    \item[Arguments:] none
    \item[Value:] a NumericalSample, of dimension 2, giving the discontinuous points and their corresponding height
    \end{description}
    \bigskip
  \item \textit{isValidData}
    \begin{description}
    \item[Usage:] \textit{isValidData(data)}
    \item[Arguments:] \textit{data}: a NumericalSample
    \item[Value:] a boolean which indicates if the type of data is conform to the type of the drawable (here a Staircase): a NumericalSampl of dimension 2, giving the discontinuous points and their corresponding height
    \end{description}
    \bigskip
  \item \textit{getPattern}
    \begin{description}
    \item[Usage:] \textit{getPattern()}
    \item[Arguments:] none
    \item[Value:] a String, the pattern of the staircase.
    \end{description}
  \end{description}

  All the methods \textit{getColor},  \textit{getLegend},  \textit{getLineStyle}  and \textit{getPattern} have their corresponding \textit{setMethod}.

\end{description}




% -------------------------------------------------
\newpage
% \index{Graphs!Pairs}
\index{Pairs}
\subsection{Pairs}


It inherits from the methods of the Drawable class.

\begin{description}
\item[Usage:] \rule{0pt}{1em}
  \begin{description}
  \item \textit{Pairs(data, legend)}
  \item \textit{Pairs(data, legend, labels, color, pointStyle)}
  \end{description}

\item[Arguments:] \rule{0pt}{1em}
  \begin{description}
  \item \textit{data}: a NumericalSample of dimension $n$, the points from which 2d clouds are built
  \item \textit{legend}: a String, the legend
  \item \textit{labels}: a Description of dimension $n$ that gives the name of each component of the sample
  \item \textit{color}: a String, the color of the points. If not specified, by default equal to "blue"
  \item \textit{pointStyle}: a String, the style of the points. If not specified, by default equal to "plus"
  \end{description}

\item[Some methods:]  \rule{0pt}{1em}
  \begin{description}

  \item \textit{isValidData}
    \begin{description}
    \item[Usage:] \textit{isValidData(data)}
    \item[Arguments:] \textit{data}: a NumericalSample
    \item[Value:] a boolean which indicates if the type of data is conform to the type of the drawable : a NumericalSample of dimension $n\geq 2$
    \end{description}
    \bigskip
  \item \textit{getData}
    \begin{description}
    \item[Usage:] \textit{getData()}
    \item[Arguments:] none
    \item[Value:] a NumericalSample of dimension $n$, the data from which the clouds are built
    \end{description}
  \end{description}

  All the methods \textit{getColor},  \textit{getLegend}, \textit{getPointCode} and \textit{getPointStyle} have their corresponding \textit{setMethod}.

\end{description}




% -------------------------------------------------
\newpage
% \index{Graphs!Pie}
\index{Pie}
\subsection{Pie}

It inherits from the methods of the Drawable class.

\begin{description}
\item[Usage:] \rule{0pt}{1em}
  \begin{description}
  \item \textit{Pie(data)}
  \item \textit{Pie(data, labels)}
  \item \textit{Pie(data, labels,  center, radius, palette)}
  \end{description}

\item[Arguments:] \rule{0pt}{1em}
  \begin{description}
  \item \textit{data}: a NumericalPoint, giving the percentiles of the pie
  \item \textit{labels}: a StringCollection, the names of each group. If not specified, by default equal to the descrition of the probabilistic input vector
  \item \textit{center}: a NumericalPoint, the center of the pie inside the bounding box. If not specified, by default equal to $(0,0)$
  \item \textit{radius}: a real positive value, the radius of the pie. If not specified, by default equal to 1
  \item \textit{palette}: a StringCollection, the names of the colors. If not specified, colors are successively taken from the list given below, in the same order
  \end{description}

\item[Some methods:]  \rule{0pt}{1em}

  \begin{description}

  \item \textit{getCenter}
    \begin{description}
    \item[Usage:] \textit{getCenter()}
    \item[Arguments:] none
    \item[Value:] a NumericalPoint, the center of the pie inside the bounding box
    \end{description}
    \bigskip

  \item \textit{getData}
    \begin{description}
    \item[Usage:] \textit{getData()}
    \item[Arguments:] none
    \item[Value:] a NumericalSample of dimension 1, giving the percentiles of the pie
    \end{description}
    \bigskip
  \item \textit{getLabels}
    \begin{description}
    \item[Usage:] \textit{getLabels()}
    \item[Arguments:] none
    \item[Value:] a StringCollection, the names of each group
    \end{description}
    \bigskip
  \item \textit{getPalette}
    \begin{description}
    \item[Usage:] \textit{getPalette()}
    \item[Arguments:] none
    \item[Value:] a StringCollection, the names of the colors used for the pie
    \end{description}
    \bigskip
  \item \textit{getRadius}
    \begin{description}
    \item[Usage:] \textit{getRadius()}
    \item[Arguments:] none
    \item[Value:] a real positive value, the radius of the pie
    \end{description}
    \bigskip
  \item \textit{isValidData}
    \begin{description}
    \item[Usage:] \textit{isValidData(data)}
    \item[Arguments:] \textit{data}: a NumericalSample
    \item[Value:] a boolean which indicates if the type of data is conform to the type of the drawable (here a Pie): a NumericalSample of dimension 1
    \end{description}

    \bigskip
  \item \textit{getPattern}
    \begin{description}
    \item[Usage:] \textit{getPattern()}
    \item[Arguments:] none
    \item[Value:]
    \end{description}

  \end{description}

  All the methods \textit{getColor},  \textit{getLegend},  \textit{getLineStyle}  and \textit{getPattern} have their corresponding \textit{setMethod}.

\end{description}


\newpage
% \index{Graphs!View}
\index{View}
\subsection{View}

The class {\itshape View} enables to visualize Graphs with matplotlib, by writing files or drawing in a window.
\begin{description}
\item[Usage:] $View(graph, plot_kwargs=None, axes_kwargs=None, bar_kwargs=None, pie_kwargs=None,
  contour_kwargs=None, step_kwargs=None, clabel_kwargs, text_kwargs, legend_kwargs)$

\item[Arguments:] \textit{graph}: a Graph
\item[Arguments:] \textit{plot\_kwargs}: a dictionary
\item[Arguments:] \textit{axes\_kwargs}: a dictionary
\item[Arguments:] \textit{bar\_kwargs}: a dictionary
\item[Arguments:] \textit{pie\_kwargs}: a dictionary
\item[Arguments:] \textit{contour\_kwargs}: a dictionary
\item[Arguments:] \textit{step\_kwargs}: a dictionary
\item[Arguments:] \textit{clabel\_kwargs}: a dictionary
\item[Arguments:] \textit{text\_kwargs}: a dictionary
\item[Arguments:] \textit{legend\_kwargs}: a dictionary

\item[Value:] a View instance.
\item[Details:] For example, \textit{view = View(myDistribution.drawPDF())} where \textit{myDistribution} is a Distribution.
\end{description}

The method show enables to draw the graph in a window:
\begin{description}
\item[Usage:] \textit{show(**kwargs)}
\item[Keyword arguments:] \textit{block}: a boolean deciding whether the window is blocking(default) or non-blocking
\item[Details:] For example, \textit{view.show(block=False)}. Refer to the matplotlib documentation of matplotlib.pyplot.show
\item[Value:] None
\end{description}


The method save enables to write the graph in a file:
\begin{description}
\item[Usage:] \textit{save(fname, **kwargs)}
\item[Arguments:] \textit{fname}: a string, the name of the file. The extension enforces the format, if available.
\item[Keyword arguments:] Refer to the matplotlib documentation of matplotlib.pyplot.Figure.savefig
\item[Details:] For example, \textit{view.save('myFile.png')}
\item[Value:] None
\end{description}

The method getFigure enables to access the figure handle:
\begin{description}
\item[Usage:] \textit{getFigure()}
\item[Arguments:] None.
\item[Value:] A matplolib.pyplot.Figure
\end{description}





% -------------------------------------------------
\newpage
% \index{Graphs!Polygon}
\index{Polygon}
\subsection{Polygon}


It inherits from the methods of the Drawable class. A polygone is of dimension 2.

\begin{description}
\item[Usage:] \rule{0pt}{1em}
  \begin{description}
  \item \textit{Polygon(data, legend)}
  \item \textit{Curve(dataX, dataY, legend)}
  \item \textit{Curve(data, color, edgeColor, legend)}
  \end{description}

\item[Arguments:] \rule{0pt}{1em}
  \begin{description}
  \item \textit{data}: a NumericalSample of dimension 2, the vertices of the polygon
  \item \textit{dataX}, \textit{dataY}:  two NumericalSample of dimension 1, or two NumericalPoint: {\itshape  dataX} contains the list of the first coordinates of the vertices and {\itshape  dataY} the second ones.
  \item \textit{legend}: a String, the legend
  \item \textit{color}, \textit{edgeColor}: two String, the fill color and the edege color.
  \end{description}

\item[Value:] A polygon which is defined by the list of its vertices. The polygon is then fulfilled with the color {\itshape color} and its edges are colored in {\itshape corlorEdge}.
\item[Some methods:]  \rule{0pt}{1em}
  \begin{description}

  \item \textit{getEdgeColor}
    \begin{description}
    \item[Usage:] \textit{getEdgeColor()}
    \item[Arguments:] none
    \item[Value:] a  string the color of the edges
    \end{description}
    \bigskip

  \item \textit{getColor}
    \begin{description}
    \item[Usage:] \textit{getColor()}
    \item[Arguments:] none
    \item[Value:] a  string the fill color
    \end{description}

  \end{description}
\end{description}
