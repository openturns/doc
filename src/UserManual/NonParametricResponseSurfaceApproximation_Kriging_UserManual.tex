% Copyright (C) 2005-2015 Airbus - EDF - IMACS - Phimeca
% Permission is granted to copy, distribute and/or modify this document
% under the terms of the GNU Free Documentation License, Version 1.2
% or any later version published by the Free Software Foundation;
% with no Invariant Sections, no Front-Cover Texts, and no Back-Cover
% Texts.  A copy of the license is included in the section entitled "GNU
% Free Documentation License".

\newpage

\section{Response Surface : Kriging}
% \index{Response Surface : Kriging}
Kriging enables to approximate the output random variable of interest :
\begin{equation}\label{Model}
\underline{y} = g(\underline{x}) \in \Rset^p
\end{equation}
where $g : \Rset^n \longrightarrow \Rset^p $ is the model, $\underline{x}$ is the input random vector of dimension $n$ by the gaussian process :
\begin{equation}\label{metaModel}
\tilde{\underline{y}} = f(\underline{x}) ^t \underline{\beta} + Z(\underline{x})
\end{equation}
where $Z(\vect{x})$ is a zero-mean gaussian process with a stationary autocorrelation function depending on the distance between samples:
\begin{equation}
\mathbb{E}[Z(\underline{x}), Z(\underline{x'})] = \sigma^2 R(\underline{x} - \underline{x'}, \theta)
\end{equation}

\index{KrigingAlgorithm}
\subsection{KrigingAlgorithm}
\begin{description}
\item[Usage:] \rule{0pt}{1em}
\begin{description}
\item \textit{KrigingAlgorithm()}
\item \textit{KrigingAlgorithm(inputSample, outputSample, basis, covarianceModel)}
\item \textit{KrigingAlgorithm(inputSample, inputDistribution, outputSample, basis, covarianceModel)}
\end{description}
\bigskip

\item[Arguments:]  \rule{0pt}{1em}
\begin{description}
\item \textit{inputSample, outputSample}: NumericalSample, the input and output samples of a model evaluated apart
\item \textit{inputDistribution}: a Distribution, the joint probability density function of the physical input vector
\item \textit{basis}: a Basis, the basis of the output of the regression
\item \textit{covarianceModel}: a CovarianceModel, the correlation function
\end{description}

\item[Value:] a KrigingAlgorithm. If \textit{inputDistribution} is not specified, the \textit{inputSample} is assumed to be normally-distributed.

\item[Some methods :]  \rule{0pt}{1em}
\begin{description}

\item \textit{getOptimizer}
\begin{description}
\item[Usage:] \textit{getOptimizer()}
\item[Arguments:] none
\item[Value:] a BoundConstrainedAlgorithm, the solver used to optimize the covariance model parameters.
\end{description}
\bigskip

\item \textit{getConditionalCovarianceModel}
\begin{description}
\item[Usage:] \textit{getConditionalCovarianceModel()}
\item[Arguments:] none
\item[Value:] a CovarianceModel, the covariance model obtained after likelihood optimization when a solver is provided.
\end{description}
\bigskip

\item \textit{run}
\begin{description}
\item[Usage:] \textit{run()}
\item[Arguments:] none
\item[Value:] execute the procedure of determination of coefficients using the projection strategy selected with respect to the AdaptiveStrategy selected. It provides the results as an object of type FunctionalChaosResult.
\end{description}
\bigskip

\item \textit{getResult}
\begin{description}
\item[Usage:] \textit{getResult()}
\item[Arguments:] none
\item[Value:] a MetaModelResult, which contains all results of the execution.
\end{description}

\end{description}

\end{description}


The method \textit{getOptimizer} has its associated \textit{setOptimizer}.


%%%%%%%%%%%%%%%%%%%%%%%%%%%%%%

\newpage
\subsection{Construction of the regression basis}

% \index{Response Surface : Kriging!Construction of the multivariate orthogonal basis!OrthogonalUniVariatePolynomialFamily}
\index{BasisFactory}
\subsubsection{BasisFactory}


BasisFactory is the interface of the OrthogonalUniVariatePolynomialFactory implementation.
\begin{description}
\item[Usage:] \rule{0pt}{1em}
\begin{description}
\item \textit{BasisFactory()}
\item \textit{BasisFactory(dimension)}
\end{description}

\item[Arguments:]  \rule{0pt}{1em}
\begin{description}
\item \textit{orthogUniVarPolFactory}: an OrthogonalUniVariatePolynomialFactory that builds particular univariate polynomial (e.g. Hermite, Legendre, Laguerre, ...).
\end{description}

\item[Value:]  an OrthogonalUniVariatePolynomialFamily, represents the factory that allows the construction of any univariate orthonormal polynomial with any degree.

\item[Some methods :]  \rule{0pt}{1em}

\begin{description}

\item \textit{build}
\begin{description}
\item[Usage:] \textit{build()}
\item[Arguments:] None
\item[Value:] a Basis
\end{description}
\bigskip

\end{description}
\end{description}

%%%%%%%%%%%%%%%%%%%%%%%%%%%%%%%%%%%%%%%%%
\newpage
\index{ConstantBasisFactory}
\subsubsection{ConstantBasisFactory}

ConstantBasisFactory inherits from BasisFactory

\begin{description}
\item[Usage:] \rule{0pt}{1em}
\begin{description}
\item \textit{ConstantBasisFactory(dimension)}
\end{description}


\item[Arguments:] \rule{0pt}{1em}
\begin{description}
\item \textit{dimension}: an integer, the input dimension
\end{description}

\item[Description:] A factory for constant basis of input dimension \textit{dimension}.


\end{description}
%%%%%%%%%%%%%%%%%%%%%%
\index{LinearBasisFactory}
\subsubsection{LinearBasisFactory}

LinearBasisFactory inherits from BasisFactory

\begin{description}
\item[Usage:] \rule{0pt}{1em}
\begin{description}
\item \textit{LinearBasisFactory(dimension)}
\end{description}


\item[Arguments:] \rule{0pt}{1em}
\begin{description}
\item \textit{dimension}: an integer, the input dimension
\end{description}

\item[Description:] A factory for linear basis of input dimension \textit{dimension}.

\end{description}

%%%%%%%%%%%%%%%%%%%%%%%%%%
\index{QuadraticBasisFactory}
\subsubsection{QuadraticBasisFactory}

QuadraticBasisFactory inherits from BasisFactory

\begin{description}
\item[Usage:] \rule{0pt}{1em}
\begin{description}
\item \textit{QuadraticBasisFactory(dimension)}
\end{description}


\item[Arguments:] \rule{0pt}{1em}
\begin{description}
\item \textit{dimension}: an integer, the input dimension
\end{description}

\item[Description:] A factory for quadratic basis of input dimension \textit{dimension}.

\end{description}

%%%%%%%%%%%%%%%%%%%%%%%%%%%%%%

\newpage
\subsection{Construction of the stationary covariance function}

\index{SquaredExponential}
\subsubsection{SquaredExponential}

SquaredExponential inherits from CovarianceModel

\begin{description}
\item[Usage:] \rule{0pt}{1em}
\begin{description}
\item \textit{SquaredExponential(dimension)}
\item \textit{SquaredExponential(dimension, theta)}
\end{description}


\item[Arguments:] \rule{0pt}{1em}
\begin{description}
\item \textit{dimension}: an integer, the input dimension
\item \textit{theta}: a scalar, the coefficient $\theta$
\end{description}

\item[Description:] A covariance function of input dimension \textit{dimension}.
\begin{align*}
C(s, t) = \displaystyle e^{- \theta \left\Vert s - t \right\Vert_2}
\end{align*}

\end{description}

%%%%%%%%%%%%%%%%%%%%%%%%%%%%%%%%%%

\index{AbsoluteExponential}
\subsubsection{AbsoluteExponential}

AbsoluteExponential inherits from CovarianceModel

\begin{description}
\item[Usage:] \rule{0pt}{1em}
\begin{description}
\item \textit{AbsoluteExponential(dimension)}
\item \textit{AbsoluteExponential(dimension, theta)}
\end{description}

\item[Arguments:] \rule{0pt}{1em}
\begin{description}
\item \textit{dimension}: an integer, the input dimension
\item \textit{theta}: a scalar, the coefficient $\theta$
\end{description}

\item[Description:] A covariance function of input dimension \textit{dimension}.
\begin{align*}
C(s, t) = \displaystyle e^{- \theta \left\Vert s - t \right\Vert_1}
\end{align*}


\end{description}

%%%%%%%%%%%%%%%%%%%%%%%%%%%%%%%%%%

\index{GeneralizedExponential}
\subsubsection{GeneralizedExponential}

GeneralizedExponential inherits from CovarianceModel

\begin{description}
\item[Usage:] \rule{0pt}{1em}
\begin{description}
\item \textit{GeneralizedExponential(dimension)}
\item \textit{GeneralizedExponential(dimension, theta)}
\item \textit{GeneralizedExponential(dimension, theta, p)}
\end{description}

\item[Arguments:] \rule{0pt}{1em}
\begin{description}
\item \textit{dimension}: an integer, the input dimension
\item \textit{theta}: a scalar, the coefficient $\theta$
\item $p$ : a scalar, the exponent
\end{description}

\item[Description:] A covariance function of input dimension \textit{dimension}.
\begin{align*}
C(s, t) = \displaystyle e^{- \theta \left\Vert s - t \right\Vert_1^p}
\end{align*}
\end{description}
