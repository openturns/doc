% Copyright (C) 2005-2015 Airbus - EDF - IMACS - Phimeca
% Permission is granted to copy, distribute and/or modify this document
% under the terms of the GNU Free Documentation License, Version 1.2
% or any later version published by the Free Software Foundation;
% with no Invariant Sections, no Front-Cover Texts, and no Back-Cover
% Texts.  A copy of the license is included in the section entitled "GNU
% Free Documentation License".



\newpage
\section{Polynomials}



% ===================================================================

%\index{Polynomials!UniVariatePolynomial}
\index{UniVariatePolynomial}
\subsection{UniVariatePolynomial}

\begin{description}

\item[Usage:] \textit{UniVariatePolynomial(coefficients)}

\item[Arguments:]  \textit{coefficients}: a NumericalPoint, the list of the coefficients of each term $x^k$ (no sparse repesentation)

\item[Value:] a UniVariatePolynomial

\item[Some methods :]  \rule{0pt}{1em}

\begin{description}

\item $*$
\begin{description}
\item[Usage:]  \rule{0pt}{1em}
\begin{description}
\item \textit{Pol1*Pol2}
\item \textit{lambda*Pol1}
\end{description}
\item[Arguments:] \rule{0pt}{1em}
\begin{description}
\item $(Pol1, Pol2)$ : two UniVariatePolynomial,
\item \textit{lambda}: a NumericalScalar
\end{description}
\item[Value:] \rule{0pt}{1em}
\begin{description}
\item usage 1 : a UniVariatePolynomial, the result of the multiplication  $Pol1*Pol2$
\item usage 2 : a UniVariatePolynomial, the result of the multiplication  $lambda*Pol1$
\end{description}
\end{description}
\bigskip

\item $+$
\begin{description}
\item[Usage:]  $Pol1+Pol2$
\item[Arguments:] $(Pol1, Pol2)$ : two UniVariatePolynomial
\item[Value:] a UniVariatePolynomial, the result of the addition  $Pol1+Pol2$
\end{description}
\bigskip

\item $-$
\begin{description}
\item[Usage:]  $Pol1-Pol2$
\item[Arguments:] $(Pol1, Pol2)$ : two UniVariatePolynomial
\item[Value:] a UniVariatePolynomial, the result of the substraction  $Pol1-Pol2$
\end{description}
\bigskip

\item \textit{derivate}
\begin{description}
\item[Usage:] \textit{derivate()}
\item[Arguments:] none
\item[Value:] a UniVariatePolynomial, the derivated univariate polynomials
\end{description}
\bigskip

\item \textit{derivative}
\begin{description}
\item[Usage:] \textit{derivative(point)}
\item[Arguments:] \textit{point}: a NumericalScalar
\item[Value:] a NumericalScalar, the value of the derivated polynomials at point \textit{point}
\end{description}
\bigskip

\item \textit{draw}
\begin{description}
\item[Usage:] $draw(min, max, pointNumber)$
\item[Arguments:] \rule{0pt}{1em}
\begin{description}
\item $min, max$ : a NumericalScalar
\item \textit{pointNumber}: an integer, the number of points used for the grah
\end{description}
\item[Value:] a Graph, the polynomials curve on the range $[min, max]$.
\end{description}

\item \textit{getCoefficients}
\begin{description}
\item[Usage:] \textit{getCoefficients()}
\item[Arguments:] none
\item[Value:] a Coefficients, the coefficients of each $x^k$ for $k \leq$ to the degree of the univariate polynomials (no sparse repesentation)
\end{description}
\bigskip

\item \textit{getDegree}
\begin{description}
\item[Usage:] \textit{getDegree()}
\item[Arguments:] none
\item[Value:] an integer, the degree of the univariate polynomials
\end{description}
\bigskip

\item \textit{getRoots}
\begin{description}
\item[Usage:] \textit{getRoots()}
\item[Arguments:] none
\item[Value:] a NumericalComplexCollection, the collection of complex roots of the univariate polynomials
\end{description}
\bigskip

\item \textit{incrementDegree}
\begin{description}
\item[Usage:] \textit{incrementDegree(deg)}
\item[Arguments:] \textit{deg}: an integer
\item[Value:] a UniVariatePolynomial obtained by multiplying the polynomial by $x^{deg}$
\end{description}

\end{description}

\end{description}




% ===================================================================

\newpage
%\index{Polynomials!PolynomialCollection}
\index{PolynomialCollection}
\subsection{PolynomialCollection}

\begin{description}

\item[Usage:] $PolynomialCollection(size, univariatePol)$

\item[Arguments:]  \rule{0pt}{1em}
\begin{description}
\item \textit{size}: an integer
\item \textit{univariatePol}: a UniVariatePolynomial
\end{description}

\item[Value:] a PolynomialCollection, which contains \textit{size} polynomials each equal to  \textit{univariatePol}

\item[Some methods :]  \rule{0pt}{1em}

\item \textit{add}
\begin{description}
\item[Usage:]  $add(univariatePol)$
\item[Arguments:] \textit{univariatePol}: a UniVariatePolynomial,
\item[Value:] a PolynomialCollection which size has been increased of 1 and to which the polynomials \textit{univariatePol} has been added
\end{description}
\bigskip

\item $at$
\begin{description}
\item[Usage:]  $at(i)$
\item[Arguments:] $i$ : an integer
\item[Value:] a UniVariatePolynomial, the polynomials at position $i$ in the collection
\end{description}
\bigskip

\item \textit{resize}
\begin{description}
\item[Usage:]  $resize(newSize)$
\item[Arguments:] $i$ : an integer
\item[Value:] a PolynomialCollection which size has been modified into \textit{newSize} as follows : if $inewSize \leq getSize()$ then, the collection is truncated to the first \textit{newSize} polynomials. Otherwise, the collection is increased until the size \textit{newSize}: the added polynomials are the nul ones.
\end{description}


\end{description}



% ===================================================================

\newpage
%\index{Polynomials!ProductPolynomialEvaluationImplementation}
\index{ProductPolynomialEvaluationImplementation}
\subsection{ProductPolynomialEvaluationImplementation}

\begin{description}

\item[Usage:] \textit{ProductPolynomialEvaluationImplementation(polCollection)}

\item[Arguments:]  \textit{polCollection}: a PolynomialCollection, a collection of UniVariatePolynomial

\item[Value:] a ProductPolynomialEvaluationImplementation, the product of the polynomials of \textit{polCollection}. The result polynomials is of input dimension $n$ where $n$ is the number of polynomials in  \textit{polCollection}.

\item[Some methods :]  \rule{0pt}{1em}

\item \textit{Operator()}
\begin{description}
\item[Usage:]  $Operator(point)$
\item[Arguments:] \textit{point}: a NumericalPoint, which dimension is $n$ where $n$ is the number of polynomials in  \textit{polCollection}
\item[Value:] a NumericalPoint of dimension 1,
\end{description}
\bigskip

\item $+$
\begin{description}
\item[Usage:]  $Pol1+Pol2$
\item[Arguments:] $(Pol1, Pol2)$ : two UniVariatePolynomial
\item[Value:] a UniVariatePolynomial, the result of the addition  $Pol1+Pol2$
\end{description}


\item[Details:]  The exact gradient and hessian evaluations have been implemented for the products of polynomials.

\end{description}
