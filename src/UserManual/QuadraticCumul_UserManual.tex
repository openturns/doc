\newpage
\extanchor{taylorDecomposition}
\section{Taylor decomposition of the limit state function}

%\index{Taylor decomposition of the limit state function!QuadraticCumul}
\index{QuadraticCumul}
\subsection{QuadraticCumul}
\begin{description}

\item[Usage:]  \textit{QuadraticCumul(randVect)}

\item[Arguments:]  \textit{randVect}: a RandomVector, constraint : this RandomVector must be of type Composite, which means it must have been defined with the second usage of declaration of a RandomVector (from a NumericalMathFunction and an antecedent Distribution)

\item[Value:] a QuadraticCumul

\item[Some methods :]  \rule{0pt}{1em}

\begin{description}

\item \textit{drawImportanceFactors}
\begin{description}
\item[Usage:] \textit{drawImportanceFactors()}
\item[Arguments:] none
\item[Value:] a Graph, the structure containing the pie corresponding to the importance factors of the probabilistic variables
\end{description}
\bigskip

\item \textit{getCovariance}
\begin{description}
\item[Usage:] \textit{getCovariance()}
\item[Arguments:] none
\item[Value:] a CovarianceMatrix, approximation of first order of the covariance matrix of the random vector
\end{description}
\bigskip

\item \textit{getImportanceFactors}
\begin{description}
\item[Usage:] \textit{getImportanceFactors()}
\item[Arguments:] none
\item[Value:] a NumericalPoint, the importance factors of the inputs : only when \textit{randVect} is of dimension 1
\end{description}

\bigskip
\item \textit{getMeanFirstOrder}
\begin{description}
\item[Usage:] \textit{getMeanFirstOrder()}
\item[Arguments:] none
\item[Value:] a NumericalPoint, approximation at the first order of the mean of the random vector
\end{description}
\bigskip


\item \textit{getMeanSecondOrder}
\begin{description}
\item[Usage:] \textit{getMeanSecondOrder()}
\item[Arguments:] none
\item[Value:] a NumericalPoint, approximation at the second order of the mean of the random vector (it requires that the hessian of the NumericalMathFunction has been defined)
\end{description}
\bigskip

\item \textit{getValueAtMean}
\begin{description}
\item[Usage:] \textit{getValueAtMean()}
\item[Arguments:] none
\item[Value:] a NumericalPoint, the value of the NumericalMathFunction which defines the random vector at the mean point of the input random vector
\end{description}
\bigskip

\item \textit{getGradientAtMean}
\begin{description}
\item[Usage:] \textit{getGradientAtMean()}
\item[Arguments:] none
\item[Value:] a Matrix, the gradient of the NumericalMathFunction which defines the random vector at the mean point of the input random vector
\end{description}
\bigskip

\item \textit{getHessianAtMean}
\begin{description}
\item[Usage:] \textit{getHessianAtMean()}
\item[Arguments:] none
\item[Value:] a SymmetricTensor, the hessian of the NumericalMathFunction which defines the random vector at the mean point of the input random vector
\end{description}

\end{description}

\end{description}
